\documentclass{article}
\usepackage[a4paper,margin=1.875in,top=1.875in,bottom=1.875in]{geometry}
\usepackage{amsmath,mathtools,bbm,amssymb,stmaryrd,wasysym}
\usepackage{mathrsfs}

\usepackage{setspace}
\doublespacing

\usepackage{fancyhdr}
\renewcommand{\headrulewidth}{0pt} 
\pagestyle{fancy}
\lhead{Sheet 2 Evgenij}\rhead{Page \thepage}
\fancyfoot{}

\usepackage{tikz}
\usetikzlibrary{decorations.pathreplacing,arrows}

\usepackage[numbers]{natbib}
\bibliographystyle{alphadin}
\usepackage{url}
\usepackage{hyperref}

\begin{document}

\paragraph{Exercise 1 \textnormal{(4 Points)}.}
Let $(X_1,Y_1)$ and $(X_2,Y_2)$ be independent vectors of absolutely continuous random variables with joint distribution functions $F_{X_1,Y_1}$ and $F_{X_2,Y_2}$ respectively, with common margins $F_X=F_{X_1}=F_{X_2}$ and $F_Y=F_{Y_1}=F_{Y_2}$.
Let $C_1$ and $C_2$ denote the copulas of $(X_1,Y_1)$ and $(X_2,Y_2)$, respectively.
Show that
\begin{multline}
  P[(X_1-X_2)(Y_1-Y_2)>0]-P[(X_1-X_2)(Y_1-Y_2)<0]\\
  =4\int_{[0,1]^2}C_2(u,v)dC_1(u,v)-1=:Q(C_1,C_2)\,.\label{eq:q}
\end{multline}

In order to ensure that nothing goes wrong we follow the proof of theorem 5.1.1 from \cite{nelsen2006introduction}.
Because all random variables are continuous, points have no mass so $P[(X_1-X_2)(Y_1-Y_2)<0]=1-P[(X_1-X_2)(Y_1-Y_2)>0]$ and thus it is sufficient to show that $4\int_{[0,1]^2}C_2(u,v)dC_1(u,v)-1=2P[(X_1-X_2)(Y_1-Y_2)>0]-1$ or that $P[(X_1-X_2)(Y_1-Y_2)>0]=2\int_{[0,1]^2}C_2(u,v)dC_1(u,v)$.
We have $\omega\in\{(X_1-X_2)(Y_1-Y_2)>0\}$ iff either $X_1(\omega)>X_2(\omega)$ and $Y_1(\omega)>Y_2(\omega)$ or $X_1(\omega)<X_2(\omega)$ and $Y_1(\omega)<Y_2(\omega)$, so that $P[(X_1-X_2)(Y_1-Y_2)>0]=P({X_1>X_2},{Y_1>Y_2})+P({X_1<X_2},{Y_1<Y_2})$.
We will now show that each $P({X_1>X_2},{Y_1>Y_2})=P({X_1<X_2},{Y_1<Y_2})=\int_{[0,1]^2}C_2(u,v)dC_1(u,v)$.
We have
\begin{align*}
  P({X_1>X_2},{Y_1>Y_2})
  &=P({X_2<X_1},{Y_2<Y_1})\,,
    \intertext{and by definition of $F_{X_1,Y_1}$}
  &=\int_{\mathbb{R}^2}P({X_2\leq x},{Y_2\leq y})dF_{X_1,Y_1}(x,y)\,.
    \intertext{By definition of $C_1$ and $C_2$ and Sklar's theorem we get}
  &=\int_{\mathbb{R}^2}C_2(F_X(x),F_Y(y))dC_1(F_X(x),F_Y(y))\,,
    \intertext{and by choosing $u=F_X(x)$ and $v=F_Y(y)$}
  &=\int_{[0,1]^2}C_2(u,v)dC_1(u,v)\,.
\end{align*}
Analogously we can write
\begin{align*}
  P({X_1<X_2},{Y_1<Y_2})
  &=\int_{\mathbb{R}^2}P({X_2\geq x},{Y_2\geq y})dF_{X_1,Y_1}(x,y)\,.
    \intertext{Now since $P(A\cap B)=P(A)+P(B)-P(A\cup B)=1-P(A^C)-P(B^C)+P(A^C\cap B^C)$ we get}
  &=\int_{\mathbb{R}^2}1-F_X(x)-F_Y(y)+C_2(F_X(x),F_Y(y))\,.
    \intertext{again by definition of $C_1$ and $C_2$, Sklar's theorem, and by substituting $u=F_X(x)$ and $v=F_Y(y)$ we obtain}
  &=\int_{[0,1]^2}1-u-v+C_2(u,v)dC_1(u,v)\,.
    \intertext{Since $C_1(u,v)=P({U\leq u},{V\leq v})$ with $U\sim V\sim\operatorname{U}(0,1)$, we have $\int_{[0,1]^2}dC_1(u,v)=1$ and $\int_{[0,1]^2}udC_1(u,v)=\int_{[0,1]^2}vdC_1(u,v)=\frac{1}{2}$, so that also in this case we end up with}
  &=\int_{[0,1]^2}C_2(u,v)dC_1(u,v)\,.
\end{align*}
\pagebreak
\paragraph{Exercise 2 \textnormal{(4 Points)}.}
Let $(X,Y)$ be a vector of absolutely continuous random variables.
Use the definition of $Q$ in (\ref{eq:q}) to show that
\begin{itemize}
\item[1.]the population version of Kendall's $\tau$ is given by $\tau(X,Y)=Q(C,C)$ if $C$ is a copula for $(X,Y)$ and
\end{itemize}

this can be directly seen from (\ref{eq:q}) by setting $C_1=C_2=C$.

\begin{itemize}
\item [2.]
  the population version of Spearman's $\rho$ is given by $\rho(X,Y)=3Q(C,\Pi)$, if $C$ is a copula for $(X,Y)$ and $\Pi$ describes the independence copula $\Pi(u,v)=uv$.
\end{itemize}

Spearman's $\rho$ is given by
\begin{align*}
  \frac{1}{3}\rho(X,Y)
  &=P[(X_1-X_2)(Y_1-Y_3)>0]-P[(X_1-X_2)(Y_1-Y_3)<0]\,,
    \intertext{where $(X_1,Y_1)$, $(X_2,Y_2)$ and $(X_3,Y_3)$ are three independent vectors with joint distribution function $F_{X,Y}$.
    In analogy to exercise 1 we can write}
  &=4\int_{[0,1]^2}C_2(u,v)dC_1(u,v)-1\,,
    \intertext{where $C_1=C$ is again the copula of $(X_1,Y_1)$, but $C_2$ is now the copula of $(X_2,Y_3)$, because in the formula now there is $Y_3$ where $Y_2$ was.
    Since $(X_2,Y_2)$ and $(X_3,Y_3)$ are independent, $C_2=\Pi$ so that we get}
  &=Q(C,\Pi)\,.
\end{align*}
\pagebreak
\paragraph{Exercise 3 \textnormal{(4 Points)}.}
Show that $3Q(C,\Pi)=12\int_0^1\int_0^1(C(u,v)-uv)dudv$.

Since $Q$ is symmetric in its arguments, \emph{which still remains to be shown,} we can equivalently write $\rho(X,Y)=12\int_0^1\int_0^1C(u,v)dudv-3$.
The result then follows $\int_0^1\int_0^1uvdudv=\frac{1}{4}$.

Then show that Spearman's $\rho$ is a measure of concordance.

To this end we have to show that
\begin{itemize}
\item [1.] $\rho$ is defined for every pair $X,Y$ of continuous random variables
\end{itemize}
which is true by definition.
\begin{itemize}
\item [2.1.] $-1\leq \rho \leq 1$ 
\end{itemize}
which is true, because $0\leq P[(X_1-X_2)(Y_1-Y_3)>0]\leq1$ and $0\leq P[(X_1-X_2)(Y_1-Y_3)<0]\leq1$.
\begin{itemize}
\item [2.2.] $\rho(X,X)=1$ 
\end{itemize}
which is true, because then $C(u,v)=\min(u,v)$ so that $3Q(C,\Pi)=12\int_0^1\int_0^1(\min(u,v)-uv)dudv=1$.
\begin{itemize}
\item [3.] $\rho(X,Y)=\rho(Y,X)$
\end{itemize}
which follows from $Q$ being symmetric in its arguments.
\begin{itemize}
\item [4.] If $X$ and $Y$ are independent, then $\rho(X,Y)=0$
\end{itemize}
which follows from the fact that then $C(u,v)=uv$ so that $C(u,v)-uv=0$.
\begin{itemize}
\item [6.] If $C_1$ and $C_2$ are copulas for $(X_1,Y_1)$ and $(X_2,Y_2)$ such that $C_1\prec C_2$ then $\rho(X_1,Y_1)\leq \rho(X_2,Y_2)$,
\end{itemize}
which follows from the monotony of the integral.
\begin{itemize}
\item [7.] If $\{(X_n,Y_n)\}$ is a sequence of continuous random variables with copulas $C_n$, and if $\{C_n\}$ converges pointwise to $C$, the copula for $(X,Y)$ then $\lim_{n\to\infty}\rho(X_n,Y_n)=\rho(X,Y)$.
\end{itemize}
Because all the copulas are quasi-copulas, they fulfill $|C_n(u,v)|\leq|u|+|v|$ and $|C(u,v)|\leq|u|+|v|$, so that for all $\varepsilon>0$ we can find an $n\in\mathbb{N}$ so that $|C_n(u,v)-C(u,v)|<\varepsilon$, \emph{where however I don't know how.}
That means $C_n$ is uniformly convergent towards $C$ and we get $\lim_{n\to\infty}\int_0^1\int_0^1C_n(u,v)dudv=\int_0^1\int_0^1C(u,v)dudv$ as desired.
\bibliography{../../../books/wt}
\end{document}

%%% Local Variables:
%%% mode: latex
%%% TeX-master: t
%%% End:
