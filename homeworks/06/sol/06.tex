\documentclass{article}
\usepackage[a4paper,margin=1.875in,top=1.875in,bottom=1.875in]{geometry}

\usepackage{amsmath,mathtools,bbm,amssymb}
\usepackage[german]{babel}

\usepackage{setspace}
\doublespacing

\usepackage{fancyhdr}
\renewcommand{\headrulewidth}{0pt} 
\pagestyle{fancy}
\lhead{Blatt 6 Nicolas und Evgenij}\rhead{Seite \thepage}
\fancyfoot{}

\usepackage{tikz}
\usetikzlibrary{decorations.pathreplacing,arrows.meta}

\usepackage[numbers]{natbib}
\bibliographystyle{alphadin}
\usepackage{url}
\usepackage{hyperref}

\begin{document}

\paragraph{B6A1}
Seien $X_1,X_2,\dots$ \emph{iid} uniform auf $[0,1]$ verteilt.
Weiter sei $f\in L^1([0,1])$.
Zeigen Sie, dass die Monte-Carlo Simulation $\hat{I}_n:=\frac{1}{n}\sum_{i=1}^nf(X_i)$ fast sicher gegen das Integral $\int_0^1f(x)\mathrm{d}x$ konvergiert.

\emph{Siehe hierzu, was wir in der Letzten Vorlesung dazu hatten. (24.5. Minute 23)}
Es gilt
\begin{align*}
  \int f(x)\mathrm{d}F(x)
  &=\int f(x)\mathrm{d}F_n(x)\\
  &=\int f(x)\mathrm{d}\mu(x)\\
  &=\sum\mathrm{d}\Bigl(\frac{1}{n}\sum_{i=1}^n\delta_{X_i}\Bigr)\\
  &=\frac{1}{n}\sum_{i=1}^n f(X_i)
\end{align*}

\newpage

\paragraph{B6A2}
Für jedes $n\in\mathbb{N}$ seien $X_1^{(n)},\dots,X_n^{(n)}$ paarweise unkorrelierte Zufallsvariablen mit endlicher Varianz (also nicht notwendig identisch verteilt) und
\[
\lim_{n\to\infty}\frac{1}{n^2}\sum_{i=1}^n\operatorname{Var}\bigl[X_i^{(n)}\bigr]=0\,.
\]
Zeigen Sie, dass die $X_i^{(n)}$ dem schwachen Gesetz der großen Zahlen genügen, d.h. beweisen Sie
\[
\frac{1}{n}\sum_{i=1}^n\bigl(X_i^{(n)}-E\bigl[X_i^{(n)}\bigr]\bigr)\xrightarrow{P}0,\quad n\to\infty\,.
\]

Es sei $(X_n)_{n\geq2}$ eine Folge unabhängiger Zufallsvariablen mit
\[
P(X_n=n)=\frac{1}{n\log n}\quad\text{und}\quad P(X_n=0)=1-\frac{1}{n\log n}\,.
\]
Zeigen Sie, dass die Folge dem schwachen Gesetz der großen Zahlen genügt, in dem Sinne, dass
\[
\frac{1}{n}\sum_{i=2}^n(X_i-E[X_i])\xrightarrow{P}0\,.
\]
Zeigen Sie weiter, dass die obige Folge nicht fast sicher konvergiert und sie somit nicht dem Gesetz der großen Zahlen genügt.
Verwenden Sie dazu das Lemma von Borel--Cantelli.
\newpage

\paragraph{B6A3}

\newpage

\paragraph{B6A4}

\newpage


\bibliography{../../../books/wt}
\end{document}

%%% Local Variables:
%%% mode: latex
%%% ispell-local-dictionary: "german"
%%% TeX-master: t
%%% End:
