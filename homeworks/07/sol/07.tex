\documentclass{article}
\usepackage[a4paper,margin=1.875in,top=1.875in,bottom=1.875in]{geometry}

\usepackage{amsmath,mathtools,bbm,amssymb}
\usepackage[german]{babel}

\usepackage{setspace}
\doublespacing

\usepackage{fancyhdr}
\renewcommand{\headrulewidth}{0pt} 
\pagestyle{fancy}
\lhead{Blatt 7 Nicolas und Evgenij}\rhead{Seite \thepage}
\fancyfoot{}

\usepackage{tikz}
\usetikzlibrary{decorations.pathreplacing,arrows.meta}

\usepackage[numbers]{natbib}
\bibliographystyle{alphadin}
\usepackage{url}
\usepackage{hyperref}

\begin{document}

Im Folgenden sei $(\Omega,{\cal A},P)$ ein Wahrscheinlichkeitsraum und $X\in L^1(\Omega,{\cal A},P)$.

\emph{Definition 1.} Sei ${\cal F}\subset{\cal A}$ eine $\sigma$-Algebra.
Eine Zufallsvariable $Y$ heißt bedingte Erwartung von $X$ gegeben ${\cal F}$, symbolisch $E[X|{\cal F}]:=Y$, falls gilt:
\begin{itemize}
\item[i)] $Y$ ist ${\cal F}$-messbar.
\item[ii)] Für jedes $A\in{\cal F}$ gilt $E[X\mathbbm{1}_A]=E[Y\mathbbm{1}_A]$
\end{itemize}
\paragraph{B7A1}
Zeigen Sie, $E[X,{\cal F}]$ existiert und ist eindeutig (bis auf Gleichheit fast sicher).
Gehen Sie dabei wie folgt vor:
\begin{itemize}
\item[i)] Eindeutigkeit: Nehmen Sie an, dass $Y$ und $Y'$ Definition 1 erfüllen und betrachten Sie die Menge $A:=\{Y-Y'>0\}$.
\item[ii)] Existenz: Definieren Sie das Maß $Q^+$ auf $(\Omega,{\cal F})$ durch $Q^+[A]:=E[\mathbbm{1}_AX^+]$ und analog $Q^-$.
  Konstruieren Sie nun die bedingte Erwartung mit dem Satz von Radon--Nikodym.
\end{itemize}
\newpage

\paragraph{B7A2}
Welche der folgenden Teilmengen des Raumes der reellen Folgen
\[
\mathbb{R}^\mathbb{N}=\bigtimes_{i\in\mathbb{N}}\mathbb{R}
\]
sind messbar bezüglich ${\cal B}^\mathbb{N}:=\bigotimes_{i\in\mathbb{N}}{\cal B}(\mathbb{R})$?
\begin{alignat*}{2}
  &\text{(a)}~&&\Bigl\{(x_n)_{n\in\mathbb{N}}\in\mathbb{R}^{\mathbb{N}}\Bigm|\sup_{n\in\mathbb{N}}x_n>3\Bigr\}\\
  &\text{(b)}~&&\Bigl\{(x_n)_{n\in\mathbb{N}}\in\mathbb{R}^{\mathbb{N}}\Bigm|\sum_{k=1}^nx_k=0\text{ für mindestens ein $n\in\mathbb{N}$}\Bigr\}\\
  &\text{(c)}~&&\Bigl\{(x_n)_{n\in\mathbb{N}}\in\mathbb{R}^{\mathbb{N}}\Bigm|(x_n)_{n\in\mathbb{N}}\text{ konvergiert gegen 3}\Bigr\}
\end{alignat*}

Generell ist eine Menge $A$ genau dann messbar bezüglich $\bigotimes_{\in\in\mathbb{N}}\mathcal{B}(\mathbb{R})$, wenn $A\in\bigotimes_{\in\in\mathbb{N}}\mathcal{B}(\mathbb{R})$.
Hierbei gilt $\bigotimes_{\in\in\mathbb{N}}\mathcal{B}(\mathbb{R})=\sigma\bigl(A_J\times\Omega_{\mathbb{N}\setminus J}\mid J=\{j_1,\dots,j_n\},A_{j_k}\in\mathcal{B}(\mathbb{R})\bigr)$.
\newpage

\paragraph{B7A3}
Sei $(\Omega_i,\mathcal{A}_i)=\bigl([0,1],\mathcal{B}([0,1])\bigr)$ für $i=1,2$ und $(\Omega,\mathcal{A})=(\Omega_1\times\Omega_2,\mathcal{A}_1\otimes\mathcal{A}_2)$ der Produktraum.
\begin{enumerate}
\item[(a)] Geben Sie ein Beispiel für eine Menge
  $A\subset\Omega$, für die für alle $\omega_i\in[0,1]$ der
  $\omega_i$-Schnitt $A_{\omega_i}\in\mathcal{A}_j$ ist (für
  $i,j=1,2$ und $i\neq j$), aber $A\in\mathcal{A} gilt$.

  \emph{Hinweis: Der $\omega_1$-Schnitt der Menge $A$ ist definiert als $A_{\omega_1}=\bigl(\{\omega_1\}\times\Omega_2\bigr)\cap A=\{(\omega_1,\omega_j)\in A\}$ und der $\omega_2$-Schnitt analog.}
  
\item[(b)] Sie $D=\{(x,x)\mid x\in[0,1]\}$ die Diagonale in $\Omega$, $\lambda$ das Lebesguemaß auf $\Omega_1$ und $\mu$ das Zählmaß auf $\Omega_2$, das heißt
  \[
    \mu(A)=
    \begin{cases}
      |A|,&\text{falls $A$ endlich ist,}\\
      \infty&\text{sonst.}
    \end{cases}
  \]
  Zeigen Sie $D\in\mathcal{A}$ und berechnen Sie
  \[
    \int_{\Omega_2}\int_{\Omega_1}\mathbbm{1}_{D}(x,y)\mathrm{d}\lambda(x)\mathrm{d}\mu(y)\quad\text{und}\int_{\Omega_1}\int_{\Omega_2}\mathbbm{1}_{D}(x,y)\mathrm{d}\mu(y)\mathrm{d}\lambda(x)\,.
  \]
\item[(c)] Ist das Ergebnis in Teil (b) ein Widerspruch zum Satz von Fubini?
\end{enumerate}
\newpage

\paragraph{B7A4}
Beweisen Sie mit dem Satz von Fubini die Regel der partiellen Integration.
Seien $f,g\colon [a,b]\to\mathbb{R}$ zwei Lebesgue-integrierbare Funktionen und für $x\in[a,b]$ seien
\[
  F(x):=\int_a^xf(y)\mathrm{d}y\quad\text{und}\quad G(x):=\int_a^xg(y)\mathrm{d}y\,.
\]
Dann gilt
\[
  \int_a^bF(x)g(x)\mathrm{d}x=F(b)G(b)-\int_a^bG(x)f(x)\mathrm{d}x\,.
\]
\emph{Hinweis: Wenden Sie den Satz von Fubini auf die Funktion $h\colon (x,y)\mapsto f(y)g(x)\mathbbm{1}_E(x,y)$ an, mit $E=\{(x,y)\in[a,b]^2:y<x\}$.}
Der Satz von Fubini lautet, 
\[
  \int h\mathrm{d}(\lambda\otimes\kappa_1)=\int\Bigl(\int h(x,y)\kappa_1(x,\mathrm{d}y)\Bigr)\lambda(\mathrm{d}x)
\]
\emph{Hier ist noch unklar, was das Produkt der Übergangskern ist.}
\newpage


\bibliography{../../../books/wt}
\end{document}

%%% Local Variables:
%%% mode: latex
%%% ispell-local-dictionary: "german"
%%% TeX-master: t
%%% End:
