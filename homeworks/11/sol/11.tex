\documentclass{article}
\usepackage[a4paper,margin=1.875in,top=1.5in,bottom=1.5in]{geometry}

\usepackage{amsmath,mathtools,bbm,amssymb}
\usepackage{mathrsfs}
\usepackage[german]{babel}

\usepackage{setspace}
\doublespacing

\usepackage{fancyhdr}
\renewcommand{\headrulewidth}{0pt} 
\pagestyle{fancy}
\lhead{Blatt 11 Nicolas und Evgenij}\rhead{Seite \thepage}
\fancyfoot{}

\usepackage{tikz}
\usetikzlibrary{decorations.pathreplacing,arrows}

\usepackage[numbers]{natbib}
\bibliographystyle{alphadin}
\usepackage{url}
\usepackage{hyperref}

\begin{document}

\paragraph{Aufgabe 1.}
Sei $(\Omega,\mathscr{F},P)$ ein Wahrscheinlichkeitsraum und $\mathscr{A}=\{\emptyset,\Omega\}$ die triviale $\sigma$-Algebra.
Zeigen Sie $E[X|\mathscr{A}]=E[X]$ für alle $X\in L^1(\Omega,\mathscr{F},P)$.

Zunächst einmal ist $E[X]$ als Zufallsvariable die Abbildung auf irgendeine konstante Zahl.
Das Urbild davon ist also ganz $\Omega$.
Somit ist $E[X]$ $\mathscr{A}$-messbar.
Weiterhin gilt $\int_{\Omega}E[X]\mathrm{d}P=E[X]\int_{\Omega}\mathrm{d}P=E[X]\cdot1=\int_\Omega X\mathrm{d}P$, sodass $E[X|\mathscr{A}]=E[X]$ gilt.
\newpage

\paragraph{Aufgabe 2.} Zeigen Sie die folgenden Aussagen
\begin{enumerate}
\item[i)] Ist $(X_i)_{i\in I}$ gleichgradig integrierbar und $(\mathscr{F}_j)_{j\in J}$ eine Familie von Unter-$\sigma$-Algebren von $\mathscr{A}$.
  Dann ist die Familie $(E[X_i,\mathscr{F}_j])_{i\in I,j\in J}$ gleichgradig integrierbar.
\end{enumerate}
% https://planetmath.org/conditionalexpectationsareuniformlyintegrable

Nach Lemma 24 gilt, wenn $(X_i)$ gleichgradig integrierbar ist, äquivalent $\sup_{i\in I}E[|X_i|]<\infty$ und $\lim_{\delta\to 0}\sup_{A:P(A)<\varepsilon}\sup_{i\in I}E[|X_i|\mathbbm{1}_A]=0$.
Man kann auch sagen, dass für alle $\varepsilon>0$ ein $\delta>0$ existiert, sodass für alle $i\in I$ und alle $A\in\mathscr{A}$ mit $P(A)\leq\delta$ gilt $E[|X_i|\mathbbm{1}_A]<\varepsilon$.
Wir schreiben kurz $Y_{ij}=E[X_i\mid\mathscr{F}_j]$.
Sei nun $k=\sup_{i\in I}E[|X_i|]/\delta$.
Mit der Markov-Ungleichung kriegen wir nun
\begin{align*}
  P(|Y_{ij}|>k)
  &\leq\frac{E[|Y_{ij}|]}{k}\,.
    \intertext{Für alle $i\in I$ und $j\in J$ gilt mit der Jensen'schen Ungleichung des bedingten Erwartungswertes, dass $|Y_{ij}|\leq E[|X_i|\mid\mathscr{F}_j]$, also}
  &\leq\frac{E[|X_i|]}{k}\,.
    \intertext{Da, wie erwähnt, $(X_i)$ beschränkt in $L^1$ ist, gilt}
  &\leq\sup_{i\in I}\frac{E[|X_i|]}{k}
   \intertext{und nach der Wahl des $\delta$ schließlich} 
  &\leq\delta\,.                %
\end{align*}
Damit folgt, wie oben erklärt, $E[|Y_{ij}|\mathbbm{1}_{|Y_{ij}|>k}]\leq E[|X_i|\mathbbm{1}_{|Y_{ij}|>k}]<\varepsilon$, also ist $(E[X_i,\mathscr{F}_j])_{ij}$ gleichgradig integrierbar.
\begin{enumerate}
\item[ii)] Ist $X\in L^1(\Omega,\mathscr{A},P)$, dann ist $(E[X|\mathscr{F}_j])_{j\in J}$ gleichgradig integrierbar.
\end{enumerate}
%https://math.stackexchange.com/questions/3731917/uniform-integrability-of-all-conditional-expectations-of-a-fixed-l1-function
Nach Beispiel 23.i ist, wenn $X\in L^1(\Omega,\mathscr{A},P)$ ist, die Folge $(X)$ gleichgradig integrierbar, sodass $(E[X|\mathscr{F}_j])_{j\in J}$ nach Teilaufgabe (i) gleichgradig integrierbar ist.
\newpage

\bibliography{../../../books/wt}
\end{document}

%%% Local Variables:
%%% mode: latex
%%% ispell-local-dictionary: "german"
%%% TeX-master: t
%%% End:
