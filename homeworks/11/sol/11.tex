\documentclass{article}
\usepackage[a4paper,margin=1.875in,top=1.5in,bottom=1.5in]{geometry}

\usepackage{amsmath,mathtools,bbm,amssymb}
\usepackage{mathrsfs}
\usepackage[german]{babel}

\usepackage{setspace}
\doublespacing

\usepackage{fancyhdr}
\renewcommand{\headrulewidth}{0pt} 
\pagestyle{fancy}
\lhead{Blatt 11 Nicolas und Evgenij}\rhead{Seite \thepage}
\fancyfoot{}

\usepackage{tikz}
\usetikzlibrary{decorations.pathreplacing,arrows}

\usepackage[numbers]{natbib}
\bibliographystyle{alphadin}
\usepackage{url}
\usepackage{hyperref}

\begin{document}

\paragraph{Aufgabe 1.}
Sei $(\Omega,\mathscr{F},P)$ ein Wahrscheinlichkeitsraum und $\mathscr{A}=\{\emptyset,\Omega\}$ die triviale $\sigma$-Algebra.
Zeigen Sie $E[X|\mathscr{A}]=E[X]$ für alle $X\in L^1(\Omega,\mathscr{F},P)$.

Zunächst einmal ist $E[X]$ irgendeine konstante Zahl.
Das Urbild davon ist also ganz $\Omega$.
Somit ist $E[X]$ $\mathscr{A}$-messbar.
Weiterhin gilt $\int_{\Omega}E[X]\mathrm{d}P=E[X]\int_{\Omega}\mathrm{d}P=E[X]\cdot1=\int_\Omega X\mathrm{d}P$, sodass $E[X|\mathscr{A}]=E[X]$ gilt.

\paragraph{Aufgabe 2.} Zeigen Sie die folgenden Aussagen
\begin{enumerate}
\item[i)] Ist $(X_i)_{i\in I}$ gleichgradig integrierbar und $({\cal F}_j)_{j\in J}$ eine Familie von Unter-$\sigma$-Algebren von ${\cal A}$.
  Dann ist die Familie $(E[X_i,{\cal F}_j])_{i\in I,j\in J}$ gleichgradig integrierbar.
\end{enumerate}
% https://planetmath.org/conditionalexpectationsareuniformlyintegrable

Da $(X_i)$ gleichgradig integrierbar ist gibt es ein $L>0$, sodass für alle $i\in I$ gilt $E[|X_i|]\leq L$, \emph{wobei man sich überlegen müsste, warum genau}.
Zudem gilt, da $(X_i)$ gleichgradig integrierbar ist, dass für alle $\varepsilon>0$ ein $\delta>0$ existiert, sodass für alle $i\in I$ und alle $A\in{\cal A}$ mit $P(A)\leq\delta$ gilt $E[|X|\mathbbm{1}_A]<\varepsilon$.
Sei nun $k=L/\delta$ und $Y=E[X\mid{\cal G}]$
Dann gilt für alle $i\in I$ und $j\in J$ mit der Jensen'schen Ungleichung $|Y|\leq E[|X|\mid{\cal F}_j]$.
Mit der Markov-Ungleichung kriegen wir nun
\[
P(|Y|>k)\leq\frac{E[|Y|]}{k}\leq\frac{E[|X|]}{k}\leq L/k\leq\delta\,.
\]
Damit folgt, nach der obigen Erklärung, $E[|Y|\mathbbm{1}_{|Y|>k}]\leq E[|X|\mathbbm{1}_{|Y|>k}]<\varepsilon$, also ist $(E[X_i,{\cal F}_j])_{ij}$ gleichgradig integrierbar.
\begin{enumerate}
\item[ii)] Ist $X\in L^1(\Omega,{\cal A},P)$, dann ist $(E[X|{\cal F}_j])_{j\in J}$ gleichgradig integrierbar.
\end{enumerate}
%https://math.stackexchange.com/questions/3731917/uniform-integrability-of-all-conditional-expectations-of-a-fixed-l1-function
Für ein $k\in\mathbb{R}$ und ein $j\in J$ sei $Y=E[X|{\cal F}_j]$ und $Z=E\bigl[|X|\mid{\cal F}_j\bigr]$.
Dann gilt wegen der Jensen'schen Ungleichung der bedingten Erwartung
\begin{align*}
  E[|E[X|{\cal F}_j]|\mathbbm{1}_{|E[X|{\cal F}_j]|>k}]
  &\leq E[E[|X|\mid{\cal F}_j]\mathbbm{1}_{E[|X|\mid{\cal F}_j]>k}]\,.
    \intertext{Folglich gilt, \emph{wobei unklar ist, warum,}}
  &=E[|X|\mathbbm{1}_{E[|X|\mid{\cal F}_j]>k}]\,.
\end{align*}
Sei nun  $k$ hinreichend groß, dass $E[|E[|X|\mid{\cal F}_j]|]<k\delta$.
Dann gilt mit der Markov-Ungleichung  $P(E[|X|\mid{\cal F}_j]>k)\leq\frac{E[E[|X|\mid{\cal F}_j]]}{k}=\frac{E[|X|]}{k}<\delta$.
Da $X\in L^1$ ist, gilt dann $E[|E[X|{\cal F}_j]|\mathbbm{1}_{|E[X|{\cal F}_j]|>k}]<\varepsilon$.
\emph{Eventuell genauer zeigen, warum das gilt.}
\newpage

\bibliography{../../../books/wt}
\end{document}

%%% Local Variables:
%%% mode: latex
%%% ispell-local-dictionary: "german"
%%% TeX-master: t
%%% End:
