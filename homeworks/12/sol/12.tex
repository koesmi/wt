\documentclass{article}
\usepackage[a4paper,margin=1.875in,top=1.5in,bottom=1.5in]{geometry}

\usepackage{amsmath,mathtools,bbm,amssymb}
\usepackage{mathrsfs}
\usepackage[german]{babel}

\usepackage{setspace}
\doublespacing

\usepackage{fancyhdr}
\renewcommand{\headrulewidth}{0pt} 
\pagestyle{fancy}
\lhead{Blatt 12 Nicolas und Evgenij}\rhead{Seite \thepage}
\fancyfoot{}

\usepackage{tikz}
\usetikzlibrary{decorations.pathreplacing,arrows}

\usepackage[numbers]{natbib}
\bibliographystyle{alphadin}
\usepackage{url}
\usepackage{hyperref}

\begin{document}

\paragraph{Aufgabe 1.}
Zeigen Sie, dass für alle quadratintegrierbaren Martingale $M$, das heißt $E[M_n^2]<\infty$ für alle $n\in\mathbb{N}$ und $m\leq n$ folgende Aussagen gelten:
\begin{enumerate}
\item[i)] $E[(M_n-M_m)^2]\mid{\cal F}_m]=E[M_n^2-M_m^2\mid {\cal F}_m]$
\end{enumerate}
Durch Ausquadrieren erhalten wir
\begin{align*}
  E[(M_n-M_m)^2]\mid{\cal F}_m]
  &=E[M_n^2-2M_nM_m+M_m^2\mid{\cal F}_m]\,.
    \intertext{Da $M$ ein Martingal ist, ist es adaptiert.
    Damit ist $M_m$ ist ${\cal F}_m$ messbar und wir können es aus der bedingten Erwartung rausziehen.
    Zudem gilt dafür $E[M_m\mid {\cal F}_m]=M_m$, sodass}
  &=E[M_n^2\mid{\cal F}_m]-2M_mE[M_n\mid{\cal F}_m]+M_m^2\,.
    \intertext{Mit der Martingaleigenschaft folgt}
  &=E[M_n^2\mid{\cal F}_m]-2M_m^2+M_m^2\,.
    \intertext{Zusammenfassen der letzten beiden Terme liefert}
  &=E[M_n^2\mid{\cal F}_m]-M_m^2\,.    
    \intertext{wieder aufgrund der ${\cal F}_m$-Messbarkeit von $M_m$ erhalten wir}
  &=E[M_n^2-M_m^2\mid{\cal F}_m].
\end{align*}
\newpage
\begin{enumerate}
\item[ii)] $E[(M_n-M_m)^2]=E[M_n^2]-E[M_m^2]$
\end{enumerate}
Mit der definierenden Eigenschaft (ii) vom bedingten Erwartungswert ausgewertet auf $\Omega$ können wir schreiben
\begin{align*}
  E[(M_n-M_m)^2]
  &=E\bigl[E[(M_n-M_m)^2\mid{\cal F}_m]\bigr]
    \intertext{Einsetzten von Teilaufgabe (a) liefert}
  &=E\bigl[E[M_n^2-M_m^2\mid{\cal F}_m]\bigr]
    \intertext{und wieder die Eigenschaft (ii) auf $\Omega$ schließlich}
  &=E[M_n^2-M_m^2]\,.
\end{align*}
\newpage

% Für Aufgabe 2
% https://math.stackexchange.com/questions/3965911/prove-that-mathcal-f-tau-is-a-sigma-algebra

\paragraph{Aufgabe 3.}
%https://math.stackexchange.com/questions/2029863/central-limit-theorem-and-convergence-in-probability-from-durrett
Seien $X_1,X_2,\dots$ i.i.d. Zufallsvariablen mit $E[X_1]=0$ und $0<E[X_1^2]<\infty$.
Verwenden Sie das 0-1-Gesetz von Kolmogorov und den zentralen Grenzwertsatz, um zu zeigen, dass fast sicher
\[
\limsup_{n\to\infty}\frac{S_n}{\sqrt{n}}=\infty
\]
gilt, wobei $S_n:=\sum_{k=1}^n X_k$.

\noindent\emph{Hinweis: Verwenden Sie das Lemma von Fatou.}

Da die Inklusion gilt $\bigl\{\limsup_{n\to\infty}\frac{S_n}{\sqrt{n}}=\infty\bigr\}\subseteq\{(S_n)\text{ konvergiert nicht}\}$, ist $\bigl\{\limsup_{n\to\infty}\frac{S_n}{\sqrt{n}}=\infty\bigr\}$ nach Blatt 5 Aufgabe 5.6. ein terminales Ereignis.
Nach dem 0-1-Gesetz von Kolmogorov ist $P\bigl(\limsup_{n\to\infty}\frac{S_n}{\sqrt{n}}=\infty\bigr)\in\{0,1\}$.
Äquivalent können wir sagen, dass für ein beliebiges $N\in\mathbb{N}$ gilt $P\bigl(\limsup_{n\to\infty}\frac{S_n}{\sqrt{n}}>N\bigr)\in\{0,1\}$.
Das heißt wiederum, dass
\begin{align*}
  P\Bigl(\limsup\frac{S_n}{\sqrt{n}}>N\Bigr)
  &=P\Bigl(\frac{S_n}{\sqrt{n}}>N\text{ für unendlich viele $n$}\Bigr),
  \intertext{also}
  &=P\Bigl(\limsup_{n\to\infty}\Bigl\{\frac{S_n}{\sqrt{n}}>N\Bigr\}\Bigr)\,.
    \intertext{Mit Blatt 5 Aufgabe 2 \emph{oder eventuell auch mit dem Lemma von Fatou} erhalten wir}
  &\geq\limsup_{n\to\infty}P\Bigl(\frac{S_n}{\sqrt{n}}>N\Bigr)\,.
    \intertext{Da schließlich nach dem zentralen Grenzwertsatz $\frac{S_n}{\sqrt{n}}\Rightarrow Y$ mit $Y\sim{\cal N}(0,1)$, erhalten wir}
  &=1-P(Y\leq N)>0\,.
\end{align*}
Damit muss $P\bigl(\limsup_{n\to\infty}\frac{S_n}{\sqrt{n}}=\infty\bigr)=1$ gelten.

\paragraph{Aufgabe 4.}
% https://johnthickstun.com/docs/coinflip.pdf
Geben Sie für einen wiederholten Münzwurf (mit fairer Münze) einen Wahrscheinlichkeitsraum an und zeigen Sie, dass der Prozess $(M_n)_{n\in\mathbb{N}}$, der die Summe der Auszahlung $X=(X_n)_{n\in\mathbb{N}}$ von $1$, beziehungsweise $-1$ beschreibt ein Martingal bezüglich seiner Filtration ist.
Das Spiel endet, wenn die Auszahlung von $a\in\mathbb{N}$ erreicht ist.
Ist das gestoppte Spiel immer noch ein Martingal?
Was lässt sich über die Konvergenz (fast sicher und $L^1$) des gestoppten Spiels aussagen?

Bei jedem einzelnen Münzwurf sind die verschiedenen Auskommen $\Omega_1=\{\mathrm{K},\mathrm{Z}\}$ für Kopf oder Zahl.
Der Grundraum von $X$ ist dann $\Omega=\Omega_1^\mathbb{N}$, sodass für alle $\omega\in \Omega$ gilt $\omega_n\in \Omega_1$.
Die Filtration ist $\mathbb{F}=({\cal F}_n)$ mit ${\cal F}_n={\cal P}(\Omega_1)^{\otimes n}$.
Auf ${\cal P}(\Omega_1)$ ist ein Wahrscheinlichkeitsmaß $P_1$ gegeben durch $P_1(\emptyset)=0$, $P_1(\mathrm{K})=P_1(\mathrm{Z})=\frac{1}{2}$ und $P_1(\Omega_1)=1$.
Das die Würfe unabhängig sind, können wir definieren $P_n(\omega)=\prod_{k=1}^nP_1(\omega_k)$ setzen.
Da $\Omega$ polnisch ist, gibt es einen projektiven Limes $P$ zu $(P_n)$ auf $\Omega$, was $(\Omega,{\cal F},P)$ zu einem Wahrscheinlichkeitsraum macht.
Sei $A\in\mathbb{Z}$, dann ist $M_n^{-1}(A)\in{\cal F}_n$, da ${\cal F}_n$ aus den Potenzmengen von $\Omega_1$ besteht.
Somit ist $M_n$ adaptiert.
\emph{Es sollte noch geprüft werden, dass $\mathbb{F}$ tatsächlich die natürliche Filtration von $(M_n)$ ist.}
Wir sollen noch prüfen, ob $(M_n)$ ein Martingal ist.
Sei hierfür $m<n$.
Dann gilt mit der Definition von $(M_n)$
\begin{align*}
  E[M_n\mid{\cal F}_m]
  &=E\Bigl[\sum\nolimits_{k=1}^nX_k\mid{\cal F}_m\Bigr]\,.
    \intertext{Wegen der Linearität der bedingten Erwartung gilt}
  &=E[M_m\mid {\cal F}_m]+E\Bigl[\sum\nolimits_{k=m+1}^nX_k\mid{\cal F}_m\Bigr]\,.
    \intertext{Da $M_m$ ${\cal F}_m$-messbar und für $k>m$ $X_k$ unabhängig von ${\cal F}_m$ ist, gilt}
  &=M_m+\sum_{k=m+1}^nE[X_k]=M_m\,,
\end{align*}
Denn $E[X_k]=1\cdot\frac{1}{2}-1\cdot\frac{1}{2}=0$.
Somit ist $(M_n)$ ein Martingal.
Nach Satz 189 ist auch der gestoppte Prozess ein Martingal.
Da auch $\sup E[|M_n|]=0$, gilt nach dem Doobschen Martingalkonvergenzsatz, dass $M_n$ fast sicher und in $L^1$ konvergiert.
\bibliography{../../../books/wt}
\end{document}

%%% Local Variables:
%%% mode: latex
%%% ispell-local-dictionary: "german"
%%% TeX-master: t
%%% End:
