\documentclass{article}
\usepackage[a4paper,margin=1.875in,top=1.875in,bottom=1.875in]{geometry}

\usepackage{amsmath,mathtools,bbm,amssymb}
\usepackage{mathrsfs}
\usepackage[german]{babel}

\usepackage{setspace}
\doublespacing

\usepackage{fancyhdr}
\renewcommand{\headrulewidth}{0pt} 
\pagestyle{fancy}
\lhead{Blatt 12 Nicolas und Evgenij}\rhead{Seite \thepage}
\fancyfoot{}

\usepackage{tikz}
\usetikzlibrary{decorations.pathreplacing,arrows}

\usepackage[numbers]{natbib}
\bibliographystyle{alphadin}
\usepackage{url}
\usepackage{hyperref}

\begin{document}

\paragraph{Aufgabe 1.}
Zeigen Sie, dass für alle quadratintegrierbaren Martingale $M$, das heißt $E[M_n^2]<\infty$ für alle $n\in\mathbb{N}$ und $m\leq n$ folgende Aussagen gelten:
\begin{enumerate}
\item[i)] $E[(M_n-M_m)^2]\mid{\cal F}_m]=E[M_n^2-M_m^2\mid {\cal F}_m]$
\end{enumerate}
Durch Ausquadrieren erhalten wir
\begin{align*}
  E[(M_n-M_m)^2]\mid{\cal F}_m]
  &=E[M_n^2-2M_nM_m+M_m^2\mid{\cal F}_m]\,.
    \intertext{Da $M$ ein Martingal ist, ist es adaptiert.
    Damit ist $M_m$ ist ${\cal F}_m$ messbar und wir können es aus der bedingten Erwartung rausziehen.
    Zudem gilt dafür $E[M_m\mid {\cal F}_m]=M_m$, sodass}
  &=E[M_n^2\mid{\cal F}_m]-2M_mE[M_n\mid{\cal F}_m]+M_m^2\,.
    \intertext{Mit der Martingaleigenschaft folgt}
  &=E[M_n^2\mid{\cal F}_m]-2M_m^2+M_m^2\,.
    \intertext{Zusammenfassen der letzten beiden Terme liefert}
  &=E[M_n^2\mid{\cal F}_m]-M_m^2\,.    
    \intertext{wieder aufgrund der ${\cal F}_m$-Messbarkeit von $M_m$ erhalten wir}
  &=E[M_n^2-M_m^2\mid{\cal F}_m].
\end{align*}
\newpage
\begin{enumerate}
\item[ii)] $E[(M_n-M_m)^2]=E[M_n^2]-E[M_m^2]$
\end{enumerate}
Mit der definierenden Eigenschaft (ii) vom bedingten Erwartungswert ausgewertet auf $\Omega$ können wir schreiben
\begin{align*}
  E[(M_n-M_m)^2]
  &=E\bigl[E[(M_n-M_m)^2\mid{\cal F}_m]\bigr]
    \intertext{Einsetzten von Teilaufgabe (a) liefert}
  &=E\bigl[E[M_n^2-M_m^2\mid{\cal F}_m]\bigr]
    \intertext{und wieder die Eigenschaft (ii) auf $\Omega$ schließlich}
  &=E[M_n^2-M_m^2]\,.
\end{align*}
\newpage
\paragraph{Aufgabe 3.}
%https://math.stackexchange.com/questions/2029863/central-limit-theorem-and-convergence-in-probability-from-durrett
Seien $X_1,X_2,\dots$ i.i.d. Zufallsvariablen mit $E[X_1]=0$ und $0<E[X_1^2]<\infty$.
Verwenden Sie das 0-1-Gesetz von Kolmogorov und den zentralen Grenzwertsatz, um zu zeigen, dass fast sicher
\[
\limsup_{n\to\infty}\frac{S_n}{\sqrt{n}}=\infty
\]
gilt, wobei $S_n:=\sum_{k=1}^n X_k$.

\emph{Hinweis: Verwenden Sie das Lemma von Fatou.}

Mit der Definition des $\limsup$ gilt \emph{hier sollte der zentrale Grenzwertsatz angewendet werden}
\begin{align*}
  \Bigl\{\limsup_{n\to\infty}\frac{S_n}{\sqrt{n}}=\infty\Bigr\}
  &=\Bigl\{\inf_{n\geq 1}\sup_{m\geq n}\frac{S_m}{\sqrt{m}}=\infty\Bigr\}\,.
    \intertext{Das können wir äquivalent schreiben als}
  &=\Bigl\{\forall n\geq 1~\exists m\geq n~\frac{S_m}{\sqrt{m}}=\infty\Bigr\}\,,
    \intertext{oder}
  &=\bigcap_{n\geq 1}\bigcup_{m\geq n}\Bigl\{\frac{S_m}{\sqrt{m}}=\infty\Bigr\}\,.
    \intertext{Die ersten Terme ausgeschrieben erhalten wir}
  &=\Bigl\{\frac{S_1}{\sqrt{1}}=\infty\Bigr\}\cup\Bigl\{\frac{S_2}{\sqrt{2}}=\infty\Bigr\}\cup\Bigl\{\frac{S_3}{\sqrt{3}}=\infty\Bigr\}\cup\dots\\
  &\quad\cap\Bigl\{\frac{S_2}{\sqrt{2}}=\infty\Bigr\}\cup\Bigl\{\frac{S_3}{\sqrt{3}}=\infty\Bigr\}\cup\dots\\
  &\quad\cap\dots
\end{align*}
Somit gilt für alle $k\in\mathbb{N}$, dass $\bigl\{\limsup_{n\to\infty}\frac{S_n}{\sqrt{n}}=\infty\bigr\}\in\sigma(X_k,X_{k+1},\dots)$, es ist also ein terminales Ereignis.
Nach dem 0-1-Gesetz von Kolmogorov ist $P\bigl(\limsup_{n\to\infty}\frac{S_n}{\sqrt{n}}=\infty\bigr)\in\{0,1\}$.
\bibliography{../../../books/wt}
\end{document}

%%% Local Variables:
%%% mode: latex
%%% ispell-local-dictionary: "german"
%%% TeX-master: t
%%% End:
