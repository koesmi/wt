\documentclass{article}
\usepackage[a4paper,margin=1.875in,top=1.5in,bottom=1.5in]{geometry}

\usepackage{amsmath,mathtools,bbm,amssymb}
\usepackage[german]{babel}

\usepackage{setspace}
\doublespacing

\usepackage{fancyhdr}
\renewcommand{\headrulewidth}{0pt} 
\pagestyle{fancy}
\lhead{Blatt 9 Nicolas und Evgenij}\rhead{Seite \thepage}
\fancyfoot{}

\usepackage{tikz}
\usetikzlibrary{decorations.pathreplacing,arrows.meta}

\usepackage[numbers]{natbib}
\bibliographystyle{alphadin}
\usepackage{url}
\usepackage{hyperref}

\begin{document}

% 2 ist übung 13.3.2 im Klenke

\paragraph{B9A2}
Wir wollen zeigen, dass $\{{\cal N}(\mu,\sigma^2)\mid (\mu,\sigma^2)\in L\subset\mathbb{R}\times(0,\infty)\}$ genau dann straff ist, wenn $L$ beschränkt ist.
Angenommen $L$ ist beschränkt.
Wir wollen zeigen, dass für alle $\varepsilon>0$ ein $r>0$ existiert, sodass für alle $(\mu,\sigma)\in L$ gilt ${\cal N}(\mu,\sigma)([-r,r])>1-\varepsilon$.
Sei also $\varepsilon>0$ gegeben.
Es gilt
\begin{align*}
  {\cal N}(\mu,\sigma)([-r,r])
  &=\frac{1}{\sqrt{2\pi\sigma^2}}\int_{-r}^r\exp\Bigl[-\frac{1}{2}\Bigl(\frac{x-\mu}{\sigma}\Bigr)^2\Bigr]\mu(\mathrm{d}x)\,.
    \intertext{Substitution mit $z=(x-\mu)/\sigma$ liefert}
  &=\frac{1}{\sqrt{2\pi\sigma^2}}\int_{(-r-\mu)/\sigma}^{(r-\mu)/\sigma}\exp\Bigl[-\frac{1}{2}z^2\Bigr]\mu(\mathrm{d}x)\,.
\end{align*}
\paragraph{B9A4}
Sei $(P_i)_{i\in I}$ eine Familie von Wahrscheinlichkeitsmaßen auf $\mathbb{R}^d$.
Zeigen Sie, dass die folgenden Aussagen äquivalent sind:
\begin{enumerate}
\item $(P_i)_{i\in I}$ ist straff
\item Für alle Projektionen $\pi_1,\dots,\pi_d$ ist $(P_i^{\pi_k})_{i\in I}$ straff.
\end{enumerate}
Sei zunächst $(P_i)_{i\in I}$ straff.
Dann gibt es für alle $\varepsilon>0$ eine kompakte Menge $K\in\mathbb{R}^d$, sodass für alle $i\in I$ gilt $P_i(K)>1-\varepsilon$.
Da für alle $k\leq d$ gilt, dass $K\subset \pi_k^{-1}\bigl(\pi_k(K)\bigr)$, gilt aufgrund der Monotonie des Maßes auch für alle $\varepsilon>0$ und alle $i\in I$ dass $P_i^{\pi_k}\bigl(\pi_k(K)\bigr)>1-\varepsilon$.
Damit ist für $(P_i^{\pi_k})_{i\in I}$ für alle Projektionen $\pi_1,\dots,\pi_d$ straff.

Seien nun für alle Projektionen $\pi_1,\dots,\pi_d$ die Familien der Bildmaße $(P_i^{\pi_k})_{i\in I}$ straff.
Dann gibt es für alle $\varepsilon>0$ Kompakta $K_1,\dots,K_d$, sodass für alle $i\in I$ gilt $P_i^{\pi_k}(K_k)=P_i\bigl(\pi_k^{-1}\bigr(K_k)\bigr)>1-\varepsilon$.
Sei $r>0$ so, dass $\overline{B}_r(0)\supset K_k$.
Dann gilt auch $P_i^{\pi_k}\bigl(\overline{B}_r(0)\bigr)>1-\varepsilon$.
Betrachte $K=\bigtimes_{k=1}^d\overline{B}_r(0)$.
Dann gilt für alle $i\in I$, dass $P_i(K)=P_i\bigl(\bigcap_{k=1}^d\{|\omega_k|\leq r\}\bigr)$.
Entsprechend gilt, dass
\begin{align*}
  P_i(K^\mathrm{c})
  &=P_i\Bigl(\bigcup\nolimits_{k=1}^d\{|\omega_k|>r\}\Bigr).
    \intertext{durch die $\sigma$-Subadditivität der Maße $P_i$ können wir abschätzen}
  &\leq\sum\nolimits_{k=1}^d P_i(|\omega_k|>r)\,,
    \intertext{wobei $\{|\omega_k|>r\}=\pi_k^{-1}\bigl(\overline{B}_r(0)^\mathrm{c}\bigr)$, sodass}
  &\leq\sum\nolimits_{k=1}^d P_i^{\pi_k}\bigl(\overline{B}_r(0)^\mathrm{c}\bigr)\,.
    \intertext{Da wir $r$ so gewählt haben, dass $P_i^{\pi_k}\bigl(\overline{B}_r(0)\bigr)>1-\varepsilon$, erhalten wir}
    &\leq d\varepsilon.
\end{align*}
Sei nun $\delta>0$ gegeben.
Wähle $\varepsilon=\delta/d$, dann gilt für alle $i\in I$, dass $P_i(K)>1-\delta$.
Somit ist $(P_i)$ straff.
% Tipp: https://math.stackexchange.com/questions/2915057/if-the-coordinates-of-random-vectors-are-tight-are-the-random-vectors-tight
\newpage


\bibliography{../../../books/wt}
\end{document}

%%% Local Variables:
%%% mode: latex
%%% ispell-local-dictionary: "german"
%%% TeX-master: t
%%% End:
