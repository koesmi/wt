\documentclass{article}
\usepackage[a4paper,margin=1.875in,top=1.875in,bottom=1.875in]{geometry}

\usepackage{amsmath,mathtools,bbm,amssymb}
\usepackage[german]{babel}

\usepackage{setspace}
\doublespacing

\usepackage{fancyhdr}
\renewcommand{\headrulewidth}{0pt} 
\pagestyle{fancy}
\lhead{Blatt 9 Nicolas und Evgenij}\rhead{Seite \thepage}
\fancyfoot{}

\usepackage{tikz}
\usetikzlibrary{decorations.pathreplacing,arrows.meta}

\usepackage[numbers]{natbib}
\bibliographystyle{alphadin}
\usepackage{url}
\usepackage{hyperref}

\begin{document}

% 2 ist übung 13.3.2 im Klenke

\paragraph{B9A4}
Sei $(P_i)_{i\in I}$ eine Familie von Wahrscheinlichkeitsmaßen auf $\mathbb{R}^d$.
Zeigen Sie, dass die folgenden Aussagen äquivalent sind:
\begin{enumerate}
\item $(P_i)_{i\in I}$ ist straff
\item Für alle Projektionen $\pi_1,\dots,\pi_d$ ist $(P_i^{\pi_k})_{i\in I}$ straff.
\end{enumerate}
Sei zunächst $(P_i)_{i\in I}$ straff.
Dann gibt es für alle $\varepsilon>0$ eine kompakte Menge $K\in\mathbb{R}^d$, sodass für alle $i\in I$ gilt $P_i(K)>1-\varepsilon$.
Da für alle $k\leq d$ gilt, dass $K\subset \pi_k^{-1}\bigl(\pi_k(K)\bigr)$, gilt aufgrund der Monotonie des Maßes auch für alle $\varepsilon>0$ und alle $i\in I$ dass $P_i^{\pi_k}\bigl(\pi_k(K)\bigr)>1-\varepsilon$.
Damit ist für $(P_i^{\pi_k})_{i\in I}$ für alle Projektionen $\pi_1,\dots,\pi_d$ straff.

Seien nun für alle Projektionen $\pi_1,\dots,\pi_d$ die Familien der Bildmaße $(P_i^{\pi_k})_{i\in I}$ straff.
Dann gibt es für alle $\varepsilon>0$ Kompakta $K_1,\dots,K_d$, sodass für alle $i\in I$ gilt $P_i^{\pi_k}(K_k)=P_i\bigl(\pi_k^{-1}\bigr(K_k)\bigr)>1-\varepsilon$.
%Betrachte $K=\bigtimes_{k=1}^dK_k$.
%Für alle $k\leq d$ gilt $K\subset\pi_k^{-1}(K_k)$.
% Tipp: https://math.stackexchange.com/questions/2915057/if-the-coordinates-of-random-vectors-are-tight-are-the-random-vectors-tight
\newpage


\bibliography{../../../books/wt}
\end{document}

%%% Local Variables:
%%% mode: latex
%%% ispell-local-dictionary: "german"
%%% TeX-master: t
%%% End:
