\documentclass{article}
\usepackage[a4paper,margin=1.875in,top=1.875in,bottom=1.875in]{geometry}

\usepackage{amsmath,mathtools,bbm,amssymb}
\usepackage[german]{babel}

\usepackage{setspace}
\doublespacing

\usepackage{fancyhdr}
\renewcommand{\headrulewidth}{0pt} 
\pagestyle{fancy}
\lhead{Blatt 5 Nicolas und Evgenij}\rhead{Seite \thepage}
\fancyfoot{}

\usepackage{tikz}
\usetikzlibrary{decorations.pathreplacing,arrows.meta}

\usepackage[numbers,round]{natbib}
\bibliographystyle{alphadin}
\usepackage{url}
\usepackage{hyperref}

\begin{document}

\paragraph{B5A1}
%https://math.stackexchange.com/questions/2529434/x-k-uniformly-integrable-iff-lim-n-to-infty-limsup-k-to-infty-mathbb-e
Zeigen Sie, dass eine Folge $(X_n)_{n\in\mathbb{N}}$ genau dann gleichgradig integrierbar ist, wenn
\begin{equation}
  \label{eq:ggiblim}\tag{1}
  \lim_{k\to\infty}\limsup_{n\to\infty}E[|X_n|\mathbbm{1}_{\{|X_n|>k\}}]=0\,.
\end{equation}
Sei zunächst $(X_n)$ gleichgradig integrierbar.
Dann gilt nach Definition von gleichgradiger Integrierbarkeit, dass
\begin{align*}
  0
  &=\lim_{k\to\infty}\sup_{n\in\mathbb{N}}E[|X_n|\mathbbm{1}_{\{|X_n|>k\}}]\,.
    \intertext{Insbesondere heißt dass, dass für ein gegebenes $n\in\mathbb{N}$ gilt, dass}
  0
  &=\lim_{k\to\infty}\sup_{m\geq n}E[|X_m|\mathbbm{1}_{\{|X_m|>k\}}]\,.
    \intertext{Damit gilt auch für das Infimum über diese $n\in\mathbb{N}$, dass}
    0
  &=\lim_{k\to\infty}\inf_{n\in\mathbb{N}}\sup_{m\geq n}E[|X_m|\mathbbm{1}_{\{|X_m|>k\}}]\,,
\end{align*}
was nach der Definition des $\limsup$ Gleichung (\ref{eq:ggiblim}) liefert.
\emph{Hier sollte man sich überlegen, ob man den Limes über $k$ ebenfalls, wie in der Argumentation unten, mittels eines $\varepsilon$ umschreiben sollte.}

Genüge nun $(X_n)$ Gleichung (\ref{eq:ggiblim}).
Wir wollen zeigen, dass $(X_n)$ gleichgradig integrierbar ist.
Sei hierfür ein $\varepsilon>0$ gegeben.
Da Gleichung (\ref{eq:ggiblim}) gilt, gibt es ein $k_0\in\mathbb{N}$, sodass für alle $k>k_0$
\begin{align*}
  \limsup_{n\to\infty}E[|X_n|\mathbbm{1}_{\{|X_n|>k\}}]
  &<\varepsilon\,.
    \intertext{
    \emph{Damit ist die Teilfolge $\bigl(\sup_{m\geq n}E[|X_m]\mathbbm{1}_{\{|X_m|>k\}}\bigr)_n$ für alle $n\in\mathbb{N}$ beschränkt.
    Hier könnte man eventuell noch direkt ein Argument bringen, warum schon Beschränktheit durch $\varepsilon$ folgt.}
    Also gibt es auch ein $n_0$, sodass für alle $n\geq n_0$ gilt}
    E[|X_n|\mathbbm{1}_{\{|X_n|>k\}}]
  &<\varepsilon\,.
    \intertext{Nach Lemma 24 gilt, dass alle $X_n$ integrierbar sind.
    Also gibt es für alle $X_1,\dots,X_{n_0-1}$ jeweils Schranken $k_1,\dots,k_{n_0-1}\in\mathbb{N}$, sodass für die $n=1,\dots,n_0-1$ gilt}
    E[|X_n|\mathbbm{1}_{\{|X_n|>k_n\}}]<\varepsilon\,.
    \intertext{Setze nun $K=k_0\vee\cdots\vee k_{n_0-1}$, dann gilt für alle $k\geq K$ und alle $n\in\mathbb{N}$, dass}
    E[|X_n|\mathbbm{1}_{\{|X_n|>k\}}]<\varepsilon\,,
    \intertext{also insbesondere, dass}
    \lim_{k\to\infty}\sup_{n\in\mathbb{N}}E[|X_n|\mathbbm{1}_{\{|X_n|>k\}}]<\varepsilon\,.
\end{align*}
\emph{Hier fehlt noch die Folgerung von Korollar 27. Es sollte also gezeigt werden, dass falls $\|X_n\|_p\to \|X\|_p$, dann $\lim_{k\to\infty}\limsup_{n\to\infty}E[|X_n|^p\mathbbm{1}_{|X_n|^p>k}]=0$.}
Es gelte $\|X_n\|_p\to \|X\|_p$.
Da die Folge $E[|X_n|^p]$ dann konvergent ist, ist sie beschränkt.
Sei $c\in\mathbb{R}$ vorgegeben.
Angenommen, es gelte, dass $\lim_{k\to\infty}\limsup_{n\to\infty}E[|X_n|^p\mathbbm{1}_{|X_n|^p>k}]>0$.
\emph{Eventuell kann man folgern, dass $E[|X_n|^p]$ nicht beschränkt ist.}
\newpage

\paragraph{B5A2}
Sei $(\Omega,{\cal F},P)$ ein Wahrscheinlichkeitsraum und $(A_n)_{n\geq 1}$ eine Folge in ${\cal F}$.
Es sei
\[
  \limsup_{n\to\infty}A_n:=\bigcap_{n=1}^\infty\bigcup_{k=n}^\infty A_k\,,\qquad\liminf_{n\to\infty}A_n:=\bigcup_{n=1}^\infty\bigcap_{k=n}^\infty A_k\,.
\]
Zeigen Sie, dass gilt
\begin{align*}
  \liminf_{n\to\infty}A_n
  &\subseteq\limsup_{n\to\infty}A_n\,,\\
  P\Bigl(\liminf_{n\to\infty}A_n\Bigr)\leq \liminf_{n\to\infty}P(A_n)
  &\leq\limsup_{n\to\infty}P(A_n)\leq P\Bigl(\limsup_{n\to\infty}A_n\Bigr)\,.
\end{align*}

Wir möchten zunächst zeigen, dass $\liminf_{n\to\infty}A_n\subseteq\limsup_{n\to\infty}A_n$.
Sei hierfür $\omega\in\liminf_{n\to\infty}A_n$. Wie wir in der Übung besprochen haben, heißt das, dass ein $n_0\in\mathbb{N}$ existiert, sodass für alle $n\geq n_0$ gilt, dass $\omega\in A_n$.
Um zu zeigen, dass $\omega\in\limsup_{n\to\infty}A_n$, sei ein $m_0\in\mathbb{N}$ beliebig vorgegeben.
Da für $n\geq n_0$ gilt, dass $\omega\in A_n$, ist $\omega\in A_{m_0\vee n_0}$.
Damit gilt auch für alle $m_0\in\mathbb{N}$, dass ein $m\geq m_0$ existiert, sodass $\omega\in A_m$, nämlich $m=m_0\vee n_0$.
Somit ist $\omega\in\limsup_{n\to\infty}A_n$.

Nun überlegen wir uns, warum die Ungleichungen gelten.
$\mathbbm{1}_{A_n}$ hat $\mathbbm{1}_\Omega$ als integrierbare Majorante.
Damit gelten die erste und dritte Ungleichung gelten jeweils nach dem Lemma von Fatou, beziehungsweise die dritte zusätzlich nach majorisierter Konvergenz.
\emph{Hier fehlt noch eine Rechnung für die mittlere Ungleichung.}
\newpage

\paragraph{B5A3}
Sei $\lambda>0$ und für jedes $n\in\mathbb{N}_0$ sei $X_n$ eine Poisson-verteilte Zufallsvariable zum Parameter $\lambda$.
Zeigen Sie mit Hilfe des Lemmas von Borel--Cantelli, dass
\[
  P(\{\text{Für unendlich viele $n$ gilt $X_n>n$}\})=0\,.
\]

Es gilt genau dann $\omega\in\{\text{Für unendlich viele $n$ gilt $X_n>n$}\}$, wenn $\forall n\geq0~\exists m\geq n~\omega\in\{X_m>n\}$, also genau dann, wenn $\omega\in\limsup\{X_n>n\}$.
Das Lemma von Borel--Cantelli besagt, dass $P(\limsup A_n)=0$, falls $\sum_{n\geq0}P(A_n)<\infty$.
Die Idee ist also, zu zeigen, dass $\sum_{n\geq0}P(\{X_n>n\})<\infty$, dann folgt die Behauptung mit dem Lemma von Borel--Cantelli.
Die Poissonverteilung ist gegeben durch $P(X_n=k)=\mathrm{e}^{-\lambda}\frac{\lambda^k}{k!}$.
Damit ist
\begin{align*}
  \sum_{n=0}^\infty P(X_n>n)
  &=\sum_{n=0}^\infty\sum_{k=n+1}^\infty \mathrm{e}^{-\lambda}\frac{\lambda^k}{k!}\,.
\end{align*}
So, ähnlich wie wir es im Beweis zum starken Gesetz der Großen Zahlen in der Ausführung als Theorem 49 gemacht haben, können wir die Indices umschreiben.
Die Terme haben folgende Indices
\begin{center}
  \begin{tabular}{cc|cccccc}
    &$n$&&&&&&\\
    \hline
    $k$&0&1&2&3&4&$\cdots$\\
    &1&&2&3&4&$\cdots$\\
    &2&&&3&4&$\cdots$\\
    &3&&&&4&$\cdots$\\
    &$\vdots$&&&&&$\cdots$
  \end{tabular}
\end{center}
Es kommt also der Term mit Index $n+1$ immer $n+1$ mal vor.
Wir können für die obige Summe dann schreiben
\begin{align*}
  \sum_{n=0}^\infty P(X_n>n)
  &=\sum_{n=0}^\infty(n+1)\mathrm{e}^{-\lambda}\frac{\lambda^{(n+1)}}{(n+1)!}\,.
    \intertext{Kürzen mit $n+1$ und herausziehen von einem Faktor $\lambda\mathrm{e}^{-\lambda}$ ergibt}
  &=\lambda\mathrm{e}^{-\lambda}\sum_{n=0}^\infty\frac{\lambda^n}{n!}\,.
    \intertext{Mit der Definition der Exponentialfunktion erhalten wir}
  &=\lambda\mathrm{e}^{-\lambda}\mathrm{e}^{\lambda}=\lambda<\infty\,.
\end{align*}
\newpage

\paragraph{B5A4}
Nennen Sie jeweils ein Beispiel und ein Gegenbeispiel von reellwertigen Zufallsvariablen $(X_n)_{n\in\mathbb{N}}$ und einer Menge $A\in\mathcal{B}(\mathbb{R})$ für die folgenden Identitäten
  \begin{alignat*}{2}
&\text{1.~}&P\Bigl(\Bigl\{\limsup_{n\to\infty}X_n\in A\Bigr\}\Bigr)&=P\Bigl(\limsup_{n\to\infty}\{X_n\in A\}\Bigr)\\
&\text{2.~}&P\Bigl(\limsup_{n\to\infty}\{X_n\in A\}\Bigr)&=\limsup_{n\to\infty}P(\{X_n\in A\})\\
&\text{3.~}&\limsup_{n\to\infty}P(\{X_n\in A\})&=P\Bigl(\Bigl\{\limsup_{n\to\infty}X_n\in A\Bigr\}\Bigr)
\end{alignat*}
Als Beispiel können wir $X_n=0$ mit $A=\{0\}$ wählen.
Da für alle $n\in\mathbb{N}$ gilt, $X_n=0$, ist auch $\limsup X_n=\inf_{n\geq1}\sup_{m\geq n} X_n=0$.
Somit ist $\{\omega\in\Omega\mid\limsup X_n\in A\}=\Omega$ und $P(\limsup X_n\in A)=1$.
Genauso gilt für alle $n\in\mathbb{N}$, unabhängig von der Wahl von $\omega\in\Omega$, dass $X_n\in A$, sodass $\omega\in\limsup\{ X_n\in A\}$ für alle $\omega\in\Omega$ gilt.
Hiermit ist $P(\limsup\{X_n\in A\})=P(\Omega)=1$.
Schließlich gilt unabhängig von der Wahl von $n\in\mathbb{N}$, dass $P(\{\omega\in\Omega\mid X_n\in A\})=1$, sodass
$\limsup P(X_n\in A)=1$.

Als Gegenbeispiel zu Punkt 3 können wir Aufgabe 4 von Blatt 4 wählen, wo $X_n$ stochastisch, aber nicht fast sicher konvergierte.
Sei $A=B_\varepsilon(0)$, dann ist $\limsup_{n\to\infty}P(X_n\in A)=1$, ich finde jedoch immer ein $n\in\mathbb{N}$, sodass $X_n>\varepsilon$, wodurch $\limsup X_n\notin A$ und somit $P(\limsup X_n\in A)<1$.

\emph{Es fehlen noch Gegenbeispiele zu Punkt 1. und Punkt 2.}
\newpage

\paragraph{B5A5}
Sei $X_1,X_2,\dots$ eine Folge von reellwertigen Zufallsvariablen auf einem Wahrscheinlichkeitsraum $(\Omega,{\cal F},P)$, ${\cal F}_n:=\sigma(X_n)$ für alle $n$ und $S_n:=\sum_{k=1}^nX_k$.
Zeigen Sie oder widerlegen Sie durch ein Gegenbeispiel, dass die folgenden Ereignisse terminale Ereignisse sind, also in der terminalen $\sigma$-Algebra liegen.
\begin{align*}
  \text{1.~}&\{\omega\in\Omega\mid X_n(\omega)=0\}\text{ für ein $n\in\mathbb{N}$,}\\
  \text{2.~}&\{\omega\in\Omega\mid X_n(\omega)=0\text{ für ein $n\in\mathbb{N}$}\}\,,\\
  \text{3.~}&\{\omega\in\Omega\mid X_n(\omega)=0\text{ für endlich viele $n\in\mathbb{N}$}\}\,,\\
  \text{4.~}&\{\omega\in\Omega\mid S_n(\omega)=0\text{ für endlich viele $n\in\mathbb{N}$}\}\,,\\
  \text{5.~}&\{\omega\in\Omega\mid (X_n(\omega))_n\text{ enthält eine monoton wachsende Teilfolge}\}\,,\\
  \text{6.~}&\{\omega\in\Omega\mid \limsup S_n(\omega)>\liminf S_n(\omega)\}\,.
\end{align*}
Zu Punkt 1, wenn für alle $\omega\in\Omega$ und alle $n\in\mathbb{N}$ gilt $X_n(\omega)\neq0$, dann ist $\{\omega\in\Omega\mid X_n(\omega)=0\}=\emptyset$ und somit ein terminales Ereignis.
Sei $X_0=\mathrm{id}$, sowie $X_n=1$ für $n\geq1$.
Dann ist $\{\omega\in\Omega\mid X_n(\omega)=0\}=\{0\}$ für $n=0$ und $\{\omega\in\Omega\mid X_n(\omega)=0\}=\emptyset$ sonst.
Außerdem ist $\sigma(X_0)=\mathrm{id}^{-1}({\cal B}(\mathbb{R}))={\cal B}(\mathbb{R})$ und $\sigma(X_n)=\{\emptyset,\Omega\}$ sonst.
Dann ist $\bigcup_{m\geq1}\sigma(X_m)=\{\emptyset,\Omega\}$ und somit $\sigma\bigl(\bigcap_{m\geq 1}\sigma(X_m)\bigr)=\{\emptyset,\Omega\}$, wodurch auch $\bigcap_{n\geq0}\sigma\bigl(\bigcup_{m\geq n}\sigma(X_m)\bigr)=\{\emptyset,\Omega\}$.
Somit ist $\{\omega\in\Omega\mid X_0(\omega)=0\}$ kein terminales Ereignis.

Das Beispiel von oben ist auch ein Gegenbeispiel für Punkt 2, denn auch hier ist $\{\omega\in\Omega\mid X_n(\omega)=0\text{ für ein $n\in\mathbb{N}$}\}=\{0\}$, weil sonst ja nie irgendein $\omega\in\Omega$ auf 0 abbildet.

Damit ist es auch ein Gegenbeispiel zu Punkt 3 und Punkt 4, denn $\{0\}$ ist eine endliche Teilmenge der natürlichen Zahlen.
Außerdem ist ebenfalls $\{S_n(\omega)=0\}=\{0\}$.

$A_5=\{\omega\in\Omega\mid (X_n(\omega))_n\text{ enthält eine monoton wachsende Teilfolge}\}$ aus Punkt 5 ist in jedem $\sigma(X_n)$ enthalten sofern es nicht leer ist, denn ich kann für ein beliebiges $n\in\mathbb{N}$ die Teilfolge mit $(X_{n_k}(\omega))_k$ mit $n_0>n$ betrachten, sodass $A_5\in\sigma(X_n)$.
Da $n\in\mathbb{N}$ beliebig gewählt ist, gilt hierdurch auch $A_5\in\bigcap_{n\geq1}\sigma\bigl(\bigcup_{m\geq n}\sigma(X_m)\bigr)$, also ein terminales Ereignis.

Entsprechend ist $A_6=\{\omega\in\Omega\mid \limsup S_n(\omega)>\liminf S_n(\omega)\}$ ein terminales Ereignis.
Wähle ein beliebiges $m\in\mathbb{N}$, dann ist $\limsup S_n(\omega)>\liminf S_n(\omega)$ genau dann, wenn $\limsup_n\sum_{k=m}^n X_k>\liminf_n\sum_{k=m}^n X_k$ und $\{\omega\in\Omega\mid \limsup_n\sum_{k=m}^n X_k>\liminf_n\sum_{k=m}^n X_k\}\subset\sigma(X_n)$.
Wieder gilt $A_6\in\bigcap_{n\geq1}\sigma\bigl(\bigcup_{m\geq n}\sigma(X_m)\bigr)$, da $n\in\mathbb{N}$ beliebig gewählt war.
\newpage

\bibliography{../../../books/wt}
\end{document}

%%% Local Variables:
%%% mode: latex
%%% ispell-local-dictionary: "german"
%%% TeX-master: t
%%% End:
