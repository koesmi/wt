\documentclass{article}
\usepackage[a4paper,margin=1.875in,top=1.1in,bottom=1.1in]{geometry}

\usepackage{amsmath,mathtools,bbm,amssymb}
\usepackage[german]{babel}

\usepackage{setspace}
\doublespacing

\usepackage{fancyhdr}
\renewcommand{\headrulewidth}{0pt} 
\pagestyle{fancy}
\lhead{Blatt 5 Nicolas und Evgenij}\rhead{Seite \thepage}
\fancyfoot{}

\usepackage{tikz}
\usetikzlibrary{decorations.pathreplacing,arrows.meta}

\usepackage[numbers,round]{natbib}
\bibliographystyle{alphadin}
\usepackage{url}
\usepackage{hyperref}

\begin{document}

\paragraph{B5A1}
%https://math.stackexchange.com/questions/2529434/x-k-uniformly-integrable-iff-lim-n-to-infty-limsup-k-to-infty-mathbb-e
Zeigen Sie, dass eine Folge $(X_n)_{n\in\mathbb{N}}$ genau dann gleichgradig integrierbar ist, wenn
\begin{equation}
  \label{eq:ggiblim}\tag{1}
  \lim_{k\to\infty}\limsup_{n\to\infty}E[|X_n|\mathbbm{1}_{\{|X_n|>k\}}]=0\,.
\end{equation}
Sei zunächst $(X_n)$ gleichgradig integrierbar.
Dann gilt nach Definition von gleichgradiger Integrierbarkeit, dass
\begin{align*}
  0
  &=\lim_{k\to\infty}\sup_{n\in\mathbb{N}}E[|X_n|\mathbbm{1}_{\{|X_n|>k\}}]\,.
    \intertext{Insbesondere heißt dass, dass für ein gegebenes $n\in\mathbb{N}$ gilt, dass}
  0
  &=\lim_{k\to\infty}\sup_{m\geq n}E[|X_m|\mathbbm{1}_{\{|X_m|>k\}}]\,.
    \intertext{Damit gilt auch für das Infimum über diese $n\in\mathbb{N}$, dass}
    0
  &=\lim_{k\to\infty}\inf_{n\in\mathbb{N}}\sup_{m\geq n}E[|X_m|\mathbbm{1}_{\{|X_m|>k\}}]\,,
\end{align*}
was nach der Definition des $\limsup$ Gleichung (\ref{eq:ggiblim}) liefert.
\emph{Hier sollte man sich überlegen, ob man den Limes über $k$ ebenfalls, wie in der Argumentation unten, mittels eines $\varepsilon$ umschreiben sollte.}

Genüge nun $(X_n)$ Gleichung (\ref{eq:ggiblim}).
Wir wollen zeigen, dass $(X_n)$ gleichgradig integrierbar ist.
Sei hierfür ein $\varepsilon>0$ gegeben.
Da Gleichung (\ref{eq:ggiblim}) gilt, gibt es ein $k_0\in\mathbb{N}$, sodass für alle $k>k_0$
\begin{align*}
  \limsup_{n\to\infty}E[|X_n|\mathbbm{1}_{\{|X_n|>k\}}]
  &<\varepsilon\,.
    \intertext{
    \emph{Damit ist die Teilfolge $\bigl(\sup_{m\geq n}E[|X_m]\mathbbm{1}_{\{|X_m|>k\}}\bigr)_n$ für alle $n\in\mathbb{N}$ beschränkt.
    Hier könnte man eventuell noch direkt ein Argument bringen, warum schon Beschränktheit durch $\varepsilon$ folgt.}
    Also gibt es auch ein $n_0$, sodass für alle $n\geq n_0$ gilt}
    E[|X_n|\mathbbm{1}_{\{|X_n|>k\}}]
  &<\varepsilon\,.
    \intertext{Nach Lemma 24 gilt, dass alle $X_n$ integrierbar sind.
    Also gibt es für alle $X_1,\dots,X_{n_0-1}$ jeweils Schranken $k_1,\dots,k_{n_0-1}\in\mathbb{N}$, sodass für die $n=1,\dots,n_0-1$ gilt}
    E[|X_n|\mathbbm{1}_{\{|X_n|>k_n\}}]<\varepsilon\,.
    \intertext{Setze nun $K=k_0\vee\cdots\vee k_{n_0-1}$, dann gilt für alle $k\geq K$ und alle $n\in\mathbb{N}$, dass}
    E[|X_n|\mathbbm{1}_{\{|X_n|>k\}}]<\varepsilon\,,
    \intertext{also insbesondere, dass}
    \lim_{k\to\infty}\sup_{n\in\mathbb{N}}E[|X_n|\mathbbm{1}_{\{|X_n|>k\}}]<\varepsilon\,.
\end{align*}
\emph{Hier fehlt noch die Folgerung von Korollar 27.}
\newpage

\paragraph{B5A2}
Sei $(\Omega,{\cal F},P)$ ein Wahrscheinlichkeitsraum und $(A_n)_{n\geq 1}$ eine Folge in ${\cal F}$.
Es sei
\[
  \limsup_{n\to\infty}A_n:=\bigcap_{n=1}^\infty\bigcup_{k=n}^\infty A_k\,,\qquad\liminf_{n\to\infty}A_n:=\bigcup_{n=1}^\infty\bigcap_{k=n}^\infty A_k\,.
\]
Zeigen Sie, dass gilt
\begin{align*}
  \liminf_{n\to\infty}A_n
  &\subseteq\limsup_{n\to\infty}A_n\,,\\
  P\Bigl(\liminf_{n\to\infty}A_n\Bigr)\leq \liminf_{n\to\infty}P(A_n)
  &\leq\limsup_{n\to\infty}P(A_n)\leq P\Bigl(\limsup_{n\to\infty}A_n\Bigr)\,.
\end{align*}

Wir möchten zunächst zeigen, dass $\liminf_{n\to\infty}A_n\subseteq\limsup_{n\to\infty}A_n$.
Sei hierfür $\omega\in\liminf_{n\to\infty}A_n$. Wie wir in der Übung besprochen haben, heißt das, dass ein $n_0\in\mathbb{N}$ existiert, sodass für alle $n\geq n_0$ gilt, dass $\omega\in A_n$.
Um zu zeigen, dass $\omega\in\limsup_{n\to\infty}A_n$, sei ein $m_0\in\mathbb{N}$ beliebig vorgegeben.
Da für $n\geq n_0$ gilt, dass $\omega\in A_n$, ist $\omega\in A_{m_0\vee n_0}$.
Damit gilt auch für alle $m_0\in\mathbb{N}$, dass ein $m\geq m_0$ existiert, sodass $\omega\in A_m$, nämlich $m=m_0\vee n_0$.
Somit ist $\omega\in\limsup_{n\to\infty}A_n$.

Nun überlegen wir uns, warum die Ungleichungen gelten.
$\mathbbm{1}_{A_n}$ hat $\mathbbm{1}_\Omega$ als integrierbare Majorante.
Damit gelten die erste und dritte Ungleichung gelten jeweils nach dem Lemma von Fatou, \emph{beziehungsweise die dritte nach majorisierter Konvergenz}.

\emph{Hier fehlt noch eine Rechnung für die mittlere Ungleichung.}
\newpage

\paragraph{B5A3}
Sei $\lambda>0$ und für jedes $n\in\mathbb{N}_0$ sei $X_n$ eine Poisson-verteilte Zufallsvariable zum Parameter $\lambda$.
Zeigen Sie mit Hilfe des Lemmas von Borel--Cantelli, dass
\[
  P(\{\text{Für unendlich viele $n$ gilt $X_n>n$}\})=0\,.
\]

Die Poissonverteilung ist gegeben durch $P(X_n=k)=\mathrm{e}^{-\lambda}\frac{\lambda^k}{k!}$.
Damit ist
\begin{align*}
  \sum_{n=1}^\infty P(X_n>n)
  &=\sum_{n=1}^\infty\sum_{k=n+1}^\infty \mathrm{e}^{-\lambda}\frac{\lambda^k}{k!}
\end{align*}
Nach dem Lemma von Borel--Cantelli folgt die Behauptung aus $\sum_{n=1}^\infty P(X_n>n)<\infty$.
\emph{Also müsste der obige Ausdruck endlich sein.}
\newpage

\paragraph{B5A4}
Nennen Sie jeweils ein Beispiel und ein Gegenbeispiel von reellwertigen Zufallsvariablen $(X_n)_{n\in\mathbb{N}}$ und einer Menge $A\in\mathcal{B}(\mathbb{R})$ für die folgenden Identitäten
  \begin{alignat*}{2}
&\text{1.~}&P\Bigl(\Bigl\{\limsup_{n\to\infty}X_n\in A\Bigr\}\Bigr)&=P\Bigl(\limsup_{n\to\infty}\{X_n\in A\}\Bigr)\\
&\text{2.~}&P\Bigl(\limsup_{n\to\infty}\{X_n\in A\}\Bigr)&=\limsup_{n\to\infty}P(\{X_n\in A\})\\
&\text{3.~}&\limsup_{n\to\infty}P(\{X_n\in A\})&=P\Bigl(\Bigl\{\limsup_{n\to\infty}X_n\in A\Bigr\}\Bigr)
\end{alignat*}
Es ist
\begin{align*}
  P\Bigl(\Bigl\{\limsup_{n\to\infty}X_n\in A\Bigr\}\Bigr)
  &=P(\{\omega\in\Omega\mid \inf_{n\geq1}\sup_{m\geq n}X_m(\omega) \in A\})\,,\\
  P\Bigl(\limsup_{n\to\infty}\{X_n\in A\}\Bigr)
  &=P(\omega\in\Omega\mid \forall n\in\mathbb{N}\exists m\geq n~X_m(\omega)\in A)
\end{align*}
\newpage

\paragraph{B5A5}
Sei $X_1,X_2,\dots$ eine Folge von reellwertigen Zufallsvariablen auf einem Wahrscheinlichkeitsraum $(\Omega,{\cal F},P)$, ${\cal F}_n:=\sigma(X_n)$ für alle $n$ und $S_n:=\sum_{k=1}^nX_k$.
Zeigen Sie oder widerlegen Sie durch ein Gegenbeispiel, dass die folgenden Ereignisse terminale Ereignisse sind, also in der terminalen $\sigma$-Algebra liegen.
\begin{align*}
  \text{1.~}&\{\omega\in\Omega|X_n(\omega)=0\}\text{ für ein $n\in\mathbb{N}$,}\\
  \text{2.~}&\{\omega\in\Omega|X_n(\omega)=0\text{ für ein $n\in\mathbb{N}$,}\}
\end{align*}
\newpage

\bibliography{../../../books/wt}
\end{document}

%%% Local Variables:
%%% mode: latex
%%% ispell-local-dictionary: "german"
%%% TeX-master: t
%%% End:
