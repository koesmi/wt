\documentclass{article}
\usepackage[a4paper,margin=1.875in,top=1.5in]{geometry}

\usepackage{amsmath,mathtools,bbold}
\usepackage[german]{babel}

\usepackage{setspace}
\doublespacing

\usepackage{fancyhdr}
\renewcommand{\headrulewidth}{0pt} 
\pagestyle{fancy}
\lhead{Seite \thepage}\rhead{Blatt 2 Nicolas und Evgenij}
\fancyfoot{}

\begin{document}

\paragraph{B2A5}
Wir bestimmen zunächst das unbestimmte Integral
\begin{align*}
  I(a,b)
  &=\int_a^b\alpha\beta x^{\beta-1}\mathrm{e}^{-\alpha x^\beta}\mathrm{d}x\,.
  \intertext{Substitution $z=x^\beta$, sodass $\frac{\mathrm{d}z}{\mathrm{d}x}=\beta x^{\beta-1}$ und damit $x^{\beta-1}\mathrm{d}x=\frac{\mathrm{d}z}{\beta}$ ergibt}
  &=\int_{a^\beta}^{b^\beta}\alpha\mathrm{e}^{-\alpha z}\mathrm{d}z\,.
    \shortintertext{Integrieren liefert}
  &=\left.-\mathrm{e}^{-\alpha z}\right|_{z=a^\beta}^{b^\beta}=\mathrm{e}^{-\alpha a^\beta}-\mathrm{e}^{-\alpha b^\beta}\,.
\end{align*}
Um herauszufinden was $F^Y(x)$ ist, bestimmen wir die einzelnen Wahrscheinlichkeiten für $P(Y=1)$ und $P(Y=X)$.
Nach Definition von $Y$ ist $P(Y<1)=0$, $P(Y=1)=P(X\leq1)=I(0,1)=1-\mathrm{e}^{-\alpha}$ und $P(1<Y\leq x)=P(1<X\leq x)=I(1,x)=\mathrm{e}^{-\alpha}-\mathrm{e}^{-\alpha x^\beta}$.
Somit ist
\begin{align*}
  F^Y(x)
  &=P^Y(Y\leq x)=
  \begin{cases}
    0\,,&x<1\\
    P(Y=1)\,,&x=1\\
    P(Y=1)+P(1<Y\leq x)\,,&x>1
  \end{cases}\\
  &=
  \begin{cases}
    0\,,&x<1\\
    1-\mathrm{e}^{-\alpha}\,,&x=1\\
    1-\mathrm{e}^{-\alpha x^\beta}\,,&x>1
  \end{cases}\\
  &=\mathbb{1}_{[1,\infty)}\bigl(1-\mathrm{e}^{-\alpha x^\beta}\bigr)\,.
\end{align*}

Wir bestimmen zunächst das Maß $P^X\bigl((a,b]\bigr)$.
Nach Satz 6 und mit Ausführen der Indikatorfunkton in $f_X(x)$ gilt
\begin{align*}
  P\bigl((a,b]\bigr)
  &=
    \begin{cases}
      0\,,&b\leq0\\
      \int_0^b\mathbb{1}_{(0,\infty)}(x)\alpha\beta x^{\beta-1}\mathrm{e}^{-\alpha x^\beta}\mathrm{d}x\,&a\leq0,b>0\\
      \int_a^b\mathbb{1}_{(0,\infty)}(x)\alpha\beta x^{\beta-1}\mathrm{e}^{-\alpha x^\beta}\mathrm{d}x\,&a\leq0,b>0
\end{cases}
\end{align*}
Nach Definition von $Y$ und $F_X(x)$ ist
\begin{align*}
  P(Y=1)
  &=P^X\bigl((-\infty,1]\bigr)
    =\int_{-\infty}^1\mathbb{1}_{(0,\infty)}(y)\alpha\beta y^{\beta-1}\mathrm{e}^{-\alpha y^\beta}\mathrm{d}y\,.
    \shortintertext{Durch Anwendung der Indikatorfunktion}
  &=\int_{0}^1\alpha\beta y^{\beta-1}\mathrm{e}^{-\alpha y^\beta}\mathrm{d}y\,.
    \shortintertext{Substitution $z=y^\beta$, sodass $\frac{\mathrm{d}z}{\mathrm{d}y}=\beta y^{\beta-1}$ und damit $y^{\beta-1}\mathrm{d}y=\frac{\mathrm{d}z}{\beta}$ ergibt}
  &=\int_0^{1^\beta}\alpha\mathrm{e}^{-\alpha z}\mathrm{d}z\,.
    \shortintertext{Integrieren liefert}
  &=\left.-\mathrm{e}^{-\alpha z}\right|_{z=0}^1=1-\mathrm{e}^{-\alpha}\,.
\end{align*}
Es gilt für reellwertige Zufallsvariablen\vspace{-1em}
\begin{align*}
  F^Y(x)
  &=P(Y\leq x)\,.
    \shortintertext{Nach Definition von $Y=\max\{X,1\}$ ist das}
  &=P(1\leq x,Y\leq x)=\mathbb{1}_{(1,\infty)}(x)P(Y\leq x)\,.
  &\shortintertext{Einsetzen der Dichte $f_X(y)=\mathbb{1}_{(0,\infty)}(y)\alpha\beta y^{\beta-1}\mathrm{e}^{-\alpha y^\beta}$ liefert}
  &=\mathbb{1}_{(1,\infty)}(x)\int_{-\infty}^x\mathbb{1}_{(0,\infty)}(y)\alpha\beta y^{\beta-1}\mathrm{e}^{-\alpha y^\beta}\mathrm{d}y\,.
    \shortintertext{und mit Verschieben der inneren Indikatorfunkton in die Integralgrenze}
  &=\mathbb{1}_{(1,\infty)}(x)\int_0^x\alpha\beta y^{\beta-1}\mathrm{e}^{-\alpha y^\beta}\mathrm{d}y\,.
    \shortintertext{Substitution $z=y^\beta$, sodass $\frac{\mathrm{d}z}{\mathrm{d}y}=\beta y^{\beta-1}$ und damit $y^{\beta-1}\mathrm{d}y=\frac{\mathrm{d}z}{\beta}$ ergibt}
  &=\mathbb{1}_{(1,\infty)}(x)\int_0^{x^\beta}\alpha\mathrm{e}^{-\alpha z}\mathrm{d}z\,.
    \shortintertext{Integrieren liefert}
  &=\left.-\mathbb{1}_{(1,\infty)}(x)\mathrm{e}^{-\alpha z}\right|_{z=0}^{x^\beta}=\mathbb{1}_{(1,\infty)}(1-\mathrm{e}^{-\alpha x^\beta})\,.
\end{align*}
\end{document}

%%% Local Variables:
%%% mode: latex
%%% ispell-local-dictionary: "german"
%%% TeX-master: t
%%% End:
