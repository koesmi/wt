\documentclass{article}
\usepackage[a4paper,margin=1.875in,top=1.2in,bottom=1.2in]{geometry}

\usepackage{amsmath,mathtools,bbm,amssymb}
\usepackage[german]{babel}

\usepackage{setspace}
\doublespacing

\usepackage{fancyhdr}
\renewcommand{\headrulewidth}{0pt} 
\pagestyle{fancy}
\lhead{Blatt 2 Nicolas und Evgenij}\rhead{Seite \thepage}
\fancyfoot{}

\usepackage{tikz}
\usetikzlibrary{decorations.pathreplacing,arrows.meta}

\usepackage[numbers,round]{natbib}
\bibliographystyle{alphadin}

\begin{document}

\paragraph{B2A1}
Wir wollen zunächst $\sigma\bigl(f^{-1}({\cal C}')\bigr)\subseteq f^{-1}\bigl(\sigma({\cal C}')\bigr)$ zeigen.
Da es gilt ${\cal C}\subseteq\sigma({\cal C}')$, ist auch $f^{-1}({\cal C}')\subseteq f^{-1}\bigl(\sigma({\cal C}')\bigr)$.
Da $\sigma({\cal C}')$ eine $\sigma$-Algebra ist, ist auch $f^{-1}\bigl(\sigma({\cal C}')\bigr)$ eine $\sigma$-Algebra.
Damit muss $\sigma\bigl(f^{-1}({\cal C}')\bigr)\subseteq f^{-1}\bigl(\sigma({\cal C}')\bigr)$ gelten.

Betrachte für die andere Inklusion das System der guten Mengen gegeben durch ${\cal F}'=\{A'\in\sigma({\cal C}')\mid f^{-1}(A')\in\sigma\bigl(f^{-1}({\cal C}')\bigr)\}$.
Wir zeigen, dass ${\cal F}'$ eine $\sigma$-Algebra ist, denn dann wird auch $\sigma({\cal C}')\subseteq{\cal F}'$ gelten.
Da $f$ eine Abbildung ist, also jedem Punkt aus $\Omega$ einen in $\Omega'$ zuordnet, ist $f^{-1}(\Omega')=\Omega$.
Da $\sigma(f^{-1}({\cal C}))$ eine $\sigma$-Algebra ist, ist $\Omega\in\sigma(f^{-1}({\cal C}))$.
Somit ist $\Omega'\in{\cal F}.$.
Entsprechendes gilt für $\emptyset$. Allen Punkten wird einer zugeordnet, nichts bleibt ohne Bild und somit $f^{-1}(\emptyset)=\emptyset$, also $\emptyset\in{\cal F}'$.
Sei $A'\in{\cal F}'$, dann gilt nach Definition von ${\cal F}'$, dass $f^{-1}(A')\in\sigma(f^{-1}({\cal C}'))$ und, da $\sigma(f^{-1}({\cal C}'))$ eine $\sigma$-Algebra ist, auch $(f^{-1}(A'))^\mathrm{c}=f^{-1}({A'}^\mathrm{c})\in\sigma(f^{-1}({\cal C}'))$ (haben wir uns in der letzten Übung ausführlich überlegt), was wieder nach Definition von ${\cal F}'$ bedeutet, dass ${A'}^\mathrm{c}\in{\cal F}'$.
Sei schließlich $(A_n')$ eine Folge in ${\cal F}'$, also für alle $n$ aus der natürlichen Zahlen $f^{-1}(A_n')\in\sigma(f^{-1}({\cal C}'))$, dann gilt wieder, da $\sigma(f^{-1}({\cal C}'))$ eine $\sigma$-Algebra ist, dass $\bigcup f^{-1}(A_n')=f^{-1}\bigl(\bigcup A_n'\bigr)\in\sigma(f^{-1}({\cal C}'))$.
Nach Definition von ${\cal F}'$ ist hierdurch auch $\bigcup A_n'\in{\cal F}'$ und insgesamt ${\cal F}'$ eine $\sigma$-Algebra. Da $f^{-1}({\cal C}')\subseteq \sigma(f^{-1}({\cal C}'))$ ist ${\cal C}'\subseteq{\cal F}'$.
Wie schon angekündigt ist, da ${\cal F}'$ eine $\sigma$-Algebra ist, auch $\sigma({\cal C}')\subseteq{\cal F}'$ als kleinste $\sigma$-Algebra, die ${\cal C}'$ enthält.
Schaut man auf die Definition von ${\cal F}'$, so gilt hierdurch die andere gesuchte Inklusion $f^{-1}\bigl(\sigma({\cal C}')\bigr)\subseteq\sigma\bigl(f^{-1}({\cal C}')\bigr)$.
\newpage
\paragraph{B2A2}
Sei zunächst $\mu=\nu$ auf ${\cal B}(\mathbb{R})$.
Dann kann ich statt $\int_\Omega f\mathrm{d}\mu$ auch $\int_\Omega f\mathrm{d}\nu$ schreiben.
Somit ist $\mu[f]=\nu[f]$.

Sei nun anders herum $\mu[f]=\nu[f]$.
Ich will folgende Funktionen $f_n$ konstruieren, die $\mathbbm{1}_{[a,b]}$ approximieren sollen.
\begin{center}
  \begin{tikzpicture}
    \draw[->](-3,0)--(-2,0)node[below]{$a-\frac{1}{n}$}--(-1,0)node[below]{$a$}--(1,0)node[below]{$b$}--(2,0)node[below]{$b+\frac{1}{n}$}--(3,0);
    \draw(-2,-2pt)--(-2,2pt);
    \draw(-1,-2pt)--(-1,2pt);
    \draw(1,-2pt)--(1,2pt);
    \draw(2,-2pt)--(2,2pt);
    \draw(-2,0)to node[midway,above left]{$f_n$}(-1,1)--(1,1)--(2,0);
    \draw[decorate,decoration=brace](-2.5,0)to node[midway,left]{1}(-2.5,1);
  \end{tikzpicture}
\end{center}
$f_n$ hat zwischen $a$ und $b$ die Höhe 1.
Zudem gilt $f_n(a-\frac{1}{n})=u\cdot(a-\frac{1}{n})+v=0$ und $f_n(a)=u\cdot a+v=1$, sodass $f_n$ links die Form $1+n(x-a)$ hat.
Analog hat es rechts die Form $1+n(b-x)$.
Definiere also $f_n$, was die $\mathbbm{1}[a,b]$ approximiert, als $f_n(x)=\mathbbm{1}_{[a,b]}+(1+n(x-a))\mathbbm{1}_{[a-\frac{1}{n},a)}+(1+n(b-x))\mathbbm{1}_{(b,b+\frac{1}{n}]}$.
Dann ist
\begin{align*}
  \mu([a,b])
  &=\int_\Omega\mathbbm{1}_{[a,b]}(x)\mu(\mathrm{d}x)\,.
    \intertext{Durch die gerade gefundene Approximation können wir schreiben}
  &=\int_\Omega\lim_{n\to\infty}f_n(x)\mu(\mathrm{d}x)\,.
    \intertext{Die $f_n$ konvergieren monoton gegen $\mathbbm{1}_{[a,b]}$. Deshalb können wir den Satz über monotone Konvergenz anwenden und den Limes rausziehen, sodass}
  &=\lim_{n\to\infty}\int_\Omega f_n(x)\mu(\mathrm{d}x)\,.
    \intertext{Nun können wir unsere Annahme verwenden, dass für stetige Funktionen mit kompaktem Träger, die die $f_n$ ja sind, $\mu[f_n]$ und $\nu[f_n]$ übereinstimmen, also}
  &=\lim_{n\to\infty}\int_\Omega f_n(x)\nu(\mathrm{d}x)\,.
    \intertext{Gehen wir die Argumentation mit $\nu$ statt $\mu$ rückwärts durch, erhalten wir}
  &=\nu([a,b])\,.
\end{align*}
Da $\mu$ und $\nu$ auf dem Erzeuger $\{[a,b]\}_{a,b\in\mathbb{Q}}$ von ${\cal B}(\mathbb{R})$ übereinstimmen, gilt nach dem Eindeutigkeitssatz A.16 $\mu=\nu$.
\newpage
\paragraph{B2A3}
Zur Teilaufgabe (a), sei $\mu(A)<\varepsilon$ und $\nu(A^\mathrm{c})<\varepsilon$ für alle $\varepsilon>0$, dann ist im Limes $\varepsilon\to0$ $\mu(A)=0$ und $\nu(A^\mathrm{c})=0$, also $\mu\perp\nu$.

Zur Teilaufgabe (b), da $\mu\perp\nu$, gibt es ein $A\in{\cal A}$ so, dass $\mu(A)=0$ und $\nu(A^{\mathrm{c}})=0$.
Da $\lambda\ll\mu$, ist auch $\lambda(A)=0$ und somit $\lambda\perp\nu$.

Zur Teilaufgabe (c), hier gibt es ein $A\in{\cal A}$, sodass $\mu(A)=0$ und $\mu(A^\mathrm{c})=0$, also $\mu(\Omega)=0$ und somit $\mu({\cal A})=0$ wegen der Monotonie von $\mu$.
\newpage
\paragraph{B2A4}
Aus Proposition 15.5 aus \cite{nielsen}.
Wir zeigen zunächst, dass wenn $\nu\ll\mu$, dann (b) gilt, dass also für jedes $\varepsilon>0$ ein $\delta>0$ existiert, sodass für alle $A\in{\cal A}$ mit $\mu(A)\leq\delta$ auch $\nu(A)\leq\varepsilon$.
Angenommen, (b) stimmt gar nicht, wenn (a) gilt.
Es gibt also ein $\varepsilon>0$, sodass wie klein ich mein $\delta$ auch wähle es immer noch ein $A\in{\cal A}$ mit $\mu(A)\leq\delta$ gibt, jedoch $\nu(A)>\varepsilon$.
Wähle für dieses $\varepsilon$ ein $n\in\mathbb{N}$ beliebig und setze $\delta=\frac{1}{2^n}$.
Weiter soll $A_n\in{\cal A}$ die eine Menge sein, für die gilt $\mu(A_n)\leq\frac{1}{2^n}$, aber $\nu(A_n)>\varepsilon$.
Dann ist, für $A=\lim\sup A_n$ wegen Monotonie von $\mu$ auch
\begin{align*}
  \mu(A)
  &=\mu\Bigl(\bigcap_{n=1}^\infty\bigcup_{k=n}^\infty A_k\Bigr)\leq\mu\Bigl(\bigcup_{k=n}^\infty A_k\Bigr)\,,
    \intertext{und wiederum wegen der $\sigma$-Subadditivität von $\mu$}
  &\leq\sum_{k=n}^\infty\mu(A_k)\leq\sum_{k=n}^\infty\frac{1}{2^k}\,.
  \intertext{Wir wollen jetzt bestimmen, was $\sum_{k=n}^\infty\frac{1}{2^k}$ ist.
Sei hierfür $s_{m,n}=\sum_{k=n}^m\frac{1}{2^k}$.
Dann ist $\frac{1}{2}s_{m,n}=s_{m,n}-\frac{1}{2}s_{m,n}=\sum_{k=n}^m\bigl(\frac{1}{2^k}-\frac{1}{2^{k+1}}\bigr)=\frac{1}{2^n}-\frac{1}{2^{m+1}}$ und somit}
    &=\lim_{m\to\infty}2\Bigl(\frac{1}{2^n}-\frac{1}{2^{m+1}}\Bigr)=\frac{1}{2^{n-1}}\,,
\end{align*}
für alle $n$, sodass $\mu(A)=0$.
Andererseits gilt wegen der Monotonie von $\nu$, dass $\nu\bigl(\bigcup_{k=n}^\infty A_k\bigr)\geq\nu(A_n)>\varepsilon$ nach der Wahl der $A_n$, ebenfalls für alle $n$.
Da $\nu$ endlich ist, ist $\nu(A_1)<\infty$.
Deshalb ist $\nu$ nach Satz A.14 stetig von oben, also $\nu(A)=\lim_{n\to\infty}\nu\bigl(\bigcup_{k=n}^\infty A_k\bigr)>\varepsilon$, da ja $\nu\bigl(\bigcup_{k=n}^\infty A_k\bigr)>\varepsilon$ für beliebiges $n$.
Da wir oben gezeigt haben, dass $\mu(A)=0$, widerspricht das $\nu\ll\mu$, also muss (b) doch gelten.

Nun nehmen wir andererseits an, dass für alle $\varepsilon>0$ es ein $\delta>0$ gibt, sodass $\nu(A)\leq\varepsilon$ falls $\mu(A)\leq\delta$ und wollen zeigen, dass $\nu\ll\mu$, also, dass für ein $A\in{\cal A}$ mit $\mu(A)=0$ auch $\nu(A)=0$ gilt.
Sei hierfür $A\in{\cal A}$ mit $\mu(A)=0$ gegeben.
Dann ist $\mu(A)\leq\delta$, für ein beliebiges $\delta>0$, also nach Annahme, dass (b) gilt, auch $\nu(A)\leq\varepsilon$ für alle $\varepsilon>0$.
Damit muss $\nu(A)=0$ gelten.
\newpage
\paragraph{B2A5}
Wir bestimmen zunächst das unbestimmte Integral
\begin{align*}
  I(a,b)
  &=\int_a^b\alpha\beta x^{\beta-1}\mathrm{e}^{-\alpha x^\beta}\mathrm{d}x\,.
  \intertext{Substitution $z=x^\beta$, sodass $\frac{\mathrm{d}z}{\mathrm{d}x}=\beta x^{\beta-1}$ und damit $x^{\beta-1}\mathrm{d}x=\frac{\mathrm{d}z}{\beta}$ ergibt}
  &=\int_{a^\beta}^{b^\beta}\alpha\mathrm{e}^{-\alpha z}\mathrm{d}z\,.
    \shortintertext{Integrieren liefert}
  &=\left.-\mathrm{e}^{-\alpha z}\right|_{z=a^\beta}^{b^\beta}=\mathrm{e}^{-\alpha a^\beta}-\mathrm{e}^{-\alpha b^\beta}\,.
\end{align*}
Um herauszufinden was $F^Y(x)$ ist, bestimmen wir die einzelnen Wahrscheinlichkeiten für $P(Y=1)$ und $P(Y=X)$.
Nach Definition von $Y$ ist $P(Y<1)=0$, $P(Y=1)=P(X\leq1)=I(0,1)=1-\mathrm{e}^{-\alpha}$ und $P(1<Y\leq x)=P(1<X\leq x)=I(1,x)=\mathrm{e}^{-\alpha}-\mathrm{e}^{-\alpha x^\beta}$.
Somit ist
\begin{align*}
  F^Y(x)
  &=P^Y(Y\leq x)=
  \begin{cases}
    0\,,&x<1\\
    P(Y=1)\,,&x=1\\
    P(Y=1)+P(1<Y\leq x)\,,&x>1
  \end{cases}\\
  &=
  \begin{cases}
    0\,,&x<1\\
    1-\mathrm{e}^{-\alpha}\,,&x=1\\
    1-\mathrm{e}^{-\alpha x^\beta}\,,&x>1
  \end{cases}\\
  &=\mathbbm{1}_{[1,\infty)}(x)\bigl(1-\mathrm{e}^{-\alpha x^\beta}\bigr)\,.
\end{align*}
Wir stellen fest, dass aufgrund der Indikatorfunktion gilt $F^Y(-\infty)=0$, und wegen $\lim_{x\to\infty}\mathrm{e}^{-x}$ gilt $F^Y(\infty)=1$.
Zudem ist $\mathrm{e}^{-x}$ monoton fallend, sodass $F^Y(x)$ monoton wachsend ist.
Schließlich gilt $F^Y(x)\rightarrow F^Y(1)$ für $x\searrow 1$, sodass $F^Y$ rechtsseitig stetig ist.
Damit ist $F^Y$ eine Verteilungsfunktion.

Mit $\rho\bigl((a,b]\bigr)=I(a,b)=\mathrm{e}^{-\alpha a^\beta}-\mathrm{e}^{-\alpha b^\beta}$ können $P^Y$ dann wie folgt auf $\{(a,b]\}_{a,b\in\mathbb{Q}}$ zerlegen
\begin{align*}
  P^Y\bigl((a,b]\bigr)
  &=I(0,1)\delta_1\bigl((a,b]\bigr)+\rho\bigl((1,\infty)\cap(a,b]\bigr)\\
  &=(1-\mathrm{e}^{-\alpha})\delta_1\bigl((a,b]\bigr)+\rho\bigl((1,\infty)\cap(a,b]\bigr)\,.
\end{align*}
Wir prüfen zunächst nach, dass die Zerlegung wirklich $F^Y(b)-F^Y(a)$ ergibt.
Für $b<1$ verschwinden sowohl das Diracmaß $\delta_1$ als auch das Lebesguemaß $\rho$, sodass $P^Y\bigl((a,b]\bigr)=0$.
Für $a<1$ und $b\geq1$ ist $(1,\infty)\cap(a,b]=(1,b]$ und somit
\begin{align*}
  P^Y\bigl((a,b]\bigr)
  &=1-\mathrm{e}^{-\alpha}+\mathrm{e}^{-\alpha}-\mathrm{e}^{-\alpha b^\beta}=1-\mathrm{e}^{-\alpha b^\beta}=F^Y(b)-F^Y(a)\,.
\end{align*}
Für $a\geq1$, $b>1$ ist schließlich $(1,\infty)\cap(a,b]=(a,b]$, sodass wieder
\begin{align*}
  P^Y\bigl((a,b]\bigr)
  &=\mathrm{e}^{-\alpha a^\beta}-\mathrm{e}^{-\alpha b^\beta}=F^Y(b)-F^Y(a)\,.
\end{align*}
Da $F^Y$, wie vorhin erklärt, eine Verteilungsfunktion ist, ist nach Satz 6 $P^Y$ ein Maß auf ganz ${\cal B}(\mathbb{R})$, $I(a,b)$ muss hierbei entsprechend auf den einzelnen Komponenten eines $A\in{\cal B}(\mathbb{R})$ ausgewertet und aufsummiert werden.

Jetzt überlegen wir uns noch, dass der Diracterm singulär und der Lebesgueterm absolut stetig bezüglich des Lebesguemaßes ist.
Da $\lambda(\{1\})=0$ und $\delta_1(\mathbb{R}\setminus\{1\})=0$ ist der erste Term tatsächlich singulär.
Für den zweiten Term benutzen wir den Satz von Radon--Nikodym in der Ausführung als Korollar 7.34 in \cite{klenke}.
Demnach ist der Lebesgueterm absolut stetig bezüglich des Lebesguemaßes, wenn es eine Dichte bezüglich des Lebesguemaßes hat.
Nach Konstruktion gilt gerade $\rho\bigl((1,\infty)\cap(a,b]\bigr)=\int_{(a,b]}\mathbbm{1}_{(1,\infty)}\alpha\beta x^{\beta-1}\mathrm{e}^{-\alpha x^\beta}\lambda(\mathrm{d}x)$ für alle $(a,b]$, sodass $\mathbbm{1}_{(1,\infty)}\alpha\beta x^{\beta-1}\mathrm{e}^{-\alpha x^\beta}$ die Dichte von $A\mapsto\rho\bigl((1,\infty)\cap A\bigr)$ für alle $A\in{\cal B}(\mathbb{R})$ ist.
Nach dem Satz von Radon--Nikodym ist das Maß $A\mapsto\rho\bigl((1,\infty)\cap A\bigr)$ somit absolut stetig bezüglich des Lebesguemaßes $\lambda$.
\bibliography{../../../books/wt}
\end{document}

%%% Local Variables:
%%% mode: latex
%%% ispell-local-dictionary: "german"
%%% TeX-master: t
%%% End:
