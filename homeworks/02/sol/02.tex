\documentclass{article}
\usepackage[a4paper,margin=1.875in,top=1.5in]{geometry}

\usepackage{amsmath,mathtools,bbold}
\usepackage[german]{babel}

\usepackage{setspace}
\doublespacing

\usepackage{fancyhdr}
\renewcommand{\headrulewidth}{0pt} 
\pagestyle{fancy}
\lhead{Blatt 2 Nicolas und Evgenij}\rhead{Seite \thepage}
\fancyfoot{}

\begin{document}

\paragraph{B2A5}
Wir bestimmen zunächst das unbestimmte Integral
\begin{align*}
  I(a,b)
  &=\int_a^b\alpha\beta x^{\beta-1}\mathrm{e}^{-\alpha x^\beta}\mathrm{d}x\,.
  \intertext{Substitution $z=x^\beta$, sodass $\frac{\mathrm{d}z}{\mathrm{d}x}=\beta x^{\beta-1}$ und damit $x^{\beta-1}\mathrm{d}x=\frac{\mathrm{d}z}{\beta}$ ergibt}
  &=\int_{a^\beta}^{b^\beta}\alpha\mathrm{e}^{-\alpha z}\mathrm{d}z\,.
    \shortintertext{Integrieren liefert}
  &=\left.-\mathrm{e}^{-\alpha z}\right|_{z=a^\beta}^{b^\beta}=\mathrm{e}^{-\alpha a^\beta}-\mathrm{e}^{-\alpha b^\beta}\,.
\end{align*}
Um herauszufinden was $F^Y(x)$ ist, bestimmen wir die einzelnen Wahrscheinlichkeiten für $P(Y=1)$ und $P(Y=X)$.
Nach Definition von $Y$ ist $P(Y<1)=0$, $P(Y=1)=P(X\leq1)=I(0,1)=1-\mathrm{e}^{-\alpha}$ und $P(1<Y\leq x)=P(1<X\leq x)=I(1,x)=\mathrm{e}^{-\alpha}-\mathrm{e}^{-\alpha x^\beta}$.
Somit ist
\begin{align*}
  F^Y(x)
  &=P^Y(Y\leq x)=
  \begin{cases}
    0\,,&x<1\\
    P(Y=1)\,,&x=1\\
    P(Y=1)+P(1<Y\leq x)\,,&x>1
  \end{cases}\\
  &=
  \begin{cases}
    0\,,&x<1\\
    1-\mathrm{e}^{-\alpha}\,,&x=1\\
    1-\mathrm{e}^{-\alpha x^\beta}\,,&x>1
  \end{cases}\\
  &=\mathbb{1}_{[1,\infty)}(x)\bigl(1-\mathrm{e}^{-\alpha x^\beta}\bigr)\,.
\end{align*}
Mit $\rho\bigl((a,b]\bigr)=I(a,b)=\mathrm{e}^{-\alpha a^\beta}-\mathrm{e}^{-\alpha b^\beta}$ können $P^Y$ dann wie folgt auf $\{(a,b]\}_{a,b\in\mathbb{Q}}$ zerlegen
\begin{align*}
  P^Y\bigl((a,b]\bigr)
  &=I(0,1)\delta_1\bigl((a,b]\bigr)+\rho\bigl((1,\infty)\cap(a,b]\bigr)\\
  &=(1-\mathrm{e}^{-\alpha})\delta_1\bigl((a,b]\bigr)+\rho\bigl((1,\infty)\cap(a,b]\bigr)\,.
\end{align*}
Wir prüfen zunächst nach, dass die Zerlegung wirklich $F^Y(b)-F^Y(a)$ ergibt.
Für $b<1$ verschwinden sowohl das Dirac- als auch das Lebesguemaß, sodass $P^Y\bigl((a,b]\bigr)=0$.
Für $a<1$ und $b\geq1$ ist $(1,\infty)\cap(a,b]=(1,b]$ und somit
\begin{align*}
  P^Y\bigl((a,b]\bigr)
  &=1-\mathrm{e}^{-\alpha}+\mathrm{e}^{-\alpha}-\mathrm{e}^{-\alpha b^\beta}=1-\mathrm{e}^{-\alpha b^\beta}=F^Y(b)-F^Y(a)\,.
\end{align*}
Für $a\geq1$, $b>1$ ist schließlich $(1,\infty)\cap(a,b]=(a,b]$, sodass wieder
\begin{align*}
  P^Y\bigl((a,b]\bigr)
  &=\mathrm{e}^{-\alpha a^\beta}-\mathrm{e}^{-\alpha b^\beta}=F^Y(b)-F^Y(a)\,.
\end{align*}
Jetzt überlegen wir uns noch, dass der Diracterm singulär und der Lebesgueterm absolut stetig bezüglich des Lebesguemaßes ist.
Da $\lambda(\{1\})=0$ und $\delta_1(\mathbb{R}\setminus\{1\})=0$ ist der erste Term tatsächlich singulär.
Für den zweiten Term benutzen wir Aufgabe 4.
Sei ein $\epsilon>0$ gegeben.
Der Lebesgueterm trägt nur für $a,b>1$ bei.
Für große $\alpha$ und $\beta$, ist dann $\mathrm{e}^{-\alpha a^\beta}-\mathrm{e}^{-\alpha b^\beta}<\epsilon$, sobald $b-a<\epsilon$.
Da $\lim_{\alpha\to0}\mathrm{e}^{-\alpha}=1$ und $\lim_{\beta\to 0}\mathrm{e}^{-x^\beta}=\frac{1}{e}$, gilt die obige Beziehung auch für kleine $\alpha$ und $\beta$. Somit ist $\rho\bigl(A\bigr)\leq\epsilon$ für alle $A$, mit $\mu(A)\leq \delta$, sobald ich $\delta<\epsilon$ wähle und nach Aufgabe 4 tatsächlich $\rho\ll\lambda$.
\end{document}

%%% Local Variables:
%%% mode: latex
%%% ispell-local-dictionary: "german"
%%% TeX-master: t
%%% End:
