\documentclass{article}
\usepackage[a4paper,margin=1.875in,top=1.875in,bottom=1.875in]{geometry}

\usepackage{amsmath,mathtools,bbm,amssymb}
\usepackage[german]{babel}

\usepackage{setspace}
\doublespacing

\usepackage{fancyhdr}
\renewcommand{\headrulewidth}{0pt} 
\pagestyle{fancy}
\lhead{Blatt 8 Nicolas und Evgenij}\rhead{Seite \thepage}
\fancyfoot{}

\usepackage{tikz}
\usetikzlibrary{decorations.pathreplacing,arrows.meta}

\usepackage[numbers]{natbib}
\bibliographystyle{alphadin}
\usepackage{url}
\usepackage{hyperref}

\begin{document}

\paragraph{B8A2}
Untersuchen Sie die folgenden Verteilungsklassen auf Faltungsstabilität.
\begin{enumerate}
\item[(a)] $\{{\cal N}(0,\sigma^2):\sigma^2>0\}$, die Klasse der Normalverteilungen mit Erwartungswert 0.
\end{enumerate}
% https://math.stackexchange.com/questions/2342797/finding-the-convolution-of-two-independent-standard-normal-distributed-random-v
% https://www.youtube.com/watch?v=345SOnIN6z4
\begin{enumerate}
\item[(b)] $\{\operatorname{Exp}(\lambda):\lambda>0\}$ die Klasse der Exponentialverteilungen.
\end{enumerate}
\emph{Eine Klasse $M$ von Verteilungen heißt stabil unter Faltung, wenn für alle $\mu,\nu\in M$ gilt, dass $\mu*\nu\in M$.}
  
\emph{Zur Erinnerung: Die oben genannten Verteilungen sind gegeben durch ihre Dichten}
\begin{align*}
  f_{{\cal N}(\mu,\sigma^2)}(x)&=\Bigl(\sqrt{2\pi\sigma^2}\Bigr)^{-1}\exp\Bigl(-\frac{(x-\mu)^2}{2\sigma^2}\Bigr)\quad\text{und}\\
f_{\operatorname{Exp}(\lambda)}(x)&=\lambda\exp(-\lambda x)\mathbbm{1}_{[0,\infty)}(x)\,.
\end{align*}

Zur (a), wir sollen also zeigen, dass es ein $\sigma_3$ existiert, sodass ${\cal N}(0,\sigma_1^2)*{\cal N}(0,\sigma_2^2)={\cal N}(0,\sigma_3^2)$.
In Aufgabe 5 zeigen wir, dass die Dichte zu ${\cal N}(0,\sigma_3^2)$ dann gegeben sein muss durch
\begin{align*}
  f_{3}(x)
  &=\int f_1(y)f_2(x-y)\lambda(\mathrm{d}y)\,.
    \intertext{wir setzen die einzelnen Dichten ein und erhalten}
  &=\int\Bigl(\sqrt{2\pi\sigma_1^2}\Bigr)^{-1}\exp\Bigl(-\frac{y^2}{2\sigma_1^2}\Bigr)\Bigl(\sqrt{2\pi\sigma_2^2}\Bigr)^{-1}\exp\Bigl(-\frac{(x-y)^2}{2\sigma_2^2}\Bigr)\lambda(\mathrm{d}y)\,.
    \intertext{Faktoren, die nicht von $y$ abhängen, können wir herausziehen, sodass}
  &=\Bigl(2\pi\sqrt{\sigma_1^2\sigma_2^2}\Bigr)^{-1}\exp\Bigl(-\frac{x^2}{2\sigma_2^2}\Bigr)\int\exp\Bigl(-\frac{y^2}{2\sigma_1^2}+\frac{xy}{\sigma_2^2}-\frac{y^2}{2\sigma_2^2}\Bigr)\lambda(\mathrm{d}y)
\end{align*}
Nun wollen wir den Exponenten im Integral quadratisch ergänzen.
Dazu fassen wir zunächst den Term proportional zu $y$ zusammen, also
\begin{align*}
  -\frac{y^2}{2\sigma_1^2}+\frac{xy}{\sigma_2^2}-\frac{y^2}{2\sigma_2^2}
  &=-\frac{\sigma_1^2+\sigma_2^2}{2\sigma_1^2\sigma_2^2}y^2+\frac{xy}{\sigma_2^2}\,.
    \intertext{wir erweiter so, dass man den Vorfaktor von $y^2$ aus dem zweiten Term ausklammern kann, sodass}
  &=-\frac{\sigma_1^2+\sigma_2^2}{2\sigma_1^2\sigma_2^2}\Bigl(y^2-2\frac{\sigma_1^2x}{\sigma_1^2+\sigma_2^2}y\Bigr)\,.
    \intertext{Nun machen wir eine quadratische Ergänzung zu}
  &=-\frac{\sigma_1^2+\sigma_2^2}{2\sigma_1^2\sigma_2^2}\Bigl(y^2-2\frac{\sigma_1^2x}{\sigma_1^2+\sigma_2^2}y+\Bigl[\frac{\sigma_1^2x}{\sigma_1^2+\sigma_2^2}\Bigr]^2-\Bigl[\frac{\sigma_1^2x}{\sigma_1^2+\sigma_2^2}\Bigr]^2\Bigr)\,.
    \intertext{Jetzt können wir die zweite binomische Formel nutzen und kriegen}
  &=-\frac{\sigma_1^2+\sigma_2^2}{2\sigma_1^2\sigma_2^2}\Bigl(\Bigl[y-\frac{\sigma_1^2x}{\sigma_1^2+\sigma_2^2}\Bigr]^2-\Bigl[\frac{\sigma_1^2x}{\sigma_1^2+\sigma_2^2}\Bigr]^2\Bigr)\,.
    \intertext{Wir können den letzten Term auch noch aus der runden Klammer ziehen, sodass}
  &=-\frac{\sigma_1^2+\sigma_2^2}{2\sigma_1^2\sigma_2^2}\Bigl(y-\frac{\sigma_1^2x}{\sigma_1^2+\sigma_2^2}\Bigr)^2+\frac{\sigma_1^2x^2}{2\sigma_2^2(\sigma_1^2+\sigma_2^2)}\,.
\end{align*}
Eingesetzt in die Rechnung für $f_3(x)$ ergibt sich
\begin{align*}
  f_3(x)
  &=\Bigl(2\pi\sqrt{\sigma_1^2\sigma_2^2}\Bigr)^{-1}\exp\Bigl(-\frac{x^2}{2\sigma_2^2}\Bigl[1-\frac{\sigma_1^2}{\sigma_1^2+\sigma_2^2}\Bigr]\Bigr)\\
  &\quad\times\int\exp\Bigl(-\frac{\sigma_1^2+\sigma_2^2}{2\sigma_1^2\sigma_2^2}\Bigl[y-\frac{\sigma_1^2x}{\sigma_1^2+\sigma_2^2}\Bigr]^2\Bigr)\lambda(\mathrm{d}y)
    \intertext{In der ersten Exponentialfunktion können wir die Terme auf einen gemeinsamen Nenner bringen und etwas zusammenfassen.
    Das Integral was wir haben ist translationsinvariant, sodass wir den zweiten Term weglassen können und insgesamt erhalten}
  &=\Bigl(2\pi\sqrt{\sigma_1^2\sigma_2^2}\Bigr)^{-1}\exp\Bigl(-\frac{x^2}{2(\sigma_1^2+\sigma_2^2)}\Bigr)\int\exp\Bigl(-\frac{\sigma_1^2+\sigma_2^2}{2\sigma_1^2\sigma_2^2}y^2\Bigr)\lambda(\mathrm{d}y)\,.
    \intertext{Das Integral lässt sich zum Beispiel mittels Übergang zu Polarkoordinaten bestimmen.
    Im Allgemeinen gilt $\int_{-\infty}^\infty\exp(-\alpha x^2)=\sqrt{\frac{\pi}{\alpha}}$.
    Wir erhalten}
  &=\Bigl(2\pi\sqrt{\sigma_1^2\sigma_2^2}\Bigr)^{-1}\sqrt{\pi}\sqrt{\frac{2\sigma_1^2\sigma_2^2}{\sigma_1^2+\sigma_2^2}}\exp\Bigl(-\frac{x^2}{2(\sigma_1^2+\sigma_2^2)}\Bigr)\,.
    \intertext{Zusammenfassen der ersten beiden Terme ergibt}
  &=\Bigl(\sqrt{2\pi\bigl(\sigma_1^2+\sigma_2^2\bigr)}\Bigr)^{-1}\exp\Bigl(-\frac{x^2}{2(\sigma_1^2+\sigma_2^2)}\Bigr)\,.
\end{align*}
Das ist die Dichte einer Zufallsvariable mit Verteilung ${\cal N}(0,\sigma_1^2+\sigma_2^2)$.
\newpage


\bibliography{../../../books/wt}
\end{document}

%%% Local Variables:
%%% mode: latex
%%% ispell-local-dictionary: "german"
%%% TeX-master: t
%%% End:
