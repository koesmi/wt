\documentclass{article}
\usepackage[a4paper,margin=1.875in,top=1.875in,bottom=1.875in]{geometry}

\usepackage{amsmath,mathtools,bbm,amssymb}
\usepackage[german]{babel}

\usepackage{setspace}
\doublespacing

%\usepackage{fancyhdr}
%\renewcommand{\headrulewidth}{0pt} 
%\pagestyle{fancy}
%\lhead{Blatt 10 Nicolas und Evgenij}\rhead{Seite \thepage}
%\fancyfoot{}

\usepackage{tikz}
\usetikzlibrary{decorations.pathreplacing,arrows}

\usepackage[numbers]{natbib}
\bibliographystyle{alphadin}
\usepackage{url}
\usepackage{hyperref}

\begin{document}

\paragraph{Verschiedene Konvergenzarten}
Wir betrachten den Wahrscheinlichkeitsraum $([0,1],{\cal B}([0,1]),P)$, wobei $P$ das Lebesgue-Maß $\lambda$ eingeschränkt auf $[0,1]$ sei.
Wir betrachten die Folge von Zufallsvariablen
\[
  X_1\equiv0,~X_n=\sqrt{n}\mathbbm{1}_{(\frac{1}{n},\frac{2}{n})}\,.
\]
Untersuchen Sie diese auf 1. stochastische Konvergenz, 2. $P$-fast-sichere Konvergenz, 3. $L^2$-Konvergenz und 4. gleichgradige  Integrierbarkeit.


1, 4. Wir prüfen $L^1$-Konvergenz.
Es gilt $E[|X_n|]=\frac{\sqrt{n}}{n}\to0$.
\emph{nochmal nachrechnen}. Benutze Konvergenzsatz von Vitali -- Stochastisch konvergent gegen 0 und gleichgradig integrierbar.
2. Wähle $x\in(0,1)$.
Dann existiert ein $N\in\mathbb{N}$ sodass $x\geq\frac{2}{n}$ für alle $n\in N$.
Also $X_n(x)=0$ für  alle $n>N$.
Weiter gilt $X_n(0)=0$ für alle $n\in\mathbb{N}$.
Also Konvergenz fast sicher gegen 0.
\emph{warum?}
3. $L^2$-Konvergenz folgt stochastische Konvergenz, also Grenzwert 0 wenn konvergent.
$E[|X_n|^2]=\frac{\sqrt{n}^2}{n}=1$.
\emph{nochmal nachrechnen}.
Also nicht konvergent.


\bibliography{../../../books/wt}
\end{document}

%%% Local Variables:
%%% mode: latex
%%% ispell-local-dictionary: "german"
%%% TeX-master: t
%%% End:
