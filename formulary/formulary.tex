\documentclass{article}
\usepackage[a4paper,margin=1.875in,top=1.875in,bottom=1.875in]{geometry}

\usepackage{amsmath,mathtools,bbm,amssymb}
\usepackage[german]{babel}

\usepackage{setspace}
\doublespacing

%\usepackage{fancyhdr}
%\renewcommand{\headrulewidth}{0pt} 
%\pagestyle{fancy}
%\lhead{Blatt 10 Nicolas und Evgenij}\rhead{Seite \thepage}
%\fancyfoot{}

\usepackage{tikz}
\usetikzlibrary{decorations.pathreplacing,arrows}

\usepackage[numbers]{natbib}
\bibliographystyle{alphadin}
\usepackage{url}
\usepackage{hyperref}

\begin{document}

\paragraph{Verschiedene Konvergenzarten}
Wir betrachten den Wahrscheinlichkeitsraum $([0,1],{\cal B}([0,1]),P)$, wobei $P$ das Lebesgue-Maß $\lambda$ eingeschränkt auf $[0,1]$ sei.
Wir betrachten die Folge von Zufallsvariablen $X_1\equiv0,~X_n=\sqrt{n}\mathbbm{1}_{(\frac{1}{n},\frac{2}{n})}$.
Untersuchen Sie diese auf 1. stochastische Konvergenz, 2. $P$-fast-sichere Konvergenz, 3. $L^2$-Konvergenz und 4. gleichgradige  Integrierbarkeit.

1, 4. Wir prüfen $L^1$-Konvergenz.
Es gilt $E[|X_n|]=\frac{\sqrt{n}}{n}\to0$.
Benutze Konvergenzsatz von Vitali -- Stochastisch konvergent gegen 0 und gleichgradig integrierbar.
2. Zu zeigen ist, dass für alle $\omega\in[0,1]$ gilt, dass $\forall\varepsilon>0\exists N\in\mathbb{N}\forall n\geq N X_n(\omega)<\varepsilon$, bis auf $P$-Nullmengen.
Zeige Aussage für $\omega\in(0,1)$.
Das reicht, denn $\{0\}$ und $\{1\}$ sind $P$-Nullmengen.
Wähle $N$ so, dass $\frac{2}{n}<\varepsilon$, also $N>\frac{2}{\varepsilon}$, dann ist $\mathbbm{1}_{(\frac{1}{n},\frac{2}{n})}=0$ und somit $X_n(\omega)=0$ für alle $n\geq N$.
Also Konvergenz fast sicher gegen 0.
3. $L^2$-Konvergenz folgt stochastische Konvergenz, also Grenzwert 0 wenn konvergent.
$E[|X_n|^2]=\frac{\sqrt{n}^2}{n}=1$.
Also nicht konvergent.

Wir betrachten den Wahrscheinlichkeitsraum $([0,1]{\cal B}([0,1]),P)$.
Das Maß $P$ sei absolut stetig bezüglich des Lebesgue-Maßes $\lambda|_{[0,1]}$ mit Dichte $f(\omega)=\frac{1}{2}\omega^{-1/2}$. Es gelte $X_n(\omega)=\omega^{1/n}$.
Zeigen Sie oder widerlegen Sie  1. $P$-Konvergenz, 2. f.s. Konvergenz, 3. $L^1$-Konvergenz, 4. gleichgradige Integrierbarkeit.
2. Für alle $\omega\in(0,1]$ gilt $\omega^{1/n}\to1$.
$\{0\}$ ist Nullmenge, also f.s. Konvergenz.
1. folgt aus 2.
3. Es gilt $\omega^{1/n}\leq1$, also $E[|X_n-1|]=E[|1-X_n|]=\int_0^1(1-\omega^{1/n})\frac{1}{2}\omega^{-1/2}=1+\int_0^1\frac{1}{2}\omega^{1/n-1/2}=1-\left.\frac{n}{n+2}\omega^{\frac{n+2}{2n}}\right|_0^1=\frac{2}{n+2}\to0$, also $L^1$-Konvergenz.
4. folgt aus 2. mit Vitali.

Für jede Zahl $n\in\mathbb{N}$ gibt es eine eindeutige Darstellung $n=2^{k_n}+m_n$ mit $0\leq m_n<2^{k_n}$.
Es sei $P$ die Gleichverteilung auf $([0,1],{\cal B}([0,1]))$ -- das heißt, $P$ hat Lebesgue-Dichte $\mathbbm{1}_{[0,1]}$ -- und außerdem $X_n\colon [0,1]\to\mathbb{R}, \omega\mapsto k_n$ für $\frac{m_n}{2^{k_n}}\leq\omega\leq\frac{m_n+1}{2^{k_n}}$ und $\omega\mapsto 0$ sonst.
Untersuchen Sie die Folge der $X_n$ bezüglich $P$ auf schwache, stochastische, fast sichere und $L^p$-Konvergenz für $p\geq1$ sowie auf gleichgradige Integrierbarkeit.

Es gilt $X_n=k_n\mathbbm{1}_{\left[\frac{m_n}{2^{k_n}},\frac{m_n+1}{2^{k_n}}\right]}$, also $X_1=1$, $X_2=1\mathbbm{1}_{\left[0,\frac{1}{2}\right]}$, $X_3=1\mathbbm{1}_{\left[\frac{1}{2},1\right]}$, $X_4=2\mathbbm{1}_{\left[0,\frac{1}{4}\right]}$, $X_5=2\mathbbm{1}_{\left[\frac{1}{4},\frac{1}{2}\right]}$, $\dots$.
$L^p$-Konvergenz: finde zunächst den Kandidaten für den Grenzwert.
Da die Folge $(X_n)$ also bezüglich der $L^1$-Norm konvergiert, ist sie nach Theorem 22 auch gleichgradig integrierbar. Folge wird immer kleiner, also Kandidat 0.
Es gilt $E[|X_n|^p]=\int_0^1 X_n(\omega)^p=\frac{k_n^p}{2^{k_n}}\to0$ mit L'Hospital.
Damit $L^p$, stochastische, schwache Konvergenz und gleichgradige Integrierbarkeit.
Für f.s. müsste Grenzwert auch 0 sein.
Möchte Divergenz zeigen, also $P(A)>0$ für $A=\{\omega\in\Omega\mid\forall c\in\mathbb{R}\exists N\in\mathbb{N}\forall n>N X_n>c\}$.
Sei $\omega\in\Omega$ und $c\in\mathbb{R}$ gegeben, wähle $N$ so, dass $k_N>c$ und $\frac{m_n}{2^{k_n}}\leq\omega\leq\frac{m_n+1}{2^{k_n}}$.
Dann gilt $X_n(\omega)>c$, also $\omega\in A$.
Da $\omega$ beliebig war, gilt $A=\Omega$ und $P(X_n$ konvergiert nicht $)=1$.
Oder auch es gibt TF $X_{n_l}$ so dass $\omega\in[\frac{m_{n_l}}{2^{k_n_l}}..]$, d.h. $\limsup\geq1$ und somit $\lim\neq0$.

Sei $P(X_n=\sqrt{n})=\frac{1}{n}=1-P(X_n=0)$.
Untersuchen Sie die Folge auf 1. stochastische, 2. $P$-fast-sichere und 3. $L^p$-Konvergenz und auf 4. gleichgradige Integrierbarkeit.
1. Wenn $X_n>\varepsilon$, dann ist $P(X_n=\sqrt{n})=\frac{1}{n}\to0$, also ja.
2. Nein. $(\{X_n\geq1\})_n$ sind unabhängig. Kann Borel-Cantelli anwenden. $\sum P(X_n\geq 1)=\sum\frac{1}{n}=\infty$, sodass $1=P(\limsup\{X_n\geq1\})=P(X_n\geq1$ für unendlich viele $n)$.
Das heißt, existiert Teilfolge für $P$ fast alle $\omega$ $X_{n_k}(\omega)=\sqrt{n_k}$.
$\limsup X_n(\omega)\geq\lim\sqrt{n_k}=\infty$.
Also $1=P(\limsup X_n\geq1)$, sodass $P(X_n\to0)=0$.
3. $E[|X|^p]=(\sqrt{n})^p\frac{1}{n}=n^{p/2-1}$, also $L^p$-Konvergenz wenn $p<2$.
4. GGIB, da $L^1$-Konvergenz nach Vitali.

\paragraph{Einfache Aufgaben}
Sei $(\Omega,{\cal F},P)=([0,1],{\cal B}([0,1]),\lambda|_{[0,1]})$, wobei $\lambda|_{[0,1]}$ das Lebesgue-Maß auf $[0,1]$ bezeichnet.
Dann haben $X_1$ und $X_2$ mit $X_1(\omega)=\omega$ und $X_2(\omega)=1-\omega$ die gleiche Verteilung.

$[a,b]$ erzeugen ${\cal B}([0,1])$, also reicht zu prüfen $\lambda\circ X_1^{-1}([a,b])=b-a$, $\lambda\circ X_2^{-1}([a,b])=\lambda([1-b,1-a])=b-a$.

Sei $(\Omega,{\cal F})=([0,1],{\cal B}([0,1]))$.
Es gibt ein Wahrscheinlichkeitsmaß $P$, sodass $X_1(\omega)=\omega$ und $X_2(\omega)=1-\omega$ nicht die gleiche Verteilung haben.
$\delta_0$, denn $\delta_0\circ X_1^{-1}(\{0\})=1$, aber $\delta_0\circ X_2^{-1}(\{0\})=0$.

Sei $X\sim{\cal N}(2,2)$. Dann gilt, dass $P[|X-2|\geq2]\leq\frac{1}{2}$.
Ja, Tschebyscheff $P(|X-E[x]|>\varepsilon)\leq\frac{\operatorname{Var}(x)}{\varepsilon^2}$.
Jede reellwertige Zufallsvariable hat Dichte bezüglich Lebesgue-Maß (also $X_\star P=f\cdot\lambda$)
Radon-Nikodym: $\nu$ Dichte bezüglich $\mu$ genau dann wenn $\nu\ll \mu$.
Sei $X=0$.
Dann ist $X$ stetig und somit messbar mit $P(X=0)=1$, also $X_\star P=\delta_0$.
Da $0=\lambda(\{0\})\neq\delta_0(\{0\})=1$, ist $\delta_0$ nicht absolut stetig bezüglich $\lambda$.
Nach dem Satz von Radon-Nikodym besitzt dann $\delta_0$ keine Dichte bezüglich des Lebesgue-Maßes.

Alle Abbildungen $f\colon (\Omega,{\cal P}(\Omega))\to(\mathbb{R},{\cal B}(\mathbb{R}))$ sind messbar.
Ja.
Sei $A\in{\cal B}(\mathbb{R})$, dann ist $f^{-1}(A)\in{\cal P}(\Omega)$, denn in ${\cal P}(\Omega)$ sind alle Mengen, die nach $A$ abbilden könnten, drin.

Für $X\sim{\cal N}(0,1)$ und $Y\sim{\cal N}(0,2)$ gilt $E[XY]\leq\sqrt{2}$?
Ja, denn nach der Cauchy--Schwarz-Ungleichung, also der Hölder-Ungleichung mit $r=1$ und $p=q=2$ gilt
$E[XY]
\leq\|X\|_2\|Y\|_2
=\sqrt{E[X^2]}\sqrt{E[Y^2]}
=\sqrt{E[(X-E[X])^2]}\sqrt{E[(Y-E[Y])^2]}
=\sqrt{\operatorname{Var}(X)}\sqrt{\operatorname{Var}(Y)}
=\sqrt{2}$.

Eine $\mathbb{N}$-wertige Zufallsvariable ist zu sich selbst unabhängig, wenn sie fast sicher konstant ist.
Wenn $X$ zu sich selbst unabhängig ist, gilt $P(X=k)=P(\{X=k\}\cap\{X=k\})=P(X=k)^2$, also $P(X=k)\in\{0,1\}$.
Das heißt, nur für ein $k_0\in\mathbb{N}$ ist $P(X=k_0)=1$, also ist $X$ konstant.
Umgekehrt sei $P(X=k_0)=1$, dann ist $P(\{X\in A\}\cap\{X\in B\})=\mathbbm{1}_{A\cap B}(k_0)=\mathbbm{1}_{A}(k_0)\mathbbm{1}_{B}(k_0)=P(X\in A)P(X\in B)$.

$X=\mathbbm{1}_A$ und $Y=\mathbbm{1}_B$.
$E[XY]=E[X]E[Y]\iff A\perp B$.
Ja, $E[XY]=P[A\cap B]=P(A)P(B)=E[X]E[Y]$ oder $P(A\cap B)=E[XY]=E[X]E[Y]=P(A)P(B)$.

$E[X^4]\geq E[X]^4$ gilt nach Jensen-Ungleichung, da $x^4$ konvex.

Auf $(\omega,\{\Omega,\emptyset\})$ gibt es keine Borel-messbare Abbildung?
Doch.
Sei $f=0$, dann für $A\in{\cal B}(\mathbb{R})$ gilt $f^{-1}(A)=\emptyset$, falls $0\notin A$, beziehungsweise $f^{-1}(A)=\Omega,$ falls $0\in A$.

Sei $X$ exponentialverteilt mit $\lambda=6$ und $Y$ mit $\lambda=1/3$. Dann ist nach Cauchy-Schwarz $E[XY]\leq E[X^2]^{1/2}E[Y^2]^{1/2}=\frac{2}{\lambda_X^2}\frac{2}{\lambda_Y^2}=1$.

$X\in L^p\Rightarrow X\in L^q$ für $q\leq p$, da $\|X\|_q=\|1\cdot X\|_q\leq\|X\|_p\|1\|_r$ mit $r$ so, dass $\frac{1}{q}=\frac{1}{p}+\frac{1}{r}$.

\paragraph{Maßtheorie}
$\sigma(f^{-1}({\cal C}))=f^{-1}(\sigma({\cal C}))$
Zeige zunächst ``$\subseteq$''.
Es ist ${\cal C}\subset\sigma({\cal C})$, also auch $f^{-1}({\cal C})\subset f^{-1}(\sigma({\cal C}))$.
$f^{-1}$ von einer $\sigma$-Algebra ist wieder $\sigma$-Algebra und da ist $\sigma(f^{-1}({\cal C}))$ als kleinste $\sigma$-Algebra drin.
Nun ``$supseteq$''.
Betrachte ${\cal F}=\{A\in\sigma({\cal C})\mid f^{-1}(A)\in\sigma(f^{-1}({\cal C}))\}$.
Zeige ,${\cal F}$ ist $\sigma$-Algebra.
$f^{-1}(\emptyset)=\emptyset\in\sigma(f^{-1}{\cal C})$, $f^{-1}(\Omega)=\Omega'\in\sigma(f^{-1}({\cal C}))$.
Entsprechend $f^{-1}(A^c)=f^{-1}(A)^c\in\sigma(f^{-1}({\cal C}))$ und so weiter.
Da $f^{-1}({\cal C})\subset\sigma(f^{-1}({\cal C}))$ ist ${\cal C}\subset{\cal F}$, also auch $\sigma({\cal C})\subset{\cal F}$.
Damit ist $f^{-1}(\sigma({\cal C}))\subset\sigma(f^{-1}({\cal C}))$.
\bibliography{../../../books/wt}
\end{document}

%%% Local Variables:
%%% mode: latex
%%% ispell-local-dictionary: "german"
%%% TeX-master: t
%%% End:
