\documentclass{article}
\usepackage[a4paper,margin=1.875in,top=1.7in,bottom=1.7in]{geometry}

\usepackage{amsmath,mathtools,bbm,amssymb}
\usepackage{mathrsfs}
\usepackage{german}

\usepackage{setspace}
\doublespacing

\usepackage{fancyhdr}
\renewcommand{\headrulewidth}{0pt} 
\pagestyle{fancy}
\lhead{Blatt 11 Nikolaus, Lukas, Evgenij}\rhead{Seite \thepage}
\fancyfoot{}

\usepackage{tikz}
\usetikzlibrary{decorations.pathreplacing,arrows}

\usepackage[numbers]{natbib}
\bibliographystyle{alphadin}
\usepackage{url}
\usepackage{hyperref}

\usepackage{eurosym}

\begin{document}
\paragraph{Aufgabe 1 \textnormal{(4 Points)}.}
\paragraph{\normalfont{{\itshape Definition} 1 ($T$-Forward-Measure).}}
Let $(B_t)_{t\leq T}$, $B_t=e^{\int_0^tr_sds}$ be the bank account/numeraire in a financial market.
If $\mathbb{Q}$ is a risk-neutral measure, then the forward measure $\mathbb{Q}^T$ on $\mathscr{F}_T$ is defined via the Radon Nikodym density process $Z$ with respect to $\mathbb{Q}$, given by
\[
  Z_t=\frac{P_t(T)}{P_0(T)B_t}\,.
\]
\paragraph{Aufgabe 1 \textnormal{(4 Points)}.}
Let $F_t(T,S)$ be the simple forward rate for $[T,S]$ prevailing at $t$ which is given by
\[
F_t(T,S)=\frac{1}{S-T}\left(\frac{P_t(T)}{P_t(S)}-1\right),\quad t\in[0,T]\,.
\]
Show that $(F_t(T,S))_{t\in[0,T]}$ is a martingale with respect to some forward measure $Q^U$; that is
\[
  F_t(T,S)=E_{Q^U}[F_T(T,S)|\mathscr{F}_t]\quad\text{für alle }t\in[0,T]\,.
\]
What is $U$?

\paragraph{\normalfont{{\scshape Hint.}}} Use the identity
\[
  P_t(T)=B_tE_Q\left[\frac{1}{B_T}\middle|\mathscr{F}_t\right],\quad t\in[0,T]\,.
\]

\paragraph{\normalfont{{\itshape Solution.}}} this is Exercise 1 from sheet 13 form Probability Theory 2.
We claim that $U=S$.
According to Definition 1, the forward measure $\mathbb{Q}^S$ with respect to $\mathbb{Q}$ is given by
\begin{equation}
Z=\frac{1}{P_0(S)}\frac{P(S)}{B}\,.\label{eq:z}
\end{equation}
We use the definition of $F_T(T,S)$ to get
\begin{align*}
  E_{Q^S}[F_T(T,S)|\mathscr{F}_t]
  &=\frac{1}{S-T}\left(E_{Q^S}\left[\frac{1}{P_T(S)}\middle|\mathscr{F}_t\right]-1\right)\,.
    \intertext{By measure change with $Z$ from equation (\ref{eq:z}) we get}
  &=\frac{1}{S-T}\left(\frac{1}{Z_t}E_{Q}\left[\frac{1}{P_T(S)}\middle|\mathscr{F}_t\right]-1\right)\,.
    \intertext{Employing the definition of $Z$ in equation (\ref{eq:z}) we get}
  &=\frac{1}{S-T}\left(\frac{B_t}{P_t(S)}E_{Q}\left[\frac{1}{B_T}\middle|\mathscr{F}_t\right]-1\right)\,.
    \intertext{Finally, we use the hint to arrive at}
  &=\frac{1}{S-T}\left(\frac{P_t(T)}{P_t(S)-1}\right)=F_t(T,S).
\end{align*}
\bibliography{../../../books/wt}
\end{document}

%%% Local Variables:
%%% mode: latex
%%% TeX-master: t
%%% End:
