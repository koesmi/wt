\documentclass{article}
\usepackage[a4paper,margin=1.875in,top=1.7in,bottom=1.7in]{geometry}

\usepackage{amsmath,mathtools,bbm,amssymb}
\usepackage{mathrsfs}
\usepackage{german}

\usepackage{setspace}
\doublespacing

\usepackage{fancyhdr}
\renewcommand{\headrulewidth}{0pt} 
\pagestyle{fancy}
\lhead{Blatt 9 Nikolaus, Lukas, Evgenij}\rhead{Seite \thepage}
\fancyfoot{}

\usepackage{tikz}
\usetikzlibrary{decorations.pathreplacing,arrows}

\usepackage[numbers]{natbib}
\bibliographystyle{alphadin}
\usepackage{url}
\usepackage{hyperref}

\usepackage{eurosym}

\begin{document}
\paragraph{Aufgabe 1 \textnormal{4 Punkte)}.}
Sei $W$ eine Standard-Brown'sche Bewegung auf $(\Omega,\mathscr{F},\mathbf{F},P)$ und $H$ ein beschränkter, previsibler Prozess.
Sei $X$ ein stochastischer Prozess definiert durch
\[
  X_t=\int_0^tH_sds+W_t\,.
\]
Definiere das Wahrscheinlichkeitsmaß $Q$ durch
\[
  \frac{dQ}{dP}=e^{-\int_0^TH_sdW_s-\frac{1}{2}\int_0^TH_s^2ds}\,,
\]
mit $T>0$.
Zeigen Sie, dass $X$ unter $Q$ eine Standard-Brown'sche Bewegung bis zum Zeitpunkt $T$ ist.

\paragraph{\textnormal{\emph{Lösung.}}}
Vergleiche \cite[Theorem 42]{Protter2005}.
Schreibe $Z_t=E[\frac{dQ}{dP}|\mathscr{F}_t]$.
Nach Theorem 86 löst $Z$
\[
Z_t=1-\int_0^tZ_{s-}H_sdW_s\,.
\]
Nach dem Satz von Girsanov ist
\begin{equation}
  N_t=W_t-\int_0^t\frac{1}{Z_s}d[Z,W]_s\label{eq:nt}
\end{equation}

ein $Q$-lokales Martingal.
Allerdings gilt mit der Definition von $Z$
\begin{align*}
  [Z,W]_t
  &=[-Z_-H\cdot W,W]_t\,.
    \intertext{Mit der Definition der quadratischen Kovariation und den  Rechenregeln für das stochastische Integral erhalten wir, \emph{wobei noch nicht ganz klar ist, wie},}
  &=\int_0^t-Z_sH_sd[W,W]_s=-\int_0^tZ_sH_sds\,,
\end{align*}
denn für eine Brown'sche Bewegung gilt $[W,W]_t=t$.
Deshalb gilt mit Einsetzen in Gleichung (\ref{eq:nt})
\begin{align*}
  N_t
  &=W_t-\int_0^t-\frac{1}{Z_s}Z_sH_sds=W_t+\int_0^tH_sds=X_t\,,
\end{align*}
wie in der Aufgabenstellung gegeben.
Damit ist auch $X$ ein $Q$-lokales Martingal.
Da $\left(\int_0^tH_sds\right)_{t\geq0}$ ein stetiger Prozess endlicher Variation ist, gilt $[X,X]_t=[W,W]_t=t$.
Mit Theorem 2.1 aus Moritz' Skript kriegen wir schließlich, dass $X$ eine Standard Brown'sche Bewegung ist.
\bibliography{../../../books/wt}
\end{document}

%%% Local Variables:
%%% mode: latex
%%% ispell-local-dictionary: "german"
%%% TeX-master: t
%%% End:
