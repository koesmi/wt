\documentclass{article}
\usepackage[a4paper,margin=1.875in,top=1.7in,bottom=1.7in]{geometry}

\usepackage{amsmath,mathtools,bbm,amssymb}
\usepackage{mathrsfs}
\usepackage{german}

\usepackage{setspace}
\doublespacing

\usepackage{fancyhdr}
\renewcommand{\headrulewidth}{0pt} 
\pagestyle{fancy}
\lhead{Blatt 10 Nikolaus, Lukas, Evgenij}\rhead{Seite \thepage}
\fancyfoot{}

\usepackage{tikz}
\usetikzlibrary{decorations.pathreplacing,arrows}

\usepackage[numbers]{natbib}
\bibliographystyle{alphadin}
\usepackage{url}
\usepackage{hyperref}

\usepackage{eurosym}

\begin{document}
\paragraph{Aufgabe 1 \textnormal{(4 Punkte)}.}
Beweisen Sie Theorem 3.3.

\emph{Hinweis: Verwenden Sie Lemma 3.5.}

Das Theorem 3.3. sagt aus, die folgenden Aussagen sind äquivalent
\begin{enumerate}
\item [(i)] es gilt $\mathrm{NA}({\cal P})$;
\item [(ii)] für alle $P\in{\cal P}$ gibt es ein $Q\in{\cal Q}$, sodass $P\ll Q$.
\end{enumerate}
Das wird in \cite[Seite 15]{Bouchard_2015} bewiesen.
Es gelte zunächst $\operatorname{NA}({\cal P})$.
Sei $P\in{\cal P}$ gegeben. Mit der zweiten Aussage aus dem Fundamentallemma gilt $0\in\operatorname{ri}\{E_R[\Delta S]\colon R\in\mathfrak{P}(\Omega), P\ll R\lll {\cal P}, E_R[|\Delta S|]<\infty\}\subseteq \mathbb{R}^d$.
Es gibt also ein $Q\in\mathfrak{P}(\Omega)$ mit  $Q\lll {\cal P}$, sodass $E_Q[\Delta S]=0$, also mit der Definition von $\mathcal{Q}$, ein $Q\in{\cal Q}$, aber auch $P\ll Q$.
\emph{Hier ist noch nicht so ganz klar, ob es sein kann, dass $0\in\operatorname{ri}\{E_R[\Delta S]\}$, aber nicht $0\in\{E_R[\Delta S]\}$.}

Es gebe nun anders herum für alle $P\in {\cal P}$ ein $Q\in{\cal Q}$, sodass $P\ll Q$.
Sei $H\in\mathbb{R}^d$ so, dass $H\Delta S\geq 0$ ${\cal P}$-quasi sicher.
Angenommen, es gibt nun ein $P\in{\cal P}$, sodass $P\{H\Delta S>0\}>0$.
Dann gibt es nach Annahme auch ein Martingalmaß $Q$, sodass $P\ll Q\lll {\cal P}$.
Damit gilt aber auch $Q\{H\Delta S>0\}>0$.
Dies steht aber im Widerspruch zu $E_Q[H\Delta S]=0$.
\bibliography{../../../books/wt}
\end{document}

%%% Local Variables:
%%% mode: latex
%%% ispell-local-dictionary: "german"
%%% TeX-master: t
%%% End:
