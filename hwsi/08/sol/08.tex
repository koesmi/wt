\documentclass{article}
\usepackage[a4paper,margin=1.875in,top=1.7in,bottom=1.7in]{geometry}

\usepackage{amsmath,mathtools,bbm,amssymb}
\usepackage{mathrsfs}
\usepackage{german}

\usepackage{setspace}
\doublespacing

\usepackage{fancyhdr}
\renewcommand{\headrulewidth}{0pt} 
\pagestyle{fancy}
\lhead{Blatt 7 Nikolaus, Lukas, Evgenij}\rhead{Seite \thepage}
\fancyfoot{}

\usepackage{tikz}
\usetikzlibrary{decorations.pathreplacing,arrows}

\usepackage[numbers]{natbib}
\bibliographystyle{alphadin}
\usepackage{url}
\usepackage{hyperref}

\usepackage{eurosym}

\begin{document}
\paragraph{Aufgabe 1 \textnormal{Komlós Lemma für $L^0$; 6 Punkte)}.}
Sei $(f_n)_{n\in\mathbb{N}}$ eine Folge nicht-negativer Zufallsvariablen. Zeigen Sie:
\begin{enumerate}
\item [i)] Es existieren $g\in\langle f_n,f_{n+1},\dots\rangle_\text{conv}$, sodass $(g_n)_{n\in\mathbb{N}}$ fast sicher gegen eine Zufallsvariable mit Werten in $[0,\infty]$ konvergiert.
\item [ii)] Es gilt $P[g<\infty]=1$, falls $\langle f_n,f_{n+1},\dots\rangle_\text{conv}$ in $L^0$ beschränkt ist.

  \emph{Hinweis: Eine Familie $F\subset L^0$ heißt beschränkt, wenn für jedes $\varepsilon>0$ eine Konstante $C>0$ existiert, sodass $P[|X|\geq C]<\varepsilon$ für alle $X\in F$.}
  
\item [iii)] Es gilt $P[g>0]>0$, falls ein $\alpha>0$ existiert, sodass $P[f_n\geq \alpha]\geq a>0$.
\end{enumerate}
\emph{Hinweis:}
\begin{enumerate}
\item [i)] Das Resultat folgt, falls es eine Folge $g_n\in\langle f_n,f_{n+1},\dots\rangle_\text{conv}$ gibt, sodass $e^{-g_n}$ in $L^1$ konvergiert.
  Definieren Sie
  \[
    J_n:= \inf\{E[g^{-g}]|g\in\langle f_n,f_{n+1},\dots\rangle_\text{conv}\}.
  \]
  Setzen Sie
  \begin{align*}
    A_\varepsilon
    &=\{(x,y)\in\mathbb{R}_+^2\big||x-y|\leq\varepsilon\}\,,\\
    B_\varepsilon
    &=\{(x,y)\in\mathbb{R}_+^2\big| x\wedge y\geq 1/\varepsilon\}\,,\\
    C_\varepsilon
    &=\mathbb{R}_+^2\setminus(A_\varepsilon \cup B_\varepsilon)\,.
  \end{align*}
Für $(x,y)\in\mathbb{R}_+^2$ existiert für jedes $\varepsilon$ ein $\delta$, sodass
\begin{equation}
  e^{-(x+y)/2}\leq\left(\frac{e^{-x}+e^{-y}}{2}\right)-\delta\mathbf{1}_{C_\varepsilon}(x,y)\,.\label{eq:ceineq}
\end{equation}
\item [iii)] Zeigen Sie, dass $g_n=\sum_{n\leq k\leq N}\lambda_k f_k\in\langle f_n,f_{n+1},\dots\rangle_\text{conv}$ die Ungleichung
  \[
    E[e^{-g_n}]\leq(1-\alpha)+e^{-\alpha}
  \]
  erfüllt.
\end{enumerate}
\paragraph{\textnormal{\emph{Lösung:}}} Nach \cite{schweizer}.
$J_n$ wächst bis zu einem $J\leq 1$.
Sei $(g'_n)_{n\in\mathbb{N}}$ mit $g'_n\in\langle f_n,f_{n+1},\dots\rangle_\text{conv}$ und $E[e^{-g_n'}]\leq J_n+\frac{1}{n}$.
Da $z\mapsto e^{-z}$ konvex ist, haben wir immer
\[
  e^{-(x+y)/2}\leq\frac{1}{2}(e^{-x}+e^{-y})\,.
\]
Für $(x,y)\in C_\varepsilon$ gilt für ein $\delta=\delta(\varepsilon)>0$, dass
\[
  e^{-(x+y)/2}-\frac{1}{2}(e^{-x}+e^{-y})\leq-\delta\,,
\]
\emph{was noch überprüft werden sollte}.
Hierdurch erhalten wir Gleichung \ref{eq:ceineq}.
Wenn wir nun $x:=g'_m$ und $y:= g'_n$ setzen, ergibt sich für $n\neq m$ mit der Definition von $J_n$, dass
\begin{align*}
  J_m
  &\leq E\left[ e^{(-g'_m+g'_n)/2}\right]\,.
    \intertext{Mit Gleichung \ref{eq:ceineq} erhalten wir}
  &\leq\frac{1}{2}\left(E[e^{-g'_m}]+E[e^{-g'_n}]\right)-\delta P[(g'_m,g'_n)\in C_\varepsilon]\,.
    \intertext{Mit der Definition von $g'$ erhalten wir}
  &\leq \frac{1}{2}\bigl(J_m+\frac{1}{m}+J_n+\frac{1}{n}\bigr)-\delta P[(g'_m,g'_n)\in C_\varepsilon]\,,
\end{align*}
sodass $\lim_{n,m\to\infty}P[(g'_m,g'_n)\in C_\varepsilon]=0$, \emph{was auch nicht so ganz klar ist.}

Für $(x,y)\in A_\varepsilon$ und $(x,y)\in B_\varepsilon$ erhalten wir die Abschätzung
\[
  |e^{-x}-e^{-y}|\leq\varepsilon+2e^{-1/\varepsilon}+2\mathbf{1}_{C_\varepsilon}(x,y).
\]
Eine analoge Rechnung wie oben ergibt
\[\left|E[e^{-g'_m}-e^{-g'_n}]\right|\leq \varepsilon+2e^{-1/\varepsilon}+2P[(g'_m,g'_n)\in C_\varepsilon]\,,\]
\emph{was ebenfalls überprüft werden sollte}.
Damit ist $(e^{-g'_n})_{n\in\mathbb{N}}$ eine Cauchyfolge in $L^1(P)$ und somit konvergiert sie in $L^1(P)$.
Deswegen hat die seine Teilfolge $(e^{-g_n})_{n\in\mathbb{N}}$, die $P$-fast sicher konvergiert.
Hierdurch konvergiert $(g_n)_{n\in\mathbb{N}}$ schon selbst und wie für $g'_n$ gilt auch $g_n\in\langle f_n,f_{n+1},\dots\rangle_\text{conv}$.
\bibliography{../../../books/wt}
\end{document}

%%% Local Variables:
%%% mode: latex
%%% ispell-local-dictionary: "german"
%%% TeX-master: t
%%% End:
