\documentclass{article}
\usepackage[a4paper,margin=1.875in,top=1.4in,bottom=1.4in]{geometry}

\usepackage{amsmath,mathtools,bbm,amssymb}
\usepackage{mathrsfs}
\usepackage{german}

\usepackage{setspace}
\doublespacing

\usepackage{fancyhdr}
\renewcommand{\headrulewidth}{0pt} 
\pagestyle{fancy}
\lhead{Blatt 7 Nikolaus, Lukas, Evgenij}\rhead{Seite \thepage}
\fancyfoot{}

\usepackage{tikz}
\usetikzlibrary{decorations.pathreplacing,arrows}

\usepackage[numbers]{natbib}
\bibliographystyle{alphadin}
\usepackage{url}
\usepackage{hyperref}

\usepackage{eurosym}

\begin{document}
\paragraph{Aufgabe 3 \textnormal{(4 Punkte)}.}
Zeige, dass der Prozess $\rho$, gegeben durch
\[
  \rho_t=e^{(t-t_0)\beta}x+\int_{t_0}^te^{(t-s)\beta}\alpha(s)ds+\int_{t_0}^te^{(t-s)\beta}\sigma dW_s\,,
\]
für $\rho_{t_0}=x$, eine Lösung des \emph{Hull-White-extended Vasicek-Modells} ist, das durch die SDE
\[
  d\rho_t=(\alpha(t)+\beta \rho_t)dt+\sigma dW_t
\]
gegeben ist, wobei $W$ eine Brownsche Bewegung, $\beta\in\mathbb{R}$ der Drift, $\sigma\in\mathbb{R}_+$ die Volatilität und $\alpha\in{\cal C}(\mathbb{R}_+)$ die Hull-White-Erweiterung ist.
\paragraph{\textnormal{\emph{Lösung:}}} Definiere $h(t,\rho_t)=e^{(t_0-t)\beta}\rho_t$ und berechne die Ableitungen
\[
  \frac{\partial h}{\partial t}=-\beta e^{(t_0-t)\beta}\rho_t\,,\quad
  \frac{\partial h}{\partial \rho_t}=e^{(t_0-t)\beta}\,,\quad
  \frac{\partial^2 h}{\partial\rho_t^2}=0\,.
\]
Mit der Ito-Formel ergibt sich
\begin{align*}
  dh(t,\rho_t)
  &=d(e^{(t_0-t)\beta}\rho_t)\\
  &=\frac{\partial h}{\partial t}+\frac{\partial h}{\partial \rho_t}\left((\alpha(t)+\beta \rho_t)+\frac{1}{2}\frac{\partial^2h}{\partial \rho_t^2} \sigma^2\right)dt+\frac{\partial h}{\partial \rho_t}\sigma dW_t\\
  &=\bigl(-\beta e^{(t_0-t)\beta}\rho_t+(\alpha(t)+\beta \rho_t)e^{(t_0-t)\beta}\bigr)dt+\sigma e^{(t_0-t)\beta}dW_t\\
  &=\alpha(t)e^{(t_0-t)\beta}dt+\sigma e^{(t_0-t)\beta}dW_t
    \intertext{wodurch wir in integraler Schreibweise haben, dass}
    h(t,\rho_t)
  &=h(t_0,\rho_{t_0})+\int_{t_0}^te^{(t_0-t+s)\beta}\alpha(s)ds+\int_{t_0}^te^{(t_0-t+s)}\sigma dW_s\,,
    \intertext{wobei $h(t_0,\rho_{t_0})=\rho_{t_0}$ Durch Multiplizieren der Gleichung mit $e^{(t_0-t)\beta}$ kriegen wir schließlich}
    \rho_t
  &=e^{(t-t_0)\beta}x+\int_{t_0}^te^{(t-s)\beta}\alpha(s)ds+\int_{t_0}^te^{(t-s)\beta}\sigma dW_s\,,
\end{align*}
\bibliography{../../../books/wt}
\end{document}

%%% Local Variables:
%%% mode: latex
%%% ispell-local-dictionary: "german"
%%% TeX-master: t
%%% End:
