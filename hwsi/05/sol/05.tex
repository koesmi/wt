\documentclass{article}
\usepackage[a4paper,margin=1.875in,top=1.875in,bottom=1.875in]{geometry}

\usepackage{amsmath,mathtools,bbm,amssymb}
\usepackage{mathrsfs}
\usepackage{german}

\usepackage{setspace}
\doublespacing

\usepackage{fancyhdr}
\renewcommand{\headrulewidth}{0pt} 
\pagestyle{fancy}
\lhead{Blatt 5 Nikolaus, Lukas, Evgenij}\rhead{Seite \thepage}
\fancyfoot{}

\usepackage{tikz}
\usetikzlibrary{decorations.pathreplacing,arrows}

\usepackage[numbers]{natbib}
\bibliographystyle{alphadin}
\usepackage{url}
\usepackage{hyperref}

\begin{document}
\paragraph{Aufgabe 1 \textnormal{(4 Punkte)}.}
Es seien $F$ und $G$ stetige Funktionen und $f$ die Lösung der gewöhnlichen Differentialgleichung $f'(t)=F(s)f(s)$ mit $f(0)=1$.
Ferner sei $W$ eine Brownsche Bewegung.
Zeigen Sie, dass der Prozess $X$ gegeben durch
\[
X_t:=f(t)(x+\int_0^tf(s)^{-1}G(s)dW_s)
\]
folgende Darstellung besitzt
\[
  dX_t=F(t)X_tdt+G(t)dW_t,\quad X_0=x\,.
\]

\paragraph{\textnormal{\emph{Lösung.}}}
Mit $A_t=f(t)$ und $B_t=x+\int_0^t f(s)^{-1} G(s)dW_s$ und partieller Integration
\[
  AB=A_0B_0-A_-\cdot B-B_-\cdot A-[A,B]=A_0B_0-A\cdot B-B\cdot A
\]
folgt
\begin{align*}
  &f(t)(x+\int_0^tf(s)^{-1}G(s)dW_s)\\
  &=f(0)x+\int_0^tf(s)d(\int_0^sf(u)^{-1}G(u)dW_u)+\int_0^t (x+\int_0^sf(u)^{-1} G(u)dW_u)df(s)
    \intertext{und mit dem Hauptsatz der Differential- und Integralrechnung sowie der Differentialgleichung $f'(t)=F(s)f(s)$ aus der Aufgabenstellung}
  &=x+\int_0^tG(s)dW_s+\int_0^tF(s)f(s)(x+\int_0^sf^{-1}(u)G(u)dW_u)ds\,.
    \intertext{Mit der Definition von $X_t$ kriegen wir schließlich}
  &=x+\int_0^tG(s)dW_s+\int_0^tF(s)X_sds\,.
\end{align*}

\bibliography{../../../books/wt}
\end{document}

%%% Local Variables:
%%% mode: latex
%%% ispell-local-dictionary: "german"
%%% TeX-master: t
%%% End:
