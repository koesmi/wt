\documentclass{article}
\usepackage[a4paper,margin=1.875in,top=1.4in,bottom=1.4in]{geometry}

\usepackage{amsmath,mathtools,bbm,amssymb}
\usepackage{mathrsfs}
\usepackage{german}

\usepackage{setspace}
\doublespacing

\usepackage{fancyhdr}
\renewcommand{\headrulewidth}{0pt} 
\pagestyle{fancy}
\lhead{Blatt 5 Nikolaus, Lukas, Evgenij}\rhead{Seite \thepage}
\fancyfoot{}

\usepackage{tikz}
\usetikzlibrary{decorations.pathreplacing,arrows}

\usepackage[numbers]{natbib}
\bibliographystyle{alphadin}
\usepackage{url}
\usepackage{hyperref}

\usepackage{eurosym}

\begin{document}
\paragraph{Aufgabe 1 \textnormal{(4 Punkte)}.}
Es seien $F$ und $G$ stetige Funktionen und $f$ die Lösung der gewöhnlichen Differentialgleichung $f'(t)=F(s)f(s)$ mit $f(0)=1$.
Ferner sei $W$ eine Brownsche Bewegung.
Zeigen Sie, dass der Prozess $X$ gegeben durch
\[
X_t:=f(t)(x+\int_0^tf(s)^{-1}G(s)dW_s)
\]
folgende Darstellung besitzt
\[
  dX_t=F(t)X_tdt+G(t)dW_t,\quad X_0=x\,.
\]

\paragraph{\textnormal{\emph{Lösung.}}}
Mit $A_t=f(t)$ und $B_t=x+\int_0^t f(s)^{-1} G(s)dW_s$ und partieller Integration
\[
  AB=A_0B_0-A_-\cdot B-B_-\cdot A-[A,B]=A_0B_0-A\cdot B-B\cdot A
\]
folgt
\begin{align*}
  &f(t)(x+\int_0^tf(s)^{-1}G(s)dW_s)\\
  &=f(0)x+\int_0^tf(s)d(\int_0^sf(u)^{-1}G(u)dW_u)+\int_0^t (x+\int_0^sf(u)^{-1} G(u)dW_u)df(s)
    \intertext{und mit dem Hauptsatz der Differential- und Integralrechnung sowie der Differentialgleichung $f'(t)=F(s)f(s)$ aus der Aufgabenstellung}
  &=x+\int_0^tG(s)dW_s+\int_0^tF(s)f(s)(x+\int_0^sf^{-1}(u)G(u)dW_u)ds\,.
    \intertext{Mit der Definition von $X_t$ kriegen wir schließlich}
  &=x+\int_0^tG(s)dW_s+\int_0^tF(s)X_sds\,.
\end{align*}
\pagebreak
\paragraph{Aufgabe 2 \textnormal{(Siegels Paradoxon; 4 Punkte)}.} Bezeichne mit $\$_t$ den Preis eines US-Dollars in Euro zum Zeitpunkt $t$ und mit $\text{\euro}_t$ den Preis eines Euros in US-Dollar.
Angenommen, es gilt
\[
  \$_0=1,\quad d\$_t=\$_t(\mu dt+\sigma dW_t)
\]
für einen Wiener-Prozess $W_t$.
\begin{enumerate}
\item [1.] Leite die SDE für $\text{\euro}$ ab.
\end{enumerate}
Aus der Definition von $\text{\euro}_t$ und $\$_t$ folgt, dass $\text{\euro}_t=1/\$_t$, also
\[
  d\text{\euro}_t=\frac{d\text{\euro}_t}{d\$_t}d\$_t=-\frac{1}{\$_t^2}\$_t(\mu dt+\sigma dW_t)=-\text{\euro}_t(\mu dt+\sigma dW_t)\,.
\]
\begin{enumerate}
\item [2.] Sei $\sigma^2>\mu$.
  Berechne $E[\$_t-\$_0]$, d.h. den erwarteten Gewinn in Euro aus einer Investition von 1 US-Dollar.
  Ist der US-Dollar aus dieser Perspektive eine attraktive Investition?
  Berechne außerdem $E[\text{\euro}_t-\text{\euro}_0]$, d.h. den erwarteten Gewinn in US-Dollar aus der Sicht eines US-Investors.
  Ist der Euro aus letzterer Sicht eine attraktive Investition?
\end{enumerate}
In Beispiel 16 hatten wir die Lösung des Black--Scholes-Modells gegeben.
Setzen wir sie ein, erhalten wir
\begin{align*}
  E[\$_t-\$_0]
  &=\$_0 E\Bigl[\exp\Bigl(\mu t+\sigma W_t-\frac{1}{2}\sigma^2 t\Bigr)-1\Bigr]\,.
    \intertext{Mit dem Erwartungswert der Log-Normalverteilung erhalten wir}
  &=\$_0\bigl(\exp(\mu t)-1\bigr)\,.
\end{align*}
Ist $\mu>0$, so ist dieser Wert positiv, also eine attraktive Investition.
Analog rechnen wir
\begin{align*}
  E[\text{\euro}_t-\text{\euro}_0]
  &=\text{\euro}_0E\Bigl[\exp\bigl((-\mu)t+(-\sigma)W_t-\frac{1}{2}\sigma^2t\bigr)-1\Bigr]\\
  &=\text{\euro}_0\Bigl(\exp\Bigl(-\mu t+\frac{1}{2}\sigma^2t\Bigr)-1\Bigr)
\end{align*}
Für $\mu>\frac{1}{2}\sigma^2$ ist das eine attraktive Investition.
\bibliography{../../../books/wt}
\end{document}

%%% Local Variables:
%%% mode: latex
%%% ispell-local-dictionary: "german"
%%% TeX-master: t
%%% End:
