\documentclass{article}
\usepackage[a4paper,margin=1.875in,top=1.3in,bottom=1.3in]{geometry}

\usepackage{amsmath,mathtools,bbm,amssymb}
\usepackage{mathrsfs}
\usepackage{babel}

\usepackage{setspace}
\doublespacing

\usepackage{fancyhdr}
\renewcommand{\headrulewidth}{0pt} 
\pagestyle{fancy}
\lhead{Blatt 1 Nikolaus, Lukas, Evgenij}\rhead{Seite \thepage}
\fancyfoot{}

\usepackage{tikz}
\usetikzlibrary{decorations.pathreplacing,arrows}

\usepackage[numbers]{natbib}
\bibliographystyle{alphadin}
\usepackage{url}
\usepackage{hyperref}

\begin{document}
%\paragraph{Aufgabe 1 \textnormal{(4 Punkte)}.}
%
%Suche eine Dichte $Y=\frac{\mathrm{d}Q}{\mathrm{d}P}$ und nehme an, dass $Y=\prod_{i=1}^t Y_i$
%\emph{weiter aus \url{https://math.stackexchange.com/questions/2741575/constructing-equivalent-martingale-measure-lognormal-distribution}}
%$E_Q[A]=\frac{1}{E[Y]}E[1_A Y]$ nach Proposition A.16.
\paragraph{Aufgabe 2 \textnormal{(4 Punkte)}.}
Zeigen Sie, dass jedes Supermartingal $(X_n)_{n\in\mathbb{N}_0}$ mit $E[X_n]=E[X_0]$ für alle $n$ bereits ein Martingal ist.

Sei $Y_n=X_n-E[X_{n+1}|\mathscr{F}_n]$.
Dann gilt nach Turmeigenschaft, dass $E[Y_n]=E[X_n]-E[E[X_{n+1}]|\mathscr{F}_n]=0$.
Da $X$ ein Supermartingal ist, ist $Y$ nichtnega\-tiv.
Somit ist auch $Y_n=0$ und $X$ ein Martingal.
\bibliography{../../../books/wt}
\end{document}

%%% Local Variables:
%%% mode: latex
%%% ispell-local-dictionary: "german"
%%% TeX-master: t
%%% End:
