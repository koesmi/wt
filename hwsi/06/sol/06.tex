\documentclass{article}
\usepackage[a4paper,margin=1.875in,top=1.875in,bottom=1.875in]{geometry}

\usepackage{amsmath,mathtools,bbm,amssymb}
\usepackage{mathrsfs}
\usepackage{german}

\usepackage{setspace}
\doublespacing

\usepackage{fancyhdr}
\renewcommand{\headrulewidth}{0pt} 
\pagestyle{fancy}
\lhead{Blatt 6 Nikolaus, Lukas, Evgenij}\rhead{Seite \thepage}
\fancyfoot{}

\usepackage{tikz}
\usetikzlibrary{decorations.pathreplacing,arrows}

\usepackage[numbers]{natbib}
\bibliographystyle{alphadin}
\usepackage{url}
\usepackage{hyperref}

\usepackage{eurosym}

\begin{document}
\paragraph{Aufgabe 2 \textnormal{(CIR-Prozess; 4 Punkte)}.}
\begin{enumerate}
\item [1.] Überprüfen Sie die Zulässigkeitsbedingungen in \cite{duffie}, um herauszufinden, unter welchen Bedingungen an $b,\beta, \sigma\in\mathbb{R}$ ein affinierter Prozess $X$ auf $\mathbb{R}_+$ existiert, der
  \[
    dX_t=(b+\beta X_t)dt+\sigma\sqrt{X_t}dW_t
  \]
  für eine Brownsche Bewegung $W$ erfüllt.
\item [2.] Schreiben Sie die Funktionen $F$ und $R$ auf, die diesen Parametern gemäß \cite[Theorem 7]{cuchiero2013path} oder \cite[Theorem 2.7]{duffie} zugeordnet sind und finden Sie Lösungen $\phi,\psi\colon \mathbb{R}_+\times \mathbb{C}_-\to\mathbb{C}$ der Riccati-Gleichungen
  \[
    \partial_t\phi(t,u)=F\bigl(\psi(t,u)\bigr),\quad \partial_t\psi(t,u)= R\bigl(\psi(t,u)\bigr)\,.
  \]
  \emph{Hinweis: Sie können die Lösungen $\phi$, $\psi$ in der Literatur finden und überprüfen, dass sie die Riccati-Gleichungen erfüllen.}
  Nach \cite[Theorem 2.7]{duffie} gilt
  \[
    F(u)=au^2+bu-c+\int_{\mathbb{R}\setminus\{0\}}\bigl(e^{u\xi}-1-\xi\bigr)
    %was is \xi in einer Dimension?
  \]
\end{enumerate}

\paragraph{Aufgabe 3 \textnormal{(4 Punkte)}.}
Sei $W$ eine Standard Brown'sche Bewegung.
Zeigen Sie, dass der Prozess
\[
B_t=(1-t)\int_0^t\frac{1}{1-s}dW_s
\]
folgende Gleichung erfüllt
\[
dB_t=-\frac{B_t}{1-t}dt+dW_t
\]
und zeigen Sie, dass $\lim_{t\nearrow 1}E[B_t]=0$.

\paragraph{\textnormal{\emph{Lösung.}}}
Wir definieren $Y_t:=(1-t)$.
Dann gilt mit der Definition des stochastischen Integrals, mit $W_0=0$ und mit der partiellen Integration
\[
B_t=Y_t(1/Y\cdot W)_t=\bigl(Y\cdot(1/Y\cdot W)\bigr)_t+\bigl((1/Y\cdot W)\cdot Y\bigr)_t+\bigl[Y,(1/Y\cdot W)\bigr]_t
\]
denn $Y$ und $1/Y\cdot W$ sind stetig.
Da $Y$ stetig ist, gilt mit Theorem 81.iv, dass $\bigl[Y,(1/Y\cdot W)\bigr]=0$.
Mit der Kettenregel folgt dann \emph{wobei ich nicht ganz weiß, wie die Kettenregel funktioniert,}
\[
\bigl(Y\cdot (1/Y\cdot W)\bigr)_t=W_t
\]
und da $dY=Y'dt$ folgt
\[
\bigl((1/Y\cdot W)\cdot Y\bigr)_t=-\int_0^t1/Y\cdot W_s ds\,.
\]
Insgesamt erhält man die Darstellung
\[
dB_t=-\frac{B_t}{1=t}dt+dW_t\,.
\]

Für den zweiten Teil gilt, weil $Y$ deterministisch ist
\begin{align*}
  E\bigl[Y_t^2(1/Y\cdot W)_t^2\bigr]
  &=Y_t^2E\bigl[(1/Y\cdot W)_t^2\bigr]\,.
    \intertext{Mit der Itô-Isometrie können wir schreiben}
  &=Y_t^2\left[\int\nolimits_0^t(1/Y_s)^2d\langle W\rangle_s\right]
    \intertext{Nun ist $\langle W\rangle_t=t$ und $(1/Y_t)'=(1/Y_t)^2$, sodass}
  &=(1-t)^2E\left[\left.\frac{1}{1-s}\right|_{s=0}^t\right]=t(1-t)\,,
\end{align*}
wodurch
\[
  \lim_{t\nearrow1}E[B_t]=\lim_{t\nearrow1}t(1-t)=0\,.
\]
\bibliography{../../../books/wt}
\end{document}

%%% Local Variables:
%%% mode: latex
%%% ispell-local-dictionary: "german"
%%% TeX-master: t
%%% End:
