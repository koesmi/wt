\documentclass{article}
\usepackage[a4paper,margin=1.875in,top=1.0in,bottom=1.0in]{geometry}

\usepackage{amsmath,mathtools,bbm,amssymb}
\usepackage{mathrsfs}
\usepackage{german}

\usepackage{setspace}
\doublespacing

\usepackage{fancyhdr}
\renewcommand{\headrulewidth}{0pt} 
\pagestyle{fancy}
\lhead{Blatt 2 Nikolaus, Lukas, Evgenij}\rhead{Seite \thepage}
\fancyfoot{}

\usepackage{tikz}
\usetikzlibrary{decorations.pathreplacing,arrows}

\usepackage[numbers]{natbib}
\bibliographystyle{alphadin}
\usepackage{url}
\usepackage{hyperref}

\begin{document}
\paragraph{Aufgabe 5 \textnormal{(2 Punkte)}.}
Betrachten Sie das Modell aus Aufgabe 3 und die Funktion $v_t$ aus Aufgabe 4.
Geben Sie eine explizite Darstellung für $v_t$ an, falls $H=h(S_T)$.

Das wurde in Beispiel 5.42. in \cite{foellmer2016} gemacht.
Nach Aufgabe 4 ist für $t=T-1,\dots,0$
\begin{align*}
  v_T(x_0,\dots,x_T)
  &=h(x_0,\dots,x_T)\,,\\
  v_t(x_0,\dots,x_t)
  &=q v_{t+1}(x_0\dots,x_t,x_t(1+b))+(1-q)v_{t+1}(x_0,\dots,x_t,x_t(1+a))\,.
\end{align*}
Wenn man nun $H=h(S_T)$ verwendet, so gilt mit $\bar{q}=1-q$, $\hat{a}=1+a$ und $\hat{b}=1+b$
%\begin{align*}
%v_t(x_0,\dots,x_t)
%&=\mathbb{E}^*\left[h\left(x_0,\dots,x_t,x_t\frac{S_1}{S_0},\dots,x_t\frac{S_{T-t}}{S_0}\right)\right]
%\end{align*}
%Sodass $v_t(x_t)=\mathbb{E}^*[h(x_t\frac{S_{T-t}}{S_0})]$
\begin{align*}
  v_T(x_T)
  &=h(x_T)\,\\
    v_{T-1}(x_{T-1})
  &=q h(x_{T-1}\hat{b})+\bar{q}h(x_{T-1}\hat{a})\\
  v_{T-2}(x_{T-2})
  &=qv_{T-1}(x_{T-2}\hat{b})+\bar{q}v_{T-1}(x_{T-2}\hat{a})\\
  &=q\Bigl(q h(x_{T-2}\hat{b}\hat{b})+\bar{q}h(x_{T-2}\hat{b}\hat{a})\Bigr)\\
  &\quad+\bar{q}\Bigl(q h(x_{T-2}\hat{a}\hat{b})+\bar{q}h(x_{T-2}\hat{a}\hat{a})\Bigr)\\
  &=q^2\bar{q}^0h(x_{T-2}\hat{b}^2\hat{a}^0)+2q^1\bar{q}^1h(x_{T-2}\hat{b}^1\hat{a}^1)+q^0\bar{q}^2h(x_{T-2}\hat{b}^0\hat{a}^2)\,.
    \intertext{Dementsprechend ergibt sich für $v_{T-3}(x_{T-3})$}
    v_{T-3}(x_{T-3})
  &=q^3\bar{q}^0h(x_{T-3}\hat{b}^3\hat{a}^0)+3q^2\bar{q}^1h(x_{T-3}\hat{b}^2\hat{a}^1)\\
  &\quad+3q^1\bar{q}^2h(x_{T-3}\hat{b}^1\hat{q}^2)+q^0\bar{q}^3h(x_{T-3}\hat{b}^0\hat{a}^3)\,,
    \intertext{und insgesamt mit dem Pascalschen Dreieck}
    v_t(x_t)
  &=\sum_{k=0}^{T-t}h
    \bigl(\!\begin{smallmatrix}
      T-t\\k
    \end{smallmatrix}\!\bigr)
  q^k\bar{q}^{T-t-k}h(x_t\hat{b}^k\hat{a}^{T-t-k})\,.
\end{align*}
\pagebreak
\paragraph{Aufgabe 6 \textnormal{(2 Punkte)}.}
Betrachten Sie das Modell aus Aufgabe 3.
Geben Sie die zu Aufgabe 4 korrespondierende replizierende Strategie explizit an.

Das ist Proposition 5.44 aus \cite{foellmer2016}.
Wir behaupten, die replizierende Strategie hat die Form
\[
\xi_t(\omega)=\Delta_t(S_0,S_1(\omega),\dots,S_{t-1}(\omega))\,,
\]
wobei
\begin{equation}
  \begin{split}
  &\Delta_t(x_0,\dots,x_{t-1})\\
  &:=(1+r)^t\frac{v_t(x_0,\dots,x_{t-1},x_{t-1}\hat{b})-v_t(x_0,\dots,x_{t-1},x_{t-1}\hat{a})}{x_{t-1}\hat{b}-x_{t-1}\hat{a}}\,.
\end{split}
\label{eq:deltat}
\end{equation}
Damit $\xi_t$ eine replizierende Strategie ist, muss gelten,
\begin{equation}
  \label{eq:xit}
  \xi_t(\omega)(X_t(\omega)-X_{t-1}(\omega))=V_t(\omega)-V_{t-1}(\omega)
\end{equation}
\emph{wobei nach Definition einer replizierenden Strategie zu erwarten wäre, dass gilt $\xi_t(\omega)X_t(\omega)=V_t(\omega)$.}
Nach Definition von $X_t$ und $V_t$ hängen $X_{t-1}$ und $V_{t-1}$ nur von den ersten $t-1$ Komponenten von $\omega$ ab.
Auch $\xi_t$ hängt nur von den ersten $t-1$ Komponenten von $\omega$ ab (\emph{wobei unklar ist, warum}).
Für ein festes $t$ seien
\[
  \omega^\pm:=(y_1,\dots,y_{t-1},\pm1,y_{t+1},\dots,y_T)\,,
\]
Sodass $R_t(\omega^+)=b$ und $R_t(\omega^-)=a$.
Einsetzen in (\ref{eq:xit}) liefert
\begin{align*}
  \xi_t(\omega)\bigl(X_{t-1}(\omega)\hat{b}/(1+r)-X_{t-1}(\omega)\bigr)
  &=V_t(\omega^+)-V_{t-1}(\omega)\,,\\
    \xi_t(\omega)\bigl(X_{t-1}(\omega)\hat{a}/(1+r)-X_{t-1}(\omega)\bigr)
  &=V_t(\omega^-)-V_{t-1}(\omega)
\end{align*}
und mit Umstellen nach $\xi_t$ erhalten wir
\begin{align*}
  \xi_t(\omega)
  &=(1+r)\frac{V_t(\omega^+)-V_t(\omega^-)}{X_{t-1}(\omega)(\hat{b}-\hat{a})}
  \intertext{Nach der Definition $X_t$ aus Aufgabe 3 und von $V_t(\omega)$ aus Aufgabe 4 hat dann $\xi_t$ unter Verwendung von $\Delta_t$ aus (\ref{eq:deltat}) die Form}
  &=\Delta\bigl(S_0,S_1(\omega),\dots,S_{t-1}(\omega)\bigr)
\end{align*}
\bibliography{../../../books/wt}
\end{document}

%%% Local Variables:
%%% mode: latex
%%% ispell-local-dictionary: "german"
%%% TeX-master: t
%%% End:
