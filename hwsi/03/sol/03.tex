\documentclass{article}
\usepackage[a4paper,margin=1.875in,top=1.3in,bottom=1.3in]{geometry}

\usepackage{amsmath,mathtools,bbm,amssymb}
\usepackage{mathrsfs}
\usepackage{german}

\usepackage{setspace}
\doublespacing

\usepackage{fancyhdr}
\renewcommand{\headrulewidth}{0pt} 
\pagestyle{fancy}
\lhead{Blatt 2 Nikolaus, Lukas, Evgenij}\rhead{Seite \thepage}
\fancyfoot{}

\usepackage{tikz}
\usetikzlibrary{decorations.pathreplacing,arrows}

\usepackage[numbers]{natbib}
\bibliographystyle{alphadin}
\usepackage{url}
\usepackage{hyperref}

\begin{document}
\paragraph{Aufgabe 5 \textnormal{(2 Punkte)}.}
Betrachten Sie das Modell aus Aufgabe 3 und die Funktion $v_t$ aus Aufgabe 4.
Geben Sie eine explizite Darstellung für $v_t$ an, falls $H=h(S_T)$.

Das wurde in Beispiel 5.42. in \cite{foellmer2016} gemacht.
Nach Aufgabe 4 ist für $t=T-1,\dots,0$
\begin{align*}
  v_T(x_0,\dots,x_T)
  &=h(x_0,\dots,x_T)\,,\\
  v_t(x_0,\dots,x_t)
  &=q v_{t+1}(x_0\dots,x_t,x_t(1+b))+(1-q)v_{t+1}(x_0,\dots,x_t,x_t(1+a))\,.
\end{align*}
Wenn man nun $H=h(S_T)$ verwendet, so gilt mit $\bar{q}=1-q$, $\hat{a}=1+a$ und $\hat{b}=1+b$
%\begin{align*}
%v_t(x_0,\dots,x_t)
%&=\mathbb{E}^*\left[h\left(x_0,\dots,x_t,x_t\frac{S_1}{S_0},\dots,x_t\frac{S_{T-t}}{S_0}\right)\right]
%\end{align*}
%Sodass $v_t(x_t)=\mathbb{E}^*[h(x_t\frac{S_{T-t}}{S_0})]$
\begin{align*}
  v_T(x_T)
  &=h(x_T)\,\\
    v_{T-1}(x_{T-1})
  &=q h(x_{T-1}\hat{b})+\bar{q}h(x_{T-1}\hat{a})\\
  v_{T-2}(x_{T-2})
  &=qv_{T-1}(x_{T-2}\hat{b})+\bar{q}v_{T-1}(x_{T-2}\hat{a})\\
  &=q\Bigl(q h(x_{T-2}\hat{b}\hat{b})+\bar{q}h(x_{T-2}\hat{b}\hat{a})\Bigr)\\
  &\quad+\bar{q}\Bigl(q h(x_{T-2}\hat{a}\hat{b})+\bar{q}h(x_{T-2}\hat{a}\hat{a})\Bigr)\\
  &=q^2\bar{q}^0h(x_{T-2}\hat{b}^2\hat{a}^0)+2q^1\bar{q}^1h(x_{T-2}\hat{b}^1\hat{a}^1)+q^0\bar{q}^2h(x_{T-2}\hat{b}^0\hat{a}^2)\,.
    \intertext{Dementsprechend ergibt sich für $v_{T-3}(x_{T-3})$}
    v_{T-3}(x_{T-3})
  &=q^3\bar{q}^0h(x_{T-3}\hat{b}^3\hat{a}^0)+3q^2\bar{q}^1h(x_{T-3}\hat{b}^2\hat{a}^1)\\
  &\quad+3q^1\bar{q}^2h(x_{T-3}\hat{b}^1\hat{q}^2)+q^0\bar{q}^3h(x_{T-3}\hat{b}^0\hat{a}^3)\,,
    \intertext{und insgesamt mit dem Pascalschen Dreieck}
    v_t(x_t)
  &=\sum_{k=0}^{T-t}h
    \bigl(\!\begin{smallmatrix}
      T-t\\k
    \end{smallmatrix}\!\bigr)
  q^k\bar{q}^{T-t-k}h(x_t\hat{b}^k\hat{a}^{T-t-k})\,.
\end{align*}

\bibliography{../../../books/wt}
\end{document}

%%% Local Variables:
%%% mode: latex
%%% ispell-local-dictionary: "german"
%%% TeX-master: t
%%% End:
