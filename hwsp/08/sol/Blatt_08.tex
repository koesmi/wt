\documentclass{article}
\usepackage[a4paper,margin=1.875in,top=1.575in,bottom=1.575in]{geometry}
\usepackage{amsmath,mathtools,bbm,amssymb,stmaryrd,wasysym}
\usepackage{mathrsfs}
\usepackage{babel}

\usepackage{setspace}
\doublespacing

\usepackage{fancyhdr}
\renewcommand{\headrulewidth}{0pt} 
\pagestyle{fancy}
\lhead{Blatt 8 Nicole und Evgenij}\rhead{Seite \thepage}
\fancyfoot{}

\usepackage{tikz}
\usetikzlibrary{decorations.pathreplacing,arrows}

\usepackage[numbers]{natbib}
\bibliographystyle{alphadin}
\usepackage{url}
\usepackage{hyperref}

\begin{document}

\paragraph{\normalfont{\itshape Definition} 1 (ucp-Metrik).}
Sei ${\cal D}$ die Menge aller adaptierten càdlàg Prozesse von $\Omega\times\mathbb{R}_+$ nach $\mathbb{R}$.
Wir definieren die Metrik $d_{ucp}\colon {\cal D}\times {\cal D}\to[0,\infty)$ durch
\begin{equation}
(X,Y)\mapsto \sum_{n\in\mathbb{N}}2^{-n}E[(X-Y)^*_n\wedge 1]\,,
\end{equation}
wobei $X^*_n:=\sup_{s\leq n}|X_s|$.
Ebenso definiert $d_{ucp}$ eine Metrik auf dem Raum aller adaptierten càglàd Prozesse.

\paragraph{Aufgabe 1 \textnormal{(3 Punkte)}.}
Zeigen Sie folgende Aussagen:
\begin{enumerate}
\item [i)] Für Zufallsvariablen $X,X^1,X^2,\dots$ gilt stochastische Konvergenz $X^m\to X$ genau dann, wenn $E[|X^m-X|\wedge 1]\rightarrow0$.
\end{enumerate}
Das ist Lemma 17 aus dem Skript zur Wahrscheinlichkeitstheorie.
Es gelte zunächst $X^m\xrightarrow{P}X$ und sei $0<\varepsilon<1$ beliebig.
Dann gilt
\begin{align*}
  \lim_{n\to\infty}E[|X^m-X|\wedge1]
  &=\lim_{n\to\infty}\bigl(E[|X^m-X|\wedge1]\mathbbm{1}_{\{|X^m-X|>\varepsilon\}}\\
  &\quad+E[|X^m-X|]\mathbbm{1}_{\{|X^m-X|\leq\varepsilon\}}\bigr)\\
  &\leq\lim_{n\to\infty}\bigl(\mathbbm{1}_{\{|X^m-X|>\varepsilon\}}+\varepsilon\bigr)=\varepsilon\,,
\end{align*}
denn $X^m$ konvergiert stochastisch gegen $X$.
Gilt umgekehrt $0<\varepsilon\leq 1$ und $E[|X^m-X|\wedge 1]\to0$, so gilt $P(|X^m-X|>\varepsilon)\leq\frac{E[|X^m-X|\wedge1]}{\varepsilon}\to0$ nach Anwendung der Markov-Ungleichung.
\paragraph{Aufgabe 5 \textnormal{(4 Punkte)}.}
Es sei $X\in\mathscr{H}_{\text{loc}}^2$ beliebig.
Dann liegt jeder lokal beschränkte, previsible Prozess in $L_{\text{loc}}^2(X)$.

Zunächst einmal gilt für $X\in\mathscr{H}_{\text{loc}}^2$, dass $\langle X,X\rangle\in\mathscr{V}$.
Es gilt sogar, dass $\langle X,X\rangle\in\mathscr{A}_{\text{loc}}^+$, \emph{wobei das noch gezeigt werden sollte}.
Somit reicht es zu zeigen, dass für $X\in\mathscr{A}_{\text{loc}}^+$ und $H$ lokal beschränkt und previsibel gilt, dass $H\cdot X\in\mathscr{A}_{\text{loc}}^+$.
Seien $H$ und $X$ entsprechend gewählt, dann existiert eine Folge von Stoppzeiten $T_n$ mit $T_n\uparrow\infty$, sodass $X^{T_n}\in\mathscr{V}$ und $E[(X^{T_n})_\infty]<\infty$.
Nach Theorem 93 gilt auch $H\cdot X^{T_n}\in\mathscr{V}$.
\emph{Hier müsste noch gefolgert werden, dass es auch eine Folge $S_n\uparrow\infty$ von Stoppzeiten gibt, sodass $(H\cdot X)^{S_n}\in\mathscr{V}$.}
Da $H\mapsto H\cdot X$ linear ist, ist außerdem $E[(H\cdot X)^{S_n}]<\infty$.
Somit ist $H\cdot X\in\mathscr{A}_{\text{loc}}^+$.
\bibliography{../../../books/wt}
\end{document}

%%% Local Variables:
%%% mode: latex
%%% ispell-local-dictionary: "german"
%%% TeX-master: t
%%% End:
