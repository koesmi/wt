\documentclass{article}
\usepackage[a4paper,margin=1.875in,top=1.575in,bottom=1.575in]{geometry}
\usepackage{amsmath,mathtools,bbm,amssymb,stmaryrd,wasysym}
\usepackage{mathrsfs}
\usepackage{babel}

\usepackage{setspace}
\doublespacing

\usepackage{fancyhdr}
\renewcommand{\headrulewidth}{0pt} 
\pagestyle{fancy}
\lhead{Blatt 8 Nicole und Evgenij}\rhead{Seite \thepage}
\fancyfoot{}

\usepackage{tikz}
\usetikzlibrary{decorations.pathreplacing,arrows}

\usepackage[numbers]{natbib}
\bibliographystyle{alphadin}
\usepackage{url}
\usepackage{hyperref}

\begin{document}

\paragraph{Aufgabe 5 \textnormal{(4 Punkte)}.}
Es sei $X\in\mathscr{H}_{\text{loc}}^2$ beliebig.
Dann liegt jeder lokal beschränkte, previsible Prozess in $L_{\text{loc}}^2(X)$.

Zunächst einmal gilt für $X\in\mathscr{H}_{\text{loc}}^2$, dass $\langle X,X\rangle\in\mathscr{V}$.
Es gilt sogar, dass $\langle X,X\rangle\in\mathscr{A}_{\text{loc}}^+$, \emph{wobei das noch gezeigt werden sollte}.
Somit reicht es zu zeigen, dass für $X\in\mathscr{A}_{\text{loc}}^+$ und $H$ lokal beschränkt und previsibel gilt, dass $H\cdot X\in\mathscr{A}_{\text{loc}}^+$.
Seien $H$ und $X$ entsprechend gewählt, dann existiert eine Folge von Stoppzeiten $T_n$ mit $T_n\uparrow\infty$, sodass $X^{T_n}\in\mathscr{V}$ und $E[(X^{T_n})_\infty]<\infty$.
Nach Theorem 93 gilt auch $H\cdot X^{T_n}\in\mathscr{V}$.
\emph{Hier müsste noch gefolgert werden, dass es auch eine Folge $S_n\uparrow\infty$ von Stoppzeiten gibt, sodass $(H\cdot X)^{S_n}\in\mathscr{V}$.}
Da $H\mapsto H\cdot X$ linear ist, ist außerdem $E[(H\cdot X)^{S_n}]<\infty$.
Somit ist $H\cdot X\in\mathscr{A}_{\text{loc}}^+$.
\bibliography{../../../books/wt}
\end{document}

%%% Local Variables:
%%% mode: latex
%%% ispell-local-dictionary: "german"
%%% TeX-master: t
%%% End:
