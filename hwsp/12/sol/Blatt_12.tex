\documentclass{article}
\usepackage[a4paper,margin=1.875in,top=1.875in,bottom=1.875in]{geometry}
\usepackage{amsmath,mathtools,bbm,amssymb,stmaryrd,wasysym}
\usepackage{mathrsfs}
\usepackage{babel}

\usepackage{setspace}
\doublespacing

\usepackage{fancyhdr}
\renewcommand{\headrulewidth}{0pt} 
\pagestyle{fancy}
\lhead{Blatt 12 Nicole und Evgenij}\rhead{Seite \thepage}
\fancyfoot{}

\usepackage{tikz}
\usetikzlibrary{decorations.pathreplacing,arrows}

\usepackage[numbers]{natbib}
\bibliographystyle{alphadin}
\usepackage{url}
\usepackage{hyperref}

\begin{document}

\paragraph{Aufgabe 3 \textnormal{(6 Punkte)}.}
Das Cox-Ross-Rubenstein (CRR)-Modell ist wie folgt spezifiziert.
Fixiere Zahlen $N\in\mathbb{N}$, $-1<a<b$ und $r\geq0$.
Der Wahrscheinlichkeitsraum $(\Omega,{\cal F},P)$ ist gegeben durch
\begin{align*}
  \Omega&=\{1+a,1+b\}^N\,,\\
  {\cal F}={\cal P}(\Omega)
\end{align*}
und ein Wahrscheinlichkeitsmaß $P$ auf $(\Omega,{\cal F})$, so dass $P(\{\omega\})>0$ für alle $\omega\in\Omega$.
Der Numéraire-Prozess $(\widetilde{S_n^0})_{n=0,\dots,N}$ ist gegeben durch
\[
  \widetilde{S}_n^0:=(1+r)^n
\]
und der Preis eines Finanzinstruments $(\widetilde{S}_n)_{n=0,\dots,N}$ ist definiert als $\widetilde{S}_0:=1$ und
\[
  \widetilde{S}_n(\omega):=\omega_1\cdots\omega_n\quad\text{für}n=1,\dots,N\,.
\]
Die Filtration ${\cal F}=({\cal F}_n)_{n=0,\dots,N}$ ist gegeben durch
\[
{\cal F}_n:=\sigma(\widetilde{S}_0,\dots,\widetilde{S}_n)\,.
\]
Das Ziel dieser Aufgabe ist es zu beweisen, dass das CRR-Modell arbitragefrei ist, genau dann wenn $r\in(a,b)$ ist und, dass es in diesem Fall sogar vollständig ist.
\begin{itemize}
\item [1.] Ein äquivalentes Wahrscheinlichkeitsmaß $Q\sim P$ wird als \emph{Martingalmaß} bezeichnet, wenn der Prozess
  \[\biggl(\frac{\widetilde{S}_n}{\widetilde{S}_n^0}\biggr)_{n=0,\dots,N}\]
  ein $Q$-Martingal ist.
  Wir führen die Renditen $(T_i)_{i=1,\dots,N}$ ein als
  \[
    T_i:=\frac{\widetilde{S}_i}{\widetilde{S}_{i-1}}\,.
  \]
  Zeigen Sie, dass ein äquivalentes Wahrscheinlichkeitsmaß $Q\sim P$ ein Martingalmaß ist, genau dann wenn
  \[
    E_Q[T_{i+1}|{\cal F}_i]=1+r\quad\text{für alle }i=0,\dots,N-1\,.
  \]
\end{itemize}

\noindent\emph{Lösung: }
Wir folgen dem Beweis von Theorem 5.39 aus \cite{foellmer2016}.
$Q$ ist genau dann ein Martingalmaß, wenn $(\widetilde{S}_n/\widetilde{S}_n^0)$ ein $Q$-Martingal ist, also genau dann, wenn $E_Q[\widetilde{S}_{i+1}/\widetilde{S}^0_{i+1}|{\cal F}_i]=\widetilde{S}_i/\widetilde{S}^0_i$ oder mit der Definition von $\widetilde{S}^0_n$ genau dann, wenn $E_Q[\widetilde{S}_{i+1}|{\cal F}_i]=(1+r)\widetilde{S}_i$.
Da $-1<a<b$ gilt $\widetilde{S}_i>0$ und wir können dadurch teilen, sodass wir erhalten, dass $E_Q[\widetilde{S}_{i+1}|{\cal F}_i]/\widetilde{S}_i=1+r$.
Außerdem ist $\widetilde{S}_n$ nach Definition von ${\cal F}$ ${\cal F}_n$-messbar.
Als Verkettung von messbaren Funktionen ist auch $1/\widetilde{S}_n$ ${\cal F}_n$-messbar.
Mit den Rechenregeln für die bedingte Erwartung können wir $1/\widetilde{S}_n$ ${\cal F}_n$ in die bedingte Erwartung reinziehen.
Mit der Definition der Renditen $T_i$ erhalten wir $E_Q[T_{i+1}|{\cal F}_i]=1+r$ und somit die zu zeigende Aussage.
\begin{itemize}
\item [3.] Angenommen, $Q\sim P$ ist ein äquivalentes Martingalmaß.
  Zeigen Sie, dass die Renditen $(T_i)_{i=1,\dots,N}$ unabhängig und identisch verteilt sind mit
  \[Q(T_1=1+a)=1-q\quad\text{und}\quad Q(T_1=1+b)=q\,,\]
  wobei $q\in(0,1)$ gegeben ist durch
  \[
    q=\frac{r-a}{b-a}\,.
  \]
\end{itemize}
Aus Teilaufgabe 1 folgt, dass $E_Q[T_{i+1}|{\cal F}_i]=1+r$, wobei $T_{i+1}\in\{1+a,1+b\}$, also $(1+a)(1-E_Q[\mathbbm{1}_{\{T_{i+1}=1+b\}}|{\cal F}_i])+(1+b)E_Q[\mathbbm{1}_{\{T_{i+1}=1+b\}}|{\cal F}_i]=1+r$.
Durch Einsetzen von $q=E_Q[\mathbbm{1}_{\{T_{i+1}=1+b\}}|{\cal F}_i]$ und auflösen nach $q$ ergibt sich $E_Q[\mathbbm{1}_{\{T_{i+1}=1+b\}}|{\cal F}_i]=q=\frac{r-a}{b-a}$.
Das heißt, allerdings, die $T_i$ sind unabhängig und identisch verteilt, \emph{was noch genauer gezeigt werden sollte}.
Für $i=0$ erhalten wir außerdem $q=Q(T_1=1+b)$.
\begin{itemize}
\item[2.] Angenommen, es existiert ein äquivalentes Martingalmaß $Q\sim P$.
  Zeigen Sie, dass $r\in(a,b)$ ist.
\end{itemize}
Da $Q\sim P$ und $P(\{\omega\})>0$ gilt $0<q<1$, also $0<\frac{r-a}{b-a}<1$ und somit $a<r<b$.
\begin{itemize}
\item [4.] Folgern Sie, dass ein äquivalentes Martingalmaß $Q\sim P$ eindeutig ist, sofern es existiert.
\end{itemize}
Gibt es ein anderes äquivalentes Martingalmaß $Q'\sim P$, so muss nach Teil\-auf\-ga\-be 3 auch hierfür gelten, dass $Q'(\omega_i=1+b)=q=1-Q'(\omega_i=1+a)$.
Da die Mengen $\{\omega_i\}$ unabhängig sind, muss für alle $\omega\in\Omega$ gelten, dass $Q'(\{\omega\})=\prod_{i=1}^NQ'(\{\omega_i\})=q^k(1-q)^{N-k}=Q(\{\omega\})$, wobei $k$ zählt, wie oft sich der Preis erhöht hat.
Somit gilt auf ganz $\Omega$, dass $Q'=Q$, also ist das Martingalmaß eindeutig.
\begin{itemize}
\item [5.] Nehmen Sie nun an, dass $r\in(a,b)$ ist.
  Zeigen Sie, dass ein äquivalentes Martingalmaß $Q\sim P$ existiert.
\end{itemize}
Sei $r\in(a,b)$ und $Q$ wie in Teilaufgabe 4, also $Q(\{\omega\})=q^k\cdot (1-q)^{N-k}$.
Wieder gilt für alle $i=1,\dots,N$, dass $Q(\{\omega_i=1+b\})=q$, sodass die Renditen $T_i$ unter $Q$ unabhängig und identisch verteilt sind.
Nach Teilaufgabe 2 ist $Q$ hierdurch ein Martingalmaß.
\bibliography{../../../books/wt}
\end{document}

%%% Local Variables:
%%% mode: latex
%%% ispell-local-dictionary: "german"
%%% TeX-master: t
%%% End:
