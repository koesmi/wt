\documentclass{article}
\usepackage[a4paper,margin=1.875in,top=1.875in,bottom=1.875in]{geometry}
\usepackage{amsmath,mathtools,bbm,amssymb,stmaryrd,wasysym}
\usepackage{mathrsfs}
\usepackage{babel}

\usepackage{setspace}
\doublespacing

\usepackage{fancyhdr}
\renewcommand{\headrulewidth}{0pt} 
\pagestyle{fancy}
\lhead{Blatt 3 Nicole und Evgenij}\rhead{Seite \thepage}
\fancyfoot{}

\usepackage{tikz}
\usetikzlibrary{decorations.pathreplacing,arrows}

\usepackage[numbers]{natbib}
\bibliographystyle{alphadin}
\usepackage{url}
\usepackage{hyperref}

\begin{document}
Lieber Moritz, es wäre super, wenn du deine Musterlösung für das Blatt direkt hochladen könntest.
Dann könnten wir sie uns vor der Übung anschauen und hätten es einfacher zu folgen \smiley
\paragraph{Aufgabe 1 \textnormal{(4 Punkte)}.}
Sei $(\Omega,\mathscr{A},\mathbb{F},P)$ ein filtrierter Wahrscheinlichkeitsraum und $T$ und $S$ Stoppzeiten.
Zeigen Sie
\begin{itemize}
\item [i)] $\mathscr{F}_T\cap\mathscr{F}_S=\mathscr{F}_{T\wedge S}$.
\end{itemize}
Sei zunächst $A\in\mathscr{F}_{T\wedge S}$.
Da $T\wedge S\leq S,T$ gilt nach Lemma 12 $A\in\mathscr{F}_T\cap\mathscr{F}_S$.
Sei nun $A\in\mathscr{F}_T\cap\mathscr{F}_S$.
Nach Definition von $\mathscr{F}_T$ und $\mathscr{F}_S$ gilt $A\cap\{S\leq t\}\in\mathscr{F}_t$ und $A\cap\{T\leq t\}\in\mathscr{F}_t$.
Da $\mathscr{F}_t$ eine $\sigma$-Algebra ist, gilt $A\cap\bigl(\{S\leq t\}\cup \{S\leq t\}\bigr)\in\mathscr{F}_t$.
Die Behauptung folgt, denn $\{S\leq t\}\cup \{S\leq t\}=\bigl\{S\leq t\text{ oder }T\leq t\bigr\}=\{S\wedge T\leq t\}$.
\begin{itemize}
\item [ii)] Für $Y\in L^1$ gilt $E[Y\mathbbm{1}_{\{S=T\}}|\mathscr{F}_S]=E[Y\mathbbm{1}_{\{S=T\}}|\mathscr{F}_T]$
\end{itemize}
Nach Definition des bedingten Erwartungswertes müssen wir zwei Sachen zeigen. Erstens, dass $E[Y\mathbbm{1}_{\{S=T\}}|\mathscr{F}_T]$ $\mathscr{F}_S$-messbar ist und zweitens, die definierende Eigenschaft des bedingten Erwartungswertes, die besagt, dass für alle $F_S\in\mathscr{F}_S$ gilt $E[\mathbbm{1}_{F_S}\bigl[Y\mathbbm{1}_{\{S=T\}}|\mathscr{F}_T]\bigr]=E[\mathbbm{1}_{F_S}Y\mathbbm{1}_{\{S=T\}}]$.
\emph{Die Messbarkeit ist noch zu zeigen.}
Um die definierende Eigenschaft zu zeigen betrachte ein $F_S\in\mathscr{F}_S$.
Nach Lemma 12.vi gilt $\mathscr{F}_S\cap\{S=T\}\in\mathscr{F}_T$.
Da jedes $Y$ einen bedingten Erwartungswertes bezüglich $\mathscr{F}_T$ hat erhalten wir $E[\mathbbm{1}_{F_S\cap\{S=T\}}Y]=E\bigl[\mathbbm{1}_{F_S\cap\{S=T\}}E[Y|\mathscr{F}_T]\bigr]=E\bigl[\mathbbm{1}_{F_S}E[Y\mathbbm{1}_{\{S=T\}}|\mathscr{F}_T]\bigr]$, denn $\{S=T\}\in\mathscr{F}_T$ nach Lemma 15.

\paragraph{Aufgabe 3 \textnormal{(4 Punkte)}.} Zeigen Sie das \emph{Optional Stopping Theorem}: Sei $T$ eine Stoppzeit und $X$ ein $\mathbb{F}$-Supermartingal.
Dann ist der gestoppte Prozess $X^T$ wieder ein Supermartingal bzgl. der Filtrationen $\mathbb{F}$ und $\mathbb{F}^T=(\mathscr{F}_{T\wedge t})_{t\in\mathbb{R}_+}$.

\noindent\emph{Hinweis: Dieses Resultat gilt analog für Submartingale und Martingale.}

Wir gehen anhand des Beweises von Theorem 18 in \cite{protter2005stochastic} vor.
Da $t\wedge T\leq T$ gilt mit dem Optional Sampling Theorem
\begin{align*}
  X_{t\wedge T}
  &\geq E[X_T|\mathscr{F}_{t\wedge T}]\,.
    \intertext{wir benutzen die Zerlegung $1=\mathbbm{1}_{\{T<t\}}+\mathbbm{1}_{\{T\geq t\}}$ und erhalten}
  &=E[X_T\mathbbm{1}_{\{T<t\}}+X_T\mathbbm{1}_{\{T\geq t\}}|\mathscr{F}_{t\wedge T}]\,.
    \intertext{Auf $\{T<t\}$ ist $t\wedge T=T$.
    Daher ist $X_T\mathbbm{1}_{\{T<t\}}$ $\mathscr{F}_{t\wedge T}$-messbar, wodurch}
  &=X_T\mathbbm{1}_{\{T<t\}}+E[X_T\mathbbm{1}_{\{T\geq t\}}|\mathscr{F}_{t\wedge T}]\,.
    \intertext{Nun gilt für ein $H\in\mathscr{F}_t$, dass $H\mathbbm{1}_{\{T\geq t\}}\in\mathscr{F}_T$, \emph{was noch genauer gezeigt werden sollte.}}
\end{align*}

\paragraph{\normalfont{\itshape Definition} 1.} Ein \emph{einfacher Prozess} $H$ (in Finanzmathe auch: \emph{einfache Handelsstrategie}) ist ein $\mathbb{R}^d$-wertiger adaptierter stochastischer Prozess der Form
\[
H=\sum_{i=1}^nh_i\mathbbm{1}_{\rrbracket\tau_{i-1},\tau_i\rrbracket}
\]
für endliche Stoppzeiten $0\leq\tau_0\leq\tau_1\leq\cdots\leq\tau_n<\infty$ und $h_i\in L^\infty(\mathscr{F}_{\tau_{i-1}})$ für alle $i=1,\dots,n$.
\paragraph{\normalfont{\itshape Definition} 2.} Sei $S$ ein $\mathbb{R}^d$-wertiger stochastischer Prozess und $H$ ein einfacher Prozess nach Definition 1.
Das \emph{stochastische Integral für einfache Prozesse $H\cdot S$ ist definiert durch}
\[
  (H\cdot S)_t:=\int_0^tH_sdS_s:=\sum_{i=1}^n\langle h_i,S_t^{\tau_i}-S_t^{\tau_{i-1}}\rangle_{\mathbb{R}^d}\,.
\]
\paragraph{Aufgabe 4 \textnormal{(4 Punkte)}.}
\begin{itemize}
\item [iii)] Zeigen Sie: Für ein Martingal $S$ und einen einfachen Prozess $H$ ist das stochastische Integral $H\cdot S$ auch ein Martingal.
\end{itemize}

\noindent\emph{Hinweis: Zeigen Sie  die Aussage iii) für $d=1$ und $H\cdot S=h(S^{T_2}-S^{T_1})$ für eine $\mathscr{F}_{T_1}$-messbare Zufallsvariable $h\in L^\infty$ und Stoppzeiten $T_2\geq T_1$.
Wieso genügt das?
Um die vereinfachte Aussage zu zeigen, ist die Fallunterscheidung $1=\mathbbm{1}_{\{T_1>s\}}+\mathbbm{1}_{\{T_1\leq s<T_2\}}+\mathbbm{1}_{\{T_2\leq s\}}$ sehr hilfreich.}
\bibliography{../../../books/wt}
\end{document}

%%% Local Variables:
%%% mode: latex
%%% ispell-local-dictionary: "german"
%%% TeX-master: t
%%% End:
