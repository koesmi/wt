\documentclass{article}
\usepackage[a4paper,margin=1.875in,top=1.575in,bottom=1.575in]{geometry}
\usepackage{amsmath,mathtools,bbm,amssymb,stmaryrd,wasysym}
\usepackage{mathrsfs}
\usepackage{babel}

\usepackage{setspace}
\doublespacing

\usepackage{fancyhdr}
\renewcommand{\headrulewidth}{0pt} 
\pagestyle{fancy}
\lhead{Blatt 5 Nicole und Evgenij}\rhead{Seite \thepage}
\fancyfoot{}

\usepackage{tikz}
\usetikzlibrary{decorations.pathreplacing,arrows}

\usepackage[numbers]{natbib}
\bibliographystyle{alphadin}
\usepackage{url}
\usepackage{hyperref}

\begin{document}
\paragraph{Aufgabe 4 \textnormal{(4 Punkte)}.}
\begin{itemize}
\item [i)] Beschreiben Sie, welche Voraussetzungen nachzurechnen sind, um die Wohldefiniertheit von Definition 2 zu gewährleisten (Sie müssen diese nicht zeigen!).
\end{itemize}
Sei $f$ eine rechtsstetige monoton wachsende Funktion.
Wir wollen Satz 2 auf $\mu_f$ auf $\{(s,t]_{s,t\in\mathbb{R}_+}\}$ anwenden.
Wir wissen schon aus Wahrschein\-lich\-keits\-the\-o\-rie 1, dass $\sigma(\{(s,t]\}_{s,t\in\mathbb{R}_+})=\mathscr{B}(\mathbb{R}_+)$.
Wenn $f$ monoton wachsend ist, ist das Bild von $\mu_f$ auch in $[0,\infty]$.
Dann ist noch zu überprüfen, ob $\mu_f$ eine additive, $\sigma$-subadditive, $\sigma$-endliche Mengenfunktion mit  $\mu(\emptyset)=0$ ist.
Dann wäre nach Satz 2 $\mu_f$ auf $(\mathbb{R}_+,\mathscr{B}(\mathbb{R})_+)$ eindeutig gegeben.
\begin{itemize}
\item [ii)] Sei $f$ zusätzlich differenzierbar.
  Zeigen Sie, dass für messbare Ab\-bil\-dun\-gen $h\colon\mathbb{R}_+\to\mathbb{R}$ gilt, dass
  \[
    \int_{\mathbb{R}_+}h(u)\mu_f(du)=\int_{\mathbb{R}_+}h(u)f'(u)du,
  \]
  falls eines der beiden Integrale existiert.
\end{itemize}
\emph{Hinweis: Verwenden Sie für ii) den Satz über monotone Klassen.
  Sie brauchen das Argument nicht im Detail auszuführen.}

\pagebreak
Wie wir in der Übung gesehen haben und auch im Hinweis steht, ver\-wen\-den wir das Moonotone-Klassen Theorem.
Sei $\mathscr{H}$ der Vektorraum aller Ab\-bil\-dun\-gen $h$, für die Formel gilt.
Wir wollen zeigen, dass die Formel für $h=\mathbbm{1}_{(a,b]}$ mit $a\leq b\in\mathbb{R}_+$ gilt.
Dann gilt sie nach dem Monotone-Klassen Theorem auch für alle $\mathscr{B}(\mathbb{R}_+)$-messbaren Abbildungen, denn $\sigma(\{(a,b]\}_{a,b\in\mathbb{R}_+})=\mathscr{B}(\mathbb{R}_+)$.
Nach der Definition des Lebesgue-Integrals gilt
\begin{align*}
  \int_{\mathbb{R}_+}\mathbbm{1}_{(a,b]}\mu_f(du)
  &=\mu_f\bigl((a,b]\bigr)\,.
    \intertext{Nach der Definition 2 von $\mu_f$ gilt}
  &=f(b)-f(a)\,.
    \intertext{Nach dem Hauptsatz der Differential- und Integralrechnung folgt}
  &=\int_{\mathbb{R}_+}\mathbbm{1}_{(a,b]}f'(u)du\,.
\end{align*}
Somit gilt die Formel für $h=\mathbbm{1}_{(a,b]}$ und mit dem Monotone-Klassen Theorem auch für alle $\mathscr{B}(\mathbb{R}_+)$-messbaren Funktionen $h$.
Damit das Monotone-Klassen Theorem angewendet werden kann, müsste noch gezeigt werden, dass die Formel für $h=1$ gilt.
Durch Einsetzen erhalten wir
\begin{align*}
  \int_{\mathbb{R}_+}\mu_f(du)
  &=\mu_f(\mathbb{R}_+)
    =\mu_f\Bigl(\bigcup_{n\in\mathbb{N}}[0,n)\Bigr)=\lim_{n\to\infty}\mu_f((0,n])\,,
    \intertext{mit Stetigkeit von unten.}
  &=\lim_{n\to\infty}f(n)-f(0)=\lim_{n\to\infty}\int_{0}^nf'(u)du=\int_{\mathbb{R}_+}f'(u)du\,,
\end{align*}
wieder mit dem Hauptsatz der Differential- und Integralrechnung.
  \emph{Außerdem wäre zu zeigen, dass sie für ein $h$ mit $h_n\uparrow h$ gilt, wenn sie für jedes $h_n$ gilt, was bestimmt mit monotoner Konvergenz geht.}
\pagebreak
\begin{itemize}
\item [iii)] Berechnen Sie $\int_{(0,t]}h(u)df(u):=\int_{\mathbb{R}_+}\mathbbm{1}_{(0,t]}h(u)\mu_f(du)$ für
  \begin{itemize}
  \item [a)] $h(t)=t$ und $f(t)=\exp(t)$.
  \end{itemize}
\end{itemize}
Nach Teilaufgabe (ii) gilt
\begin{align*}
  \int_{\mathbb{R}_+}\mathbbm{1}_{(0,t]}(u)u\mu_{\exp}(du)
  &=\int_{\mathbb{R}_+}\mathbbm{1}_{(0,t]}(u)u\exp(u)du=\int_0^tu\exp(u)du\\
  &=\Bigl.\exp(u)(u-1)\Bigr|_{u=0}^t=\exp(t)(t-1)+1\,.
\end{align*}
\bibliography{../../../books/wt}
\end{document}

%%% Local Variables:
%%% mode: latex
%%% ispell-local-dictionary: "german"
%%% TeX-master: t
%%% End:
