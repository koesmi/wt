\documentclass{article}
\usepackage[a4paper,margin=1.875in,top=1.5in,bottom=1.5in]{geometry}

\usepackage{amsmath,mathtools,bbm,amssymb}
\usepackage{mathrsfs}
\usepackage{babel}

\usepackage{setspace}
\doublespacing

\usepackage{fancyhdr}
\renewcommand{\headrulewidth}{0pt} 
\pagestyle{fancy}
\lhead{Blatt 1 Nicole und Evgenij}\rhead{Seite \thepage}
\fancyfoot{}

\usepackage{tikz}
\usetikzlibrary{decorations.pathreplacing,arrows}

\usepackage[numbers]{natbib}
\bibliographystyle{alphadin}
\usepackage{url}
\usepackage{hyperref}

\begin{document}
\paragraph{Aufgabe 1 \textnormal{(5 Punkte)}.}
\begin{itemize}
\item [i)] Sind $X$ und $Y$ zwei ununterscheidbare Prozesse, dann sind $X$ und $Y$ Modifikationen voneinander.
\end{itemize}
Zu zeigen ist nach Definition 4.iii, dass $P(X_t\neq Y_t)=0$ für alle $t\geq 0$.
Sei also $t\geq0$ beliebig gewählt.
Wenn $\{X_t\neq Y_t\}$ leer ist, ist die Behauptung schon gezeigt.
Sei andernfalls $\omega\in\{X_t\neq Y_t\}$.
Dann gibt es für $\omega$ ein $s\geq0$, sodass $X_s(\omega)\neq Y_s(\omega)$, nämlich $s=t$.
Somit ist $\omega\in \{\exists s\geq 0~X_s\neq Y_s\}$.
Das heißt, $\{X_t\neq Y_t\}\subseteq\{\exists s\geq 0~X_s\neq Y_s\}$.
Da $X$ und $Y$ ununterscheidbar sind, gilt für sie nach Definition 4.i und 4.ii, dass $P(\exists s\geq 0~X_s\neq Y_s)=0$.
Somit gilt auch $P(X_t\neq Y_t)=0$, was zu zeigen war.
\begin{itemize}
\item [ii)]
  Ist die Indexmenge $I$, in welcher die Prozesse indiziert sind, höchstens abzählbar, dann sind die Eigenschaften Ununterscheidbar und Modifikation äquivalent.
\end{itemize}
Nach der vorigen Teilaufgabe ist nur noch zu zeigen, dass eine Modifikation $X$ von $Y$ von $Y$ ununterscheidbar ist, wenn $I$ höchstens abzählbar ist.
Sei ohne Beschränkung der Allgemeinheit $I=\mathbb{N}$.
Es gilt
\begin{align*}
  P(\exists n\geq0~X_n\neq Y_n)
  &=P\Bigl(\bigcup\nolimits_{n\geq0}\{X_n\neq Y_n\}\Bigr)\,.
    \intertext{Mit $\sigma$-Subadditivität von $P$ haben wir}
  &\leq\sum\nolimits_{n\geq0}P(X_m\neq Y_n)=0\,,
\end{align*}
denn für alle $n\in\mathbb{N}$ gilt $P(X_n\neq Y_n)=0$, da $X$ eine Modifikation von $Y$ ist.
Somit sind $X$ und $Y$ ununterscheidbar.
\begin{itemize}
\item [iii)] Gilt $I=\mathbb{R}_+$ und sind $X$ und $Y$ rechtsstetige (linksstetige) Prozesse, dann sind die Eigenschaften Ununterscheidbar und Modifikation äquivalent.
\end{itemize}
Seien $X$ und $Y$ ohne Beschränkung der Allgemeinheit rechtsstetig, $X$ Modifikation von $Y$.
Wieder ist nur zu zeigen, dass sie ununterscheidbar sind.
Es gilt
\begin{align*}
  P(\exists t\geq 0 X_t\neq Y_t)
  &=P\Bigl(\bigcup\nolimits_{t\geq0}\{X_t\neq Y_t\}\Bigr)\,.
    \intertext{Da $X_t$ und $Y_t$ rechtsstetig sind und $\mathbb{Q}_+$ dicht in $\mathbb{R}_+$ liegt, gibt es für alle $t\geq0$ eine Folge $(q_n)$ in $\mathbb{Q}_+$, sodass $\lim_{n\to\infty}X_{q_n}-Y_{q_n}\neq0$, sodass}
  &=P\Bigl(\bigcup\nolimits_{t\in\mathbb{Q}_+}\{X_t\neq Y_t\}\Bigr)\,.
    \intertext{Da auch $\mathbb{Q}_+$ abzählbar ist, können wir mit dem gleichen Argument wie bei Teilaufgabe ii erkennen, dass}
  &=0\,.
\end{align*}
\begin{itemize}
\item [iv)] Sei $I=\mathbb{R}_+$ und die Prozesse $X$ und $Y$ gegeben durch $X_t=0$ und $Y_t=\mathbbm{1}_{Z=t}$ für alle $t\in\mathbb{R}_+$ und eine Zufallsvariable $Z\sim\operatorname{Exp}(\lambda)$.
  Welche Aussagen bezüglich Modifikation und Ununterscheidbarkeit lassen sich treffen? 
\end{itemize}
Wir wollen zunächst zeigen, dass $X$ eine Modifikation von $Y$ ist.
Sei hierfür $t\geq0$ beliebig.
Einsetzen der Definitionen von $X$ und $Y$ liefert $P(X_t\neq Y_t)=P(\mathbbm{1}_{Z=t}\neq0)=P(Z=t)=0$, denn $\{Z=t\}$ ist eine $P$-Nullmenge.
$Y$ und $Y$ sind allerdings unterscheidbar, denn $P(\exists s\geq0~Z=s)=P(\Omega)=1$, da $Z$ als Funktion auf ganz $\Omega$ definiert ist.
\paragraph{Aufgabe 2 \textnormal{(3 Punkte)}.}
\begin{itemize}
\item [i)] Die $\sigma$-Algebra der $T$-Vergangenheit, definiert durch
  \[
    \mathscr{F}_T:=\bigl\{A\in\mathscr{A}\mid A\cap\{T\leq t\}\in\mathscr{F}_t\text{ für alle }t\in\mathbb{R}_+\bigr\}
  \]
  ist eine $\sigma$-Algebra.
\end{itemize}
Da $\emptyset\cap\{T\leq t\}=\emptyset\in\mathscr{F}_t$ und $\Omega\cap\{T\leq t\}=\{T\leq t\}\in\mathscr{F}_t$, sind $\emptyset$ und $\Omega$ in $\mathscr{F}_T$.
Für ein $B\in\mathscr{F}_T$ ist $B^C\in\mathscr{A}$, denn $\mathscr{A}$ ist eine $\sigma$-Algebra.
Außerdem ist $B^C\cap\{T\leq t\}=\{T\leq t\}\setminus(B\cap\{T\leq t\})\in\mathscr{F}_t$, denn $\{T\leq t\}\in\mathscr{F}_t$, da $T$ eine Stoppzeit ist und $B\cap\{T\leq t\}$ nach Voraussetzung. Also gilt $B^C\in\mathscr{F}_T$.
Schließlich gilt für $A_1,A_2,\dots\in\mathscr{F}_T$, dass $\bigcup_{n\in\mathbb{N}}A_n\in\mathscr{A}$, denn $\mathscr{A}$ ist eine $\sigma$-Algebra.
Weiterhin ist $\bigcup_{n\in\mathbb{N}}A_n\cap\{T\leq t\}=\bigcup_{n\in\mathbb{N}}(A_n\cap\{T\leq t\})\in\mathscr{F}_t$, denn für alle $n\in\mathbb{N}$ ist $A_n\cap\{T\leq t\}\in\mathscr{F}_t$ nach Voraussetzung und die $\sigma$-Algebra $\mathscr{F}_t$ ist abgeschlossen unter abzählbaren Vereinigungen.
Insgesamt ist $\mathscr{F}_T$ also eine $\sigma$-Algebra.
\begin{itemize}
\item [ii)] 
  Für die Stoppzeit $T\equiv t$ stimmt diese mit $\mathscr{F}_t$ überein für alle $t\geq 0$.
\end{itemize}
Hier gilt $\mathscr{F}_T=\bigl\{A\in\mathscr{A}\mid A\cap\{t\leq t\}\in\mathscr{F}_t\bigr\}=\{A\in\mathscr{A}\mid A\in\mathscr{F}_t\}=\mathscr{F}_t$.
\begin{itemize}
\item [iii)] Sind $T$ und $S$ Stoppzeiten mit $T\leq S$, so gilt $\mathscr{F}_T\subset\mathscr{F}_S$.
\end{itemize}

Sei $A\in\mathscr{F}_T$, es gelte also für alle $t\geq0$, dass $A\cap\{T\leq t\}\in\mathscr{F}_t$.
Da $T\leq S$, gilt $\{S\leq t\}\subseteq\{T\leq t\}$.
Hierdurch ist $A\cap\{S\leq t\}=(A\cap\{T\leq t\})\cap\{S\leq t\}$.
Da $A\in \mathscr{F}_T$ ist $A\cap\{T\leq t\}\in\mathscr{F}_t$.
Da $S$ eine Stoppzeit ist, ist $\{S\leq t\}\in\mathscr{F}_t$.
Da $\mathscr{F}_t$ eine $\sigma$-Algebra ist, ist dann auch $(A\cap\{T\leq t\})\cap\{S\leq t\}\in\mathscr{F}_t$.
Somit ist $A\cap\{S\leq t\}\in\mathscr{F}_t$, also $A\in\mathscr{F}_S$.

\paragraph{Aufgabe 3 \textnormal{(4 Punkte)}.}
\begin{itemize}
\item [i)] Zeigen Sie: Ist $X$ ein adaptierter càdlàg Prozess, dann sind $X_-$ und $\Delta X$ ebenfalls adaptiert.
\end{itemize}
Zu zeigen ist, dass für alle $A\in\mathscr{A}'$ gilt $(\lim_{s\uparrow t}X_s)^{-1}(A)\in\mathscr{A}$, wobei $|\lim_{s\uparrow t}X_s|<\infty$.
\emph{Hier fehlt der Beweis.}
\bibliography{../../../books/wt}
\end{document}

%%% Local Variables:
%%% mode: latex
%%% ispell-local-dictionary: "german"
%%% TeX-master: t
%%% End:
