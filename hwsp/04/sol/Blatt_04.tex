\documentclass{article}
\usepackage[a4paper,margin=1.875in,top=1.875in,bottom=1.875in]{geometry}
\usepackage{amsmath,mathtools,bbm,amssymb,stmaryrd,wasysym}
\usepackage{mathrsfs}
\usepackage{babel}

\usepackage{setspace}
\doublespacing

\usepackage{fancyhdr}
\renewcommand{\headrulewidth}{0pt} 
\pagestyle{fancy}
\lhead{Blatt 4 Nicole und Evgenij}\rhead{Seite \thepage}
\fancyfoot{}

\usepackage{tikz}
\usetikzlibrary{decorations.pathreplacing,arrows}

\usepackage[numbers]{natbib}
\bibliographystyle{alphadin}
\usepackage{url}
\usepackage{hyperref}

\begin{document}
\paragraph{Aufgabe 2 \textnormal{(4 Punkte)}.}
Sei $X$ ein Submartingal.
Zeigen Sie die Äquivalenz folgender Aussagen
\begin{enumerate}
\item Es gilt $X^+$ ist gleichgradig integrierbar.
\item Es existiert eine Zufallsvariable $X_\infty\in L^1(\mathscr{F}_\infty)$, sodass $E[X_\infty|\mathscr{F}_t]\geq X_t$ für alle $t\in\mathbb{R}_+$.
\end{enumerate}
Wir gehen nach dem Beweis von Theorem 9.30 in \cite{Kallenberg2021}.
Sei zunächst $X^+$ gleichgradig integrierbar.
Nach Lemma 24 aus Wahrscheinlichkeitstheorie 1 gilt $\sup_{t\geq0}E[X_t^+]<\infty$.
Mit Satz 57, dem Doob'schen Grenzwertsatz, gibt es ein $X_\infty$, sodass $X_t^+\xrightarrow{\text{f.s.}}X_\infty^+$.
Nach dem Satz 60, dem Satz von Vitaly, gilt auch $X_t^+\xrightarrow{L^1}X_\infty^+$.
Mit majorisierter Konvergenz erhalten wir
\begin{equation}
  \label{eq:xinfplus}
  E[X_t^+|\mathscr{F}_s]\rightarrow E[X_\infty^+|\mathscr{F}_s]\,.
\end{equation}
Da $X$ ein Submartingal ist, können wir für alle $\geq 0\leq t\leq s$ schreiben $X_t\leq E[X_s|\mathscr{F}_t]$ und damit auch
\begin{align*}
  X_t
  &\leq\lim_{s\to\infty}E[X_s^+]-\liminf_{s\to\infty}E[X_s^-|\mathscr{F}_t]\,.
  \intertext{Mit Gleichung (\ref{eq:xinfplus}) im ersten Term und dem Lemma von Fatou im zweiten kriegen wir}
  &\leq E[X_\infty^+]-E[\liminf X_s^-|\mathscr{F}_t]=E[X_\infty|\mathscr{F}_t]\,.
\end{align*}

Gibt es andererseits ein $X_\infty$, sodass für alle $t\geq0$ gilt $X_t\leq E[X_\infty|\mathscr{F}_t]$, so gilt nach Blatt 2 Aufgabe 3.ii, dass $X_t^+\leq E[X_\infty^+|\mathscr{F}_t]$, denn $\cdot^+$ ist konvex.
Mit Korollar 8.22 aus \cite{klenke} erhalten wir schließlich, dass $X^+$ gleichgradig integrierbar ist.
\pagebreak
\paragraph{}
Formulieren Sie eine analoge Aussage für Supermartingale.

Sei $X$ ein Supermartingal, dann ist $-X$ ein Submartingal.
Somit gibt es genau dann ein $-X_\infty\in L^1(\mathscr{F}_\infty)$, sodass $E[-X_\infty|\mathscr{F}_t]\geq -X_t$ für alle $t\in\mathbb{R}_+$, wenn $(-X)^+$ gleichgradig integrierbar ist.
Das heißt, die folgenden Aussagen sind äquivalent.
\begin{enumerate}
\item $X^-$ ist gleichgradig integrierbar.
\item Es existiert eine Zufallsvariable $X_\infty\in L^1(\mathscr{F}_\infty)$, sodass $E[X_\infty|\mathscr{F}_t]\leq X_t$ für alle $t\in\mathbb{R}_+$.
\end{enumerate}

\bibliography{../../../books/wt}
\end{document}

%%% Local Variables:
%%% mode: latex
%%% ispell-local-dictionary: "german"
%%% TeX-master: t
%%% End:
