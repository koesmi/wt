\documentclass{article}
\usepackage[a4paper,margin=1.875in,top=1.875in,bottom=1.875in]{geometry}
\usepackage{amsmath,mathtools,bbm,amssymb,stmaryrd,wasysym}
\usepackage{mathrsfs}
\usepackage{babel}

\usepackage{setspace}
\doublespacing

\usepackage{fancyhdr}
\renewcommand{\headrulewidth}{0pt} 
\pagestyle{fancy}
\lhead{Blatt 4 Nicole und Evgenij}\rhead{Seite \thepage}
\fancyfoot{}

\usepackage{tikz}
\usetikzlibrary{decorations.pathreplacing,arrows}

\usepackage[numbers]{natbib}
\bibliographystyle{alphadin}
\usepackage{url}
\usepackage{hyperref}

\begin{document}
\paragraph{Aufgabe 2 \textnormal{(4 Punkte)}.}
Sei $X$ ein Submartingal.
Zeigen Sie die Äquivalenz folgender Aussagen
\begin{enumerate}
\item Es gilt $X^+$ ist gleichgradig integrierbar.
\item Es existiert eine Zufallsvariable $X_\infty\in L^1(\mathscr{F}_\infty)$, sodass $E[X_\infty|\mathscr{F}_t]\geq X_t$ für alle $t\in\mathbb{R}_+$.
\end{enumerate}
Wir gehen nach dem Beweis von Theorem 9.30 in \cite{Kallenberg2021}.
Sei zunächst $X^+$ gleichgradig integrierbar.
Nach Lemma 24 aus Wahrscheinlichkeitstheorie 1 gilt $\sup_{t\geq0}E[X_t^+]<\infty$.
Mit Satz 57, dem Doob'schen Grenzwertsatz, gibt es ein $X_\infty$, sodass $X_t^+\xrightarrow{\text{f.s.}}X_\infty^+$.
Nach dem Satz 60, dem Satz von Vitaly, gilt auch $X_t^+\xrightarrow{L^1}X_\infty^+$.
Sei nun $A\in\mathscr{F}_t$.
Dann kriegen wir für $t\leq s$
\begin{align*}
  E[(X^+_\infty-X^+_s)\mathbbm{1}_A]
  &=E[(X^+_\infty-X^+_t)\mathbbm{1}_A]+E[(X^+_t-X^+_s)\mathbbm{1}_A]\,.
    \intertext{Da $X$ ein Submartingal ist, ist der zweite Term negativ.
    Somit kriegen wir}
  &\leq E[(X^+_\infty-X^+_t)\mathbbm{1}_A]\leq E[|X^+_\infty-X^+_t|]\to0\,,
\end{align*}
sodass auch
\begin{equation}
  \label{eq:xinfplus}
  E[X_t^+|\mathscr{F}_s]\rightarrow E[X_\infty^+|\mathscr{F}_s]\,.
\end{equation}
Da $X$ ein Submartingal ist, können wir für alle $\geq 0\leq t\leq s$ schreiben $X_t\leq E[X_s|\mathscr{F}_t]$ und damit auch
\begin{align*}
  X_t
  &\leq\lim_{s\to\infty}E[X_s^+]-\liminf_{s\to\infty}E[X_s^-|\mathscr{F}_t]\,.
  \intertext{Mit Gleichung (\ref{eq:xinfplus}) im ersten Term und dem Lemma von Fatou im zweiten kriegen wir}
  &\leq E[X_\infty^+]-E[\liminf X_s^-|\mathscr{F}_t]=E[X_\infty|\mathscr{F}_t]\,.
\end{align*}

Gibt es andererseits ein $X_\infty$, sodass für alle $t\geq0$ gilt $X_t\leq E[X_\infty|\mathscr{F}_t]$, so gilt nach Blatt 2 Aufgabe 3.ii, dass $X_t^+\leq E[X_\infty^+|\mathscr{F}_t]$, denn $\cdot^+$ ist konvex.
Mit Korollar 8.22 aus \cite{klenke} erhalten wir schließlich, dass $X^+$ gleichgradig integrierbar ist.
\paragraph{}
Formulieren Sie eine analoge Aussage für Supermartingale.

Sei $X$ ein Supermartingal, dann ist $-X$ ein Submartingal.
Somit gibt es genau dann ein $-X_\infty\in L^1(\mathscr{F}_\infty)$, sodass $E[-X_\infty|\mathscr{F}_t]\geq -X_t$ für alle $t\in\mathbb{R}_+$, wenn $(-X)^+$ gleichgradig integrierbar ist.
Das heißt, die folgenden Aussagen sind äquivalent.
\begin{enumerate}
\item $X^-$ ist gleichgradig integrierbar.
\item Es existiert eine Zufallsvariable $X_\infty\in L^1(\mathscr{F}_\infty)$, sodass $E[X_\infty|\mathscr{F}_t]\leq X_t$ für alle $t\in\mathbb{R}_+$.
\end{enumerate}

\paragraph{Aufgabe 4 \textnormal{(4 Punkte)}.}
\begin{itemize}
\item [i)] Seien $Y_i$, $i=1,2,\dots$ unabhängig identisch verteilt mit $P(Y_i=0)=1-P(Y_i=1)=\frac{1}{2}$.
  Zeigen Sie, dass ${\cal X}=(X_n)_{n\geq1}$ mit $X_n:=2^n\prod_{i=1}^n Y_i$ ein Martingal ist bzgl. einer geeigneten Filtration.
\end{itemize}
Sei $\mathbb{F}=(\mathscr{F}_n)$ mit $\mathscr{F}_n=\sigma(Y_1,\dots,Y_n)$.
Dann sind $Y_1,\dots,Y_n$ $\mathscr{F}_n$-messbar, sodass gilt
\begin{align*}
  E[X_{n+1}|\mathscr{F}_n]
  &=E\Bigl[2^{n+1}\prod\nolimits_{i=1}^{n+1}Y_i\Bigm|\mathscr{F}_n\Bigr]=2^{n+1}\prod\nolimits_{i=1}^nY_i E[Y_{n+1}|\mathscr{F}_n]\,.
    \intertext{Da $Y_{n+1}$ unabhängig von $Y_1,\dots,Y_n$ und damit von $\mathscr{F}_n$ ist, kriegen wir}
  &=2^{n+1}\prod\nolimits_{i=1}^nY_i E[Y_{n+1}]=2^{n+1}\prod\nolimits_{i=1}^nY_i E[Y_1]\,,
    \intertext{Denn alle $Y_i$ sind ja identisch verteilt.
    Mit $E[Y_1]=0\cdot\frac{1}{2}+1\cdot\frac{1}{2}=\frac{1}{2}$ kriegen wir schließlich}
  &=2^n\prod\nolimits_{i=1}^nY_i=X_n\,,
\end{align*}
sodass ${\cal X}$ ein Martingal ist.
\begin{itemize}
\item [ii)] Sei $(Z_n)_{n\geq1}$ eine Folge unabhängiger Zufallsvariablen mit $P(Z_n=1)=\frac{1}{n}=1-P(Z_n=0)$.
  Zeigen Sie, dass die Folge in $L^1$ konvergiert, aber nicht fast sicher.
\end{itemize}
Es gilt $E[|Z_n-0|]=1\cdot\frac{1}{n}+0\cdot(1-\frac{1}{n})=\frac{1}{n}\rightarrow0$, sodass $Z_n\xrightarrow{L^1}0$.
Da die $Z_n$ unabhängig sind, sind auch die Mengen $\{Z_n=1\}$ unabhängig.
Außerdem gilt $\sum_{n\geq1}P(Z_n=1)=\sum_{n\geq1}\frac{1}{n}=\infty$.
Somit können wir Borel-Cantelli zwei anwenden.
Hiernach folgt, $1=P(\limsup\{Z_n=1\})=P(\forall n\geq 1\exists m\geq n~Z_m=1)$.
Angenommen, es gilt $1=P(\lim_{n\to\infty}Z_n=0)=P(\exists k\geq1\forall l\geq k~Z_l=0)$, dann würde aber auch für alle $m\geq k$ $P$-fast-sicher gelten, dass $Z_m=0$.
Somit kann $Z_n$ nicht fast sicher gegen 0 konvergieren und auch nicht gegen etwas anderes, denn $Z_n\xrightarrow{L^1}0$ impliziert $Z_n\xrightarrow{P}0$ und $Z_n\xrightarrow{\text{f.s.}}Z$ impliziert $Z_n\xrightarrow{P}Z$, sodass $Z=0$ sein muss.

\begin{itemize}
\item [iii)] Gibt es ein Martingal $X=(X_n)_{n\geq1}$, dass in $L^1$, aber nicht fast sicher konvergiert?
\end{itemize}
Nein, denn Konvergenz in $L^1$ impliziert für Martingale fast sicherere Kon\-ver\-genz.
Angenommen, $X_n\xrightarrow{L^1}X_\infty$, also insbesondere auch $X_n\xrightarrow{P}X_\infty$.
Dann gilt $\sup_n E[|X_n|]<\infty$, also auch $\sup_n E[X_n^+]<\infty$.
Somit können wir den Doobschen Konvergenzsatz, Satz 57, anwenden.
Demnach gibt es ein $X_\infty'$, sodass $X_n\xrightarrow{\text{f.s.}}X_\infty'$.
Da hierdurch auch gilt $X_n\xrightarrow{P}X_\infty'$, muss folgen $X_\infty=X_\infty'$.
\bibliography{../../../books/wt}
\end{document}

%%% Local Variables:
%%% mode: latex
%%% ispell-local-dictionary: "german"
%%% TeX-master: t
%%% End:
