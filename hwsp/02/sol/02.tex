\documentclass{article}
\usepackage[a4paper,margin=1.875in,top=1.875in,bottom=1.875in]{geometry}

\usepackage{amsmath,mathtools,bbm,amssymb}
\usepackage{mathrsfs}
\usepackage{babel}

\usepackage{setspace}
\doublespacing

\usepackage{fancyhdr}
\renewcommand{\headrulewidth}{0pt} 
\pagestyle{fancy}
\lhead{Blatt 2 Nicole und Evgenij}\rhead{Seite \thepage}
\fancyfoot{}

\usepackage{tikz}
\usetikzlibrary{decorations.pathreplacing,arrows}

\usepackage[numbers]{natbib}
\bibliographystyle{alphadin}
\usepackage{url}
\usepackage{hyperref}

\begin{document}
\paragraph{Aufgabe 1 \textnormal{(4 Punkte)}.}
\begin{itemize}
\item [i)] Sei $W$ eine Brownsche Bewegung.
  Dann ist $W$ ein Martingal.
\end{itemize}
$W$ ist adaptiert und stetig, also insbesondere càdlàg.
Weiterhin gilt $0=E[|X_t|]<\infty$.
Sei schließlich $0\leq s\leq t$.
Alle $F\in{\cal F}_s$ sind unabhängig von den Zuwächsen $X_t-X_s$.
Somit gilt $E[\mathbbm{1}_F(X_t-X_s)]=P(F)E[X_t-X_s]=0$, denn $E[X_t]=E[X_s]=0$.
Somit ist $W$ ein Martingal.
\begin{itemize}
\item [ii)] Sei $N$ ein Poisson-Prozess mit Intensität $\Lambda(t)=\mathbb{E}[N_t]$.
  Dann ist der kompensierte Poisson-Prozess $N_t-\Lambda(t)_t$ ein Martingal.
\end{itemize}
\bibliography{../../../books/wt}
\end{document}

%%% Local Variables:
%%% mode: latex
%%% ispell-local-dictionary: "german"
%%% TeX-master: t
%%% End:
