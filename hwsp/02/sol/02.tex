\documentclass{article}
\usepackage[a4paper,margin=1.875in,top=1.875in,bottom=1.875in]{geometry}

\usepackage{amsmath,mathtools,bbm,amssymb}
\usepackage{mathrsfs}
\usepackage{babel}

\usepackage{setspace}
\doublespacing

\usepackage{fancyhdr}
\renewcommand{\headrulewidth}{0pt} 
\pagestyle{fancy}
\lhead{Blatt 2 Nicole und Evgenij}\rhead{Seite \thepage}
\fancyfoot{}

\usepackage{tikz}
\usetikzlibrary{decorations.pathreplacing,arrows}

\usepackage[numbers]{natbib}
\bibliographystyle{alphadin}
\usepackage{url}
\usepackage{hyperref}

\begin{document}
\paragraph{Aufgabe 1 \textnormal{(4 Punkte)}.}
\begin{itemize}
\item [i)] Sei $W$ eine Brownsche Bewegung.
  Dann ist $W$ ein Martingal.
\end{itemize}
$W$ ist adaptiert und stetig, also insbesondere càdlàg.
Weiterhin gilt $0=E[|X_t|]<\infty$.
Sei schließlich $0\leq s\leq t$.
Alle $F\in{\cal F}_s$ sind unabhängig von den Zuwächsen $X_t-X_s$.
Somit gilt $E[\mathbbm{1}_F(X_t-X_s)]=P(F)E[X_t-X_s]=0$, denn $E[X_t]=E[X_s]=0$.
Somit ist $W$ ein Martingal.
\begin{itemize}
\item [ii)] Sei $N$ ein Poisson-Prozess mit Intensität $\Lambda(t)=\mathbb{E}[N_t]$.
  Dann ist der kompensierte Poisson-Prozess $N_t-\Lambda(t)_t$ ein Martingal.
\end{itemize}
Ein Poisson-Prozess ist càdlàg.
Da für den kompensierten Poisson-Prozess wie für eine Brownsche Bewegung für alle $t\geq0$ gilt $E[N_t-\Lambda(t)]=E[N_t]-E[N_t]=0$, ist die Argumentation sonst analog zu der von Teilaufgabe i.
\pagebreak
\paragraph{Aufgabe 2 \textnormal{(4 Punkte)}.}
Zeigen Sie, dass für alle quadratintegrierbaren Martingale $M$, dh. $E[M_t^2]<\infty$ für alle $t\in\mathbb{R}_+$, und $s\leq t$ folgende Aussagen gelten:
\begin{itemize}
\item [i)] $E[(M_t-M_s)^2\mid{\cal F}_s]=E[M_t^2-M_s^2\mid{\cal F}_s]$.
\end{itemize}
Durch Ausquadrieren erhalten wir
\begin{align*}
  E[(M_t-M_s)^2]\mid{\cal F}_s]
  &=E[M_t^2-2M_tM_s+M_s^2\mid{\cal F}_s]\,.
    \intertext{Da $M$ ein Martingal ist, ist es adaptiert.
    Damit ist $M_s$ ist ${\cal F}_s$ messbar und wir können es aus der bedingten Erwartung rausziehen.
    Zudem gilt dafür $E[M_s\mid {\cal F}_s]=M_s$, sodass}
  &=E[M_t^2\mid{\cal F}_s]-2M_sE[M_t\mid{\cal F}_s]+M_s^2\,.
    \intertext{Mit der Martingaleigenschaft folgt}
  &=E[M_t^2\mid{\cal F}_s]-2M_s^2+M_s^2\,.
    \intertext{Zusammenfassen der letzten beiden Terme liefert}
  &=E[M_t^2\mid{\cal F}_s]-M_s^2\,.    
    \intertext{wieder aufgrund der ${\cal F}_s$-Messbarkeit von $M_s$ erhalten wir}
  &=E[M_t^2-M_s^2\mid{\cal F}_s].
\end{align*}
\begin{itemize}
\item [ii)] $E[(M_t-M_s)^2]=E[M_t^2]-E[M_s^2]$.
\end{itemize}
Mit der definierenden Eigenschaft (ii) vom bedingten Erwartungswert ausgewertet auf $\Omega$ können wir schreiben
\begin{align*}
  E[(M_t-M_s)^2]
  &=E\bigl[E[(M_t-M_s)^2\mid{\cal F}_s]\bigr]
    \intertext{Einsetzten von Teilaufgabe (i) liefert}
  &=E\bigl[E[M_t^2-M_s^2\mid{\cal F}_s]\bigr]
    \intertext{und wieder die Eigenschaft (ii) auf $\Omega$ schließlich}
  &=E[M_t^2-M_s^2]\,.
\end{align*}

\paragraph{Aufgabe 3 \textnormal{(4 Punkte)}.}
Zeigen Sie folgende Aussagen:
\begin{itemize}
\item [i)] Jedes nicht-negative lokale Martingal ist ein Supermartingal.
\end{itemize}
Sei $X$ ein nicht-negatives lokales Martingal, $(T_n)$ die dazugehörige lokalisierende Folge.
Da $(T_n)$ fast sicher gegen unendlich konvergiert, konvergiert für jedes $t\geq0$ die Folge $(X^{T_n}_t)_{n\in\mathbb{N}}$ fast sicher gegen $X_t$.
Somit gilt
\begin{align*}
  E[X_t|{\cal F}_s]
  &=E[\liminf_{n\to\infty}X_t^{T_n}\mid{\cal F}_s]\,.
    \intertext{Es gilt $X_t\geq0$ für alle $t\geq0$.
    Somit können wir das Lemma von Fatou für den bedingten Erwartungswert anwenden und erhalten}
  &\leq\liminf_{n\to\infty}E[X_t^{T_n}\mid{\cal F}_s]\,.
    \intertext{Da $X$ ein lokales Martingal ist, ist $(X_t^{T_n})_{t\geq0}$ ein Martingal.
    Somit kriegen wir}
  &=\liminf_{n\to\infty}X_s^{T_n}=X_s\,,
\end{align*}
wieder weil $(T_n)$ fast sicher gegen unendlich konvergiert.
Durch eine analoge Argumentation für $E[|X_t|]$ erhalten wir zudem die Integrabilität von $X_t$, sodass $X$ ein Supermartingal ist.
\bibliography{../../../books/wt}
\end{document}

%%% Local Variables:
%%% mode: latex
%%% ispell-local-dictionary: "german"
%%% TeX-master: t
%%% End:
