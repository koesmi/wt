\documentclass{article}
\usepackage[a4paper,margin=1.875in,top=1.875in,bottom=1.875in]{geometry}
\usepackage{amsmath,mathtools,bbm,amssymb,stmaryrd,wasysym}
\usepackage{mathrsfs}
\usepackage{babel}

\usepackage{setspace}
\doublespacing

\usepackage{fancyhdr}
\renewcommand{\headrulewidth}{0pt} 
\pagestyle{fancy}
\lhead{Blatt 13 Nicole und Evgenij}\rhead{Seite \thepage}
\fancyfoot{}

\usepackage{tikz}
\usetikzlibrary{decorations.pathreplacing,arrows}

\usepackage[numbers]{natbib}
\bibliographystyle{alphadin}
\usepackage{url}
\usepackage{hyperref}

\begin{document}

\paragraph{}

\paragraph{\normalfont{{\itshape Definition} 1 ($T$-Forward-Measure).}}
Sei $(B_t)_{t\leq T}$, $B_t=e^{\int_0^tr_sds}$ der Bankkonto/Numéraire in einem Finanzmarkt.
Wenn $\mathbb{Q}$ ein risikoneutrales Maß ist, dann ist das Forward measure $\mathbb{Q}^T$ auf $\mathscr{F}_T$ definiert durch den Radon-Nikodym-Dichteprozess $Z$ bezüglich $\mathbb{Q}$, gegeben durch
\[
  Z_t=\frac{P_t(T)}{P_0(T)B_t}\,.
\]
\paragraph{\normalfont{{\itshape Definition} 2 (Zero-Coupon Bond).}}
Der Prozess $(P_t(T))_{t\geq T}$ bezeichne den Preis einer Geldeinheit bei $T$ am Zeitpunkt $t\leq T$.
Dieser wird zero-coupon bond (Nullkuponanleihe) genannt.

% \paragraph{Aufgabe 1 \textnormal{(4 Punkte)}.}
%Sei $F_t(T,S)$ die einfache Forward Rate für $[T,S]$ zum Zeitpunkt $t$ gegeben durch
%\[
%F_t(T,S)=\frac{1}{S-T}\left(\frac{P_t(T)}{P_t(S)}-1\right),\quad t\in[0,T]\,.
%\]
%Erklären Sie die Forward Rate in eigenen Worten.
%Zeigen Sie, dass $(F_t(T,S))_{t\in[0,T]}$ ein Martingal bezüglich $Q^S$ ist; das heißt
%\[
%  F_t(T,S)=E_{Q^S}[F_T(T,S)|\mathscr{F}_t]\quad\text{für alle }t\in[0,T]\,.
%\]
%\textsc{Hinweis.} Verwenden Sie die Identität
%\[
%  P_t(T)=B_tE_Q\left[\frac{1}{B_T}\middle|\mathscr{F}_t\right],\quad t\in[0,T]\,.
%\]

%Wir kamen auf keinen grünen Zweig und folgen deshalb dem Beweis von Proposition 2.5.1. in \cite{brigo2006interest}.
%Multiplizieren wir die Forward Rate mit $P_t(S)$, erhalten wir $F_{t(T,S)}=\frac{1}{S-T}\bigl(P_t(T)-P_t(S)\bigr)$.
%Laut Hinweis gilt $P_t(T)=B_tE_Q\bigl(\frac{1}{B_T}\big|\mathscr{F}_t\bigr)=E_Q\bigl(e^{-\int_t^Tr_sds}\big|\mathscr{F}_t\bigr)$.
%Wir leiten nach $T$ ab und erhalten
%\begin{align*}
%  -\frac{\partial P_t(T)}{\partial T}
%  &=E_Q[e^{-\int_t^Tr_sds}r_T\big|\mathscr{F}_t\bigr].
%\intertext{Nun wechseln wir zum Forward-Measure.
%    Nach Definition 1 ist $\frac{d Q}{d Q^T}=Z_t=\frac{P_t(T)}{e^{-\int_t^Tr_sds}}$, sodass}
%  &=E_{Q^T}\biggl(e^{-\int_t^Tr_sds}r_T\frac{P_t(T)}{e^{-\int_t^Tr_sds}}\bigg|\mathscr{F}_t\biggr)\,.\,.
%  &=P_t(T)E_{Q^T}[r_T|\mathscr{F}_t]\,.
     %  \end{align*}
\paragraph{Aufgabe 2 \textnormal{(12 Punkte)}.}
Betrachten Sie das Modell für die Instantaneous Forward Rate
\[
  f_t(T)=f_0(T)+\int_0^t\alpha_s(T)ds+\int_0^t\sigma_s(T)dW_s
\]
mit einem Standard-$Q$-Wienerprozess $W$.
Hierbei gilt folgende Gleichheit:
\[
  P_t(T)=\operatorname{exp}\left({-}\int_t^Tf(t,s)ds\right)\,.
\]
Wir fixieren $T$ und $S$.
\begin{itemize}
\item [1.] Zeigen Sie, dass der Dichteprozess von $Q^S$ bezüglich $Q^T$ gegeben ist durch
\[
  \frac{P_0(T)}{P_0(S)}\frac{P(S)}{P(T)}\,.
\]
\end{itemize}

Nach Definition 1 gilt
\begin{align*}
  \frac{d Q^S}{d Q^T}=\frac{d Q^S}{d Q}\frac{dQ}{dQ^T}
  =\frac{P(S)}{P_0(S)B_t}\frac{P_0(T)B_t}{P_t(T)}=\frac{P_0(T)}{P_0(S)}\frac{P(S)}{P(T)}\,.
\end{align*}


%\bibliography{../../../books/wt}
\end{document}

%%% Local Variables:
%%% mode: latex
%%% ispell-local-dictionary: "german"
%%% TeX-master: t
%%% End:
