\documentclass{article}
\usepackage[a4paper,margin=1.875in,top=1.575in,bottom=1.575in]{geometry}
\usepackage{amsmath,mathtools,bbm,amssymb,stmaryrd,wasysym}
\usepackage{mathrsfs}
\usepackage{babel}

\usepackage{setspace}
\doublespacing

\usepackage{fancyhdr}
\renewcommand{\headrulewidth}{0pt} 
\pagestyle{fancy}
\lhead{Blatt 6 Nicole und Evgenij}\rhead{Seite \thepage}
\fancyfoot{}

\usepackage{tikz}
\usetikzlibrary{decorations.pathreplacing,arrows}

\usepackage[numbers]{natbib}
\bibliographystyle{alphadin}
\usepackage{url}
\usepackage{hyperref}

\begin{document}

\paragraph{Aufgabe 2 \textnormal{(4 Punkte)}.}
Zeigen Sie $\mathscr{M}_{\text{loc}}\cap\mathcal{V}\subset\mathscr{A}_{\text{loc}}$

\noindent\emph{Hinweis: Betrachten Sie für $A\in\mathscr{M}\cap\mathcal{V}$ die Stoppzeitenfolge $T_n:=\inf\{t\in\mathbb{R}_+\mid\operatorname{Var}(A)>n\}$}

Das ist Lemma 3.11 in \cite{jacod2013limit}.
Es reicht zu zeigen, dass für jedes $A\in\mathscr{M}\cap{\cal V}$ gilt $A\in\mathscr{A}_{\text{loc}}$.
\emph{Hierfür fehlt die Begründung}.
Entsprechend Hinweis wählen wir $T_n:=\inf\{t\in\mathbb{R}_+\mid\operatorname{Var}(A)_t>n\}$.
Damit gilt $T_n\uparrow\infty$.
Für jedes $t\in\mathbb{R}_+$ gilt nach Aufgabe 1.1, dass $|A_{t^-}|\leq\operatorname{Var}(A)_{t^-}$.
Weiterhin gilt nach Aufgabe 1.2 und 1.3, dass $\Delta[\operatorname{Var}(A)]_t=|\Delta A_t|\leq|A_{t^-}|+|A_t|$.
Das heißt, $\operatorname{Var}(A)_{T_n}\leq 2n+|A_{T_n}|$.
\emph{Hier fehlt ebenfalls die Begründung.}
Es gilt $A\in\mathscr{M}$.
Nach dem Doobschen Grenzwertsatz gilt $E[|A_{T_n}|]<\infty$.
Mit der gefundenen Abschätzung gilt auch $E[|\operatorname{Var}(A)_{T_n}|]<\infty$, also $\operatorname{Var}(A)\in\mathscr{A}_{\text{loc}}^+$.
Das heißt $A\in\mathscr{A}_{\text{loc}}$.

\paragraph{Aufgabe 3 \textnormal{(4 Punkte)}.}
Zeigen Sie: Jedes $A\in\mathscr{A}^+$ ist ein Submartingal der Klasse (D).

Sei $A\in\mathscr{A}^+$.
Da $A$ wachsend ist, gilt $A_s=E[A_s|\mathscr{F}_s]\leq A_t$.
Aufgrund der Monotonie der bedingten Erwartung ist $A$ ein Submartingal.
Da $A$ gleichgradig integrierbar ist, ist $A$ nach Satz 65 der Klasse (D).

\paragraph{Aufgabe 4 \textnormal{(4 Punkte)}.}
Zeigen Sie die Eindeutigkeit in Satz 104.

Wir sollen zeigen, dass für $M,N\in\mathscr{H}^2_{\text{loc}}$ der vorhersehbare Prozess $\langle M,N\rangle\in\mathscr{V}$, so dass $MN-\langle M,N\rangle\in\mathscr{M}_{\text{loc}}$ eindeutig ist.
Sei hierfür $\langle M,N\rangle'\in\mathscr{V}$ ein anderer vorhersehbarer Prozess, so dass $MN-\langle M,N\rangle\in\mathscr{M}_{\text{loc}}$.
Dann ist $\langle M,N\rangle-\langle M,N\rangle'=MN-\langle M,N\rangle-(MN-\langle M,N\rangle')\in\mathscr{V}$ ein vorhersehbares lokales Martingal.
Nach Satz 98 gilt $\langle M,N\rangle-\langle M,N\rangle'=0$, zumindest bis auf eine verschwindende Menge.
\bibliography{../../../books/wt}
\end{document}

%%% Local Variables:
%%% mode: latex
%%% ispell-local-dictionary: "german"
%%% TeX-master: t
%%% End:
