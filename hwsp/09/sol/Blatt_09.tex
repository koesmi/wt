\documentclass{article}
\usepackage[a4paper,margin=1.875in,top=1.375in,bottom=1.375in]{geometry}
\usepackage{amsmath,mathtools,bbm,amssymb,stmaryrd,wasysym}
\usepackage{mathrsfs}
\usepackage{babel}

\usepackage{setspace}
\doublespacing

\usepackage{fancyhdr}
\renewcommand{\headrulewidth}{0pt} 
\pagestyle{fancy}
\lhead{Blatt 8 Nicole und Evgenij}\rhead{Seite \thepage}
\fancyfoot{}

\usepackage{tikz}
\usetikzlibrary{decorations.pathreplacing,arrows}

\usepackage[numbers]{natbib}
\bibliographystyle{alphadin}
\usepackage{url}
\usepackage{hyperref}

\begin{document}

\paragraph{Aufgabe 1 \textnormal{(4 Punkte)}.}
Zeigen Sie die Polarisierungsformel $[X,Y]=\frac{1}{4}([X+Y]-[X-Y])$.

Es gilt
\begin{align*}
  [X+Y]-[X-Y]
  &=(X+Y)^2-(X_0+Y_0)^2-2(X_-+Y_-)\cdot(X+Y)
  \\&\quad-(X-Y)^2+(X_0-Y_0)^2+2(X_--Y_-)\cdot(X-Y)\\
  &=4XY-4X_0Y_0-4X_-\cdot Y-4Y_-\cdot X=4[X,Y]\,.
\end{align*}

\paragraph{Aufgabe 3 \textnormal{(1 Punkt)}.}
Sei $X\in\mathscr{S}$ ein reellwertiger Prozess mit stetig differenzierbaren Pfaden.
Wenden Sie die Itô-Formel auf $f(X)$ mit $f=\operatorname{id}$ an.
Was fällt Ihnen auf?

Es gilt $D_i X=e_i$ und $D_{ij}X=0$, sodass $X=X_0+\sum_{i\leq d}e_i\cdot X^i+\sum_{s\leq t}\Bigl(X_s-X_{s-}-\sum_{i\leq d}e_i\Delta X_s^i\Bigr)$.

\paragraph{\normalfont{\itshape Definition} 1 (Die Differentielle Schreibweise).}
Wir vereinbaren folgende Notation, unter der Bedingung, dass alle Ausdrücke wohldefiniert sind.
Wir schreiben
\[
dX_t=H_tdt+K_tdY_t\,,
\]
falls folgende Integraldarstellung gilt
\[
X_t=X_o+\int_{0}^tH_sds+\int_0^tK_sdY_s\,.
\]

\paragraph{Aufgabe 5 \textnormal{(4 Punkte)}.}
Sei $W$ eine Standard Brown'sche Bewegung.
Zeigen Sie, dass der Prozess
\[
B_t=(1-t)\int_0^t\frac{1}{1-s}dW_s
\]
folgende Gleichung erfüllt
\[
dB_t=-\frac{B_t}{1-t}dt+dW_t
\]
und zeigen Sie, dass $\lim_{t\nearrow 1}E[B_t]=0$.

Da $\frac{B_t}{1-t}=\int_0^t\frac{1}{1-s}dW_s$ gilt $d\Bigl(\frac{B_t}{1-t}\Bigr)=K_tdW_t$ mit $K_t=\frac{1}{1-t}$, wobei
\begin{align*}
  d\Bigl(\frac{B_t}{1-t}\Bigr)
  &=\frac{dB_t}{1-t}+B_td\frac{1}{1-t}
    \intertext{und mit Kettenregel}
  &=\frac{d B_t}{1-t}+B_t\frac{dt}{(1-t)^2}=\frac{1}{1-t}dW_t\,.
\end{align*}
Multiplikation mit $1-t$ liefert die Behauptung.

\paragraph{Aufgabe 6 \textnormal{(4 Punkte)}.}
Es seien $F$ und $G$ stetige Funktionen und $f$ die Lösung der gewöhnlichen Differentialgleichung $f'(t)=F(t)f(t)$ mit $f(0)=1$.
Ferner sei $W$ eine Brown'sche Bewegung.
Zeigen Sie, dass der Prozess $X$ gegeben durch
\[
X_t:=f(t)\Bigl(x+\int_0^t f(s)^{-1}G(s)dW_s\Bigr)
\]
folgende Darstellung besitzt
\[
dX_t=F(t)X_tdt+G(t)dW_t,\quad X_0=x\,.
\]

$X_0=x$ folgt mit Definition 1.
Es gilt $\frac{X_t}{f(t)}=x+\int_0^tK_sdW_s$ mit $K_s=f(s)^{-1}G(s)$.
Wie in Aufgabe 5 rechnen mittels Einsetzen in die differentielle Schreibweise, dass
  $K_tdW_t
  =\frac{G(t)}{f(t)}dW_t
  =d\frac{X_t}{f(t)}
  =\frac{dX_t}{f(t)}-\frac{X_t}{f(t)^2}f'(t)dt$.
Einsetzen der Differentialgleichung für $f$, multiplizieren mit $f$ und Umstellen ergibt $dX_t=F(t)X_tdt+G(t)dW_t$.
\bibliography{../../../books/wt}
\end{document}

%%% Local Variables:
%%% mode: latex
%%% ispell-local-dictionary: "german"
%%% TeX-master: t
%%% End:
