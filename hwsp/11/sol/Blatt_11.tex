\documentclass{article}
\usepackage[a4paper,margin=1.875in,top=1.875in,bottom=1.875in]{geometry}
\usepackage{amsmath,mathtools,bbm,amssymb,stmaryrd,wasysym}
\usepackage{mathrsfs}
\usepackage{babel}

\usepackage{setspace}
\doublespacing

\usepackage{fancyhdr}
\renewcommand{\headrulewidth}{0pt} 
\pagestyle{fancy}
\lhead{Blatt 11 Nicole und Evgenij}\rhead{Seite \thepage}
\fancyfoot{}

\usepackage{tikz}
\usetikzlibrary{decorations.pathreplacing,arrows}

\usepackage[numbers]{natbib}
\bibliographystyle{alphadin}
\usepackage{url}
\usepackage{hyperref}

\begin{document}

\paragraph{Aufgabe 3 \textnormal{(Binomiales Modell; 6 Punkte)}.}
Sei $(\Omega,\mathscr{F},P)$ ein Wahr\-schein\-lich\-keits\-raum, so dass $\Omega=\{\omega_u,\omega_d\}$, $\mathscr{F}=2^\Omega$ und $P[\omega_u]=p=1-P[\omega_d]$.
Die gehandelten Vermögenswerte werden durch den zweidimensionalen Pro\-zess $S=(S^0,S^1)$ gegeben, mit
\begin{equation}
  \label{eq:a3s}
  S_0^0\equiv1,\quad S^0_1\equiv1+r,\quad S^1_0\equiv s_0,\quad S^1_1(\omega_u)=s_0(1+u),\quad S_1^1(\omega_d)=s_0(1+d).
\end{equation}
\begin{itemize}
\item [i)] Zeige, dass es ein Maß $Q\sim P$ auf $(\Omega,\mathscr{F})$ gibt, so dass der dis\-kon\-tier\-te Preisprozess ein Martingal ist.
  Ist dieses Marktmodell arbitragefrei?
\end{itemize}

Wir folgen der Argumentation von Beispiel 3.3.1 in \cite{Delbaen2006}.
Der dis\-kon\-tier\-te Prozess $X_t=(1,S^1_t/S^0_t)$ ergibt sich zu
\[
  X^0_0\equiv 1,\quad X^0_1\equiv 1,\quad X^1_0\equiv s_0,\quad X^1_1(\omega_u)=s_0(1+\tilde{u}),\quad X^1_1=s_0(1+\tilde{d})\,,
\]
mit $1+\tilde{u}=\frac{1+u}{1+r}$ und $1+\tilde{d}=\frac{1+d}{1+r}$.
Damit $X$ ein Martingal unter $Q$ ist, muss gelten $E_Q[X^i_1]=X^i_0$, also $Q[\omega_u]+Q[\omega_d]=1$ und $s_0(1+\tilde{u})Q[\omega_u]+s_0(1+\tilde{d})Q[\omega_d]=s_0$.
Schreiben wir $Q[\omega_u]=q$ gilt nach der ersten Gleichung $Q[\omega_d]=1-q$.
Wenn wir das in die zweite Gleichung einsetzen ergibt sich $(1+\tilde{u})q+(1+\tilde{d})(1-q)=1$, sodass $Q[\omega_u]=q=\frac{\tilde{d}}{\tilde{d}-u}=\frac{r-d}{u-d}$ und $Q[\omega_d]=1-q=\frac{\tilde{u}}{\tilde{u}-\tilde{d}}=\frac{u-r}{u-d}$.
Da $Q(A)=0$ nur für $A=\emptyset$ gilt ist $Q\sim P$, sodass $Q$ ein äquivalentes Martingalmaß ist.
Nach dem Fundamental Theorem of Asset Pricing ist dieses Marktmodell arbitragefrei.
\pagebreak
\begin{itemize}
\item [ii)] Angenommen, $p=0{,}7$, $s_0=100$, $r=0$, $1+d=0{,}8$ und $1+u=1{,}2$.
  Was ist das entsprechende risikoneutrale Maß?
  Zeige, dass $E_P[(S_1-100)^+]$ kein arbitragefreier Preis für $H=(S_1-100)^+$ (Kaufoption mit Ausübungspreis $K=100$) ist.
\end{itemize}
Das risikoneutrale Maß ist das Martingalmaß $Q$ aus Teilaufgabe (i).
Durch Einsetzen der Werte erhalten wir $d=0{,}8-1=-0{,}2$ und $u=1{,}2-1=0{,}2$, sodass $Q[\omega_u]=\frac{r-d}{u-d}=\frac{-(-0{,}2)}{0{,}4}=\frac{1}{2}=1-Q[\omega_d]=Q[\omega_d]$.
Das Maß $P$ gegeben durch $P[\omega_u]=0{,}7$ und $P[\omega_d]=0{,}3$ ist kein äquivalentes Martingalmaß für $H$.
Durch Einsetzen erhalten wir nämlich $S_1(\omega_u)=s_0(1+u)=120$ und $S_1(\omega_d)=80$, sodass $H(\omega_u)=20$ und $H(\omega_d)=0$ und damit $E_P[H]=0{,}7\cdot20+0{,}3\cdot0=14\neq S^1_0=100$.
Nach dem Fundamental Theorem of Asset Pricing ist der Preis damit auch nicht arbitragefrei.
\pagebreak
\begin{itemize}
\item [iii)] Finden Sie einen Wert $\xi_1$, so dass $x+\xi_1(S_1-S_0)=H$.
  Was ist der anfängliche Preis $x$?
  Berechnen Sie $E_Q[H]$.
  Diskutieren Sie Ihre Ergebnisse.
\end{itemize}
Um $\xi_1$ und $x$ herauszufinden, lösen wir das lineare Gleichungssystem gegeben durch $x+\xi_1(S_1(\omega_u)-s_0)=H(\omega_u)$ und $x+\xi_1(S_1(\omega_d)-s_0)=H(\omega_d)$.
Durch Einsetzen erhalten wir $x+20\xi_1=20$ und $x-20\xi_1=0$.
Durch addieren der beiden Gleichungen erhalten wir $2x=20$, sodass $x=10$ und durch abziehen der zweiten Gleichung von der ersten $40\xi_1=20$, sodass $\xi_1=\frac{1}{2}$.
Schließlich berechnen wir durch Einsetzen $E_Q[H]=0{,}5\cdot20+0{,}5\cdot0=10$.
Damit ist $Q$ ein risikoneutrales Maß für die Kaufoption mit Ausübungspreis $K=100$, falls der anfängliche Preis $x=10$ ist.
\bibliography{../../../books/wt}
\end{document}

%%% Local Variables:
%%% mode: latex
%%% ispell-local-dictionary: "german"
%%% TeX-master: t
%%% End:
