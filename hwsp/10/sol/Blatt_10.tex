\documentclass{article}
\usepackage[a4paper,margin=1.875in,top=1.375in,bottom=1.375in]{geometry}
\usepackage{amsmath,mathtools,bbm,amssymb,stmaryrd,wasysym}
\usepackage{mathrsfs}
\usepackage{babel}

\usepackage{setspace}
\doublespacing

\usepackage{fancyhdr}
\renewcommand{\headrulewidth}{0pt} 
\pagestyle{fancy}
\lhead{Blatt 10 Nicole und Evgenij}\rhead{Seite \thepage}
\fancyfoot{}

\usepackage{tikz}
\usetikzlibrary{decorations.pathreplacing,arrows}

\usepackage[numbers]{natbib}
\bibliographystyle{alphadin}
\usepackage{url}
\usepackage{hyperref}

\begin{document}

\paragraph{Aufgabe 1 \textnormal{(Black-Scholes-Modell; 4 Punkte)}.}
Zeigen Sie, dass das Se\-mi\-mar\-tin\-gal
\[X_t=X_0e^{\sigma W_t+t(\mu-\sigma^2/2)}\]
für $\mu\in\mathbb{R}$, $\sigma\in\mathbb{R}_+$ und einer Standard Brown'schen Bewegung $W$ folgende Darstellung besitzt
\[dX_t=\mu X_tdt+\sigma X_tdW_t=X_td(\mu t+\sigma W_t)\,.\]

Wir wenden die Itô-Formel auf $f(Y_t)=e^{Y_t}$ mit $Y_t=\sigma W_t+t(\mu-\sigma^2/2)$ an
Da $Y_t$ stetig ist, gilt $Y_-=Y$ und $\langle Y^c,Y^c\rangle=\langle Y\rangle$.
Mit $f'(Y_t)=f''(Y_t)=e^{Y_t}$ erhalten wir
\begin{align*}
  X_t
  &=X_0+\int_0^t X_sdY_s+\frac{1}{2}\int_0^tX_td\langle Y\rangle_s\,.
    \intertext{Durch Nachdifferenzieren, sowie $d\langle Y\rangle_s=\sigma^2ds$, \emph{was noch gezeigt werden sollte}, erhalten wir}
  &=X_0+\int_0^tX_s\sigma dW_s+\int_0^s\Bigl(\mu-\frac{\sigma^2}{2}\Bigr)ds+\frac{1}{2}\int_0^tX_s\sigma^2ds\\
  &=X_0+\int_0^tX_s\sigma dW_s+\int_0^s X_s\mu ds\,.
\end{align*}
Nach Definition 1 von Blatt 9 mit $H_t=\mu X_t$ und $K_t=\sigma X_t$ besitzt $X_t$ dann die angegebene Darstellung.

%https://en.wikipedia.org/wiki/It%C3%B4%27s_lemma

\end{document}

%%% Local Variables:
%%% mode: latex
%%% ispell-local-dictionary: "german"
%%% TeX-master: t
%%% End:
