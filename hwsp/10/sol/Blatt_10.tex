\documentclass{article}
\usepackage[a4paper,margin=1.875in,top=1.375in,bottom=1.375in]{geometry}
\usepackage{amsmath,mathtools,bbm,amssymb,stmaryrd,wasysym}
\usepackage{mathrsfs}
\usepackage{babel}

\usepackage{setspace}
\doublespacing

\usepackage{fancyhdr}
\renewcommand{\headrulewidth}{0pt} 
\pagestyle{fancy}
\lhead{Blatt 10 Nicole und Evgenij}\rhead{Seite \thepage}
\fancyfoot{}

\usepackage{tikz}
\usetikzlibrary{decorations.pathreplacing,arrows}

\usepackage[numbers]{natbib}
\bibliographystyle{alphadin}
\usepackage{url}
\usepackage{hyperref}

\begin{document}

\paragraph{Aufgabe 1 \textnormal{(Black-Scholes-Modell; 4 Punkte)}.}
Zeigen Sie, dass das Semimartingal
\[X_t=X_0e^{\sigma W_t+t(\mu-\sigma^2/2)}\]
für $\mu\in\mathbb{R}$, $\sigma\in\mathbb{R}_+$ und einer Standard Brown'schen Bewegung $W$ folgende Darstellung besitzt
\[dX_t=\mu X_tdt+\sigma X_tdW_t=X_td(\mu t+\sigma W_t)\,.\]

Da $W_t$ stetig ist, gilt $W_-=W$ und $\langle W^c,W^c\rangle=\langle W\rangle$.
\emph{Es sollte noch gezeigt werden, warum $\langle W\rangle=t$.}
Nach den Definitionen von Blatt 9 besitzt $X_t$ die gegebene Darstellung, wenn $X_t=X_0+\int_0^t\mu X_sds+\int_0^t\sigma X_sdW_s$.
Anwenden der Itô-Formel liefert
%https://en.wikipedia.org/wiki/It%C3%B4%27s_lemma

\end{document}

%%% Local Variables:
%%% mode: latex
%%% ispell-local-dictionary: "german"
%%% TeX-master: t
%%% End:
