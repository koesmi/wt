\documentclass{article}
\usepackage[a4paper,margin=1.875in,top=1.375in,bottom=1.375in]{geometry}
\usepackage{amsmath,mathtools,bbm,amssymb,stmaryrd,wasysym}
\usepackage{mathrsfs}
\usepackage{babel}

\usepackage{setspace}
\doublespacing

\usepackage{fancyhdr}
\renewcommand{\headrulewidth}{0pt} 
\pagestyle{fancy}
\lhead{Blatt 10 Nicole und Evgenij}\rhead{Seite \thepage}
\fancyfoot{}

\usepackage{tikz}
\usetikzlibrary{decorations.pathreplacing,arrows}

\usepackage[numbers]{natbib}
\bibliographystyle{alphadin}
\usepackage{url}
\usepackage{hyperref}

\begin{document}

\paragraph{Aufgabe 1 \textnormal{(Black-Scholes-Modell; 4 Punkte)}.}
Zeigen Sie, dass das Se\-mi\-mar\-tin\-gal
\[X_t=X_0e^{\sigma W_t+t(\mu-\sigma^2/2)}\]
für $\mu\in\mathbb{R}$, $\sigma\in\mathbb{R}_+$ und einer Standard Brown'schen Bewegung $W$ folgende Darstellung besitzt
\[dX_t=\mu X_tdt+\sigma X_tdW_t=X_td(\mu t+\sigma W_t)\,.\]

Wir wenden die Itô-Formel auf $f(Y_t)=e^{Y_t}$ mit $Y_t=\sigma W_t+t(\mu-\sigma^2/2)$ an
Da $Y_t$ stetig ist, gilt $Y_-=Y$ und $\langle Y^c,Y^c\rangle=\langle Y\rangle=\sigma^2t$.
% Außerdem ist $\operatorname{Var}(Y_t)=\sigma^2\operatorname{Var}(W_t)=\sigma^2t$.
Zunächst ist nämlich $\sigma^2t$ stetig, verschwindet für $t=0$ und ist wachsend, also ist $\sigma^2t\in\mathscr{V}$.
\emph{Es müsste noch gezeigt werden, dass $Y_t^2-\sigma^2t$ ein Martingal ist.}
Mit $f'(Y_t)=f''(Y_t)=e^{Y_t}=X_t$ erhalten wir
\begin{align*}
  X_t
  &=X_0+\int_0^t X_sdY_s+\frac{1}{2}\int_0^tX_sd\langle Y\rangle_s\,.
    \intertext{Durch Nachdifferenzieren, sowie $d\langle Y\rangle_s=\sigma^2ds$, erhalten wir}
  &=X_0+\int_0^tX_s\sigma dW_s+\int_0^s\Bigl(\mu-\frac{\sigma^2}{2}\Bigr)ds+\frac{1}{2}\int_0^tX_s\sigma^2ds\\
  &=X_0+\int_0^tX_s\sigma dW_s+\int_0^s X_s\mu ds\,.
\end{align*}
Nach Definition 1 von Blatt 9 mit $H_t=\mu X_t$ und $K_t=\sigma X_t$ besitzt $X_t$ dann die angegebene Darstellung.

\paragraph{Aufgabe 4 \textnormal{(4 Punkte)}.}
Sei $S$ ein lokal beschränkter càdlàg Prozess.
Die Menge $K^\text{simple}\subset L^\infty(\Omega,\mathscr{A},P)$ sei gegeben durch
\[
  K^\text{simple}:=\{(H\cdot S)_\infty\mid H=\sum_{i=1}^n h_i\mathbbm{1}_{\rrbracket\tau_{i-1},\tau_i\rrbracket}\text{ einfacher Prozess, $S^{\tau_n}$ beschränkt}\}.
\]
Weiter existiere ein Maß $Q$ mit den Eigenschaften
\begin{enumerate}
\item $Q\sim P$, d.h. $Q$ ist äquivalent zu $P$, und
\item der Prozess $S$ ist ein lokales Martingal unter $Q$.
\end{enumerate}
Sei weiter $L_+^\infty(\Omega,\mathscr{A},P):=\{f\in L^\infty(\Omega,\mathscr{A},P\mid f\geq0)\}$.
Zeigen Sie
\[
K^\text{simple}\cap L^\infty_+(\Omega,\mathscr{A},P)=\{0\}.
\]
Formulieren Sie die ökonomische Interpretation dieser Aussage.

\noindent\emph{Hinweis: Verwenden Sie Aufgabe 5.}

Das ist eine Richtung des Fundamental Theorem of Asset Pricing -- Existiert ein äquivalentes Martingalmaß $Q$, so ist der Markt frei von Arbitrage.

\paragraph{Aufgabe 5 \textnormal{(Bonus 4 Punkte)}.}
Zeigen Sie: Ein lokal beschränkter càdlàg Prozess $S$ ist ein lokales Martingal genau dann, wenn
\[
  E[(H\cdot S)_\infty]=0\,,
\]
für alle einfachen Prozesse $H=\sum_{i=1}^nh_i\mathbbm{1}_{\rrbracket \tau_{i-1},\tau_i\rrbracket}$, sodass $S^{\tau_n}$ beschränkt ist.

\noindent\emph{Hinweis: Betrachten Sie eine lokalisierende Folge von Stoppzeiten $(T_n)_{n\in\mathbb{N}}$, sodass $S^{T_n}$ ein beschränkter Prozess ist.
Die Martingaleigenschaft von $S^{T_n}$ folgt nun, falls $E[S_{\sigma_2}^{T_n}\mid \mathscr{F}_{\sigma_1}]=S_{\sigma_1}^{T_n}$ für alle Stoppzeiten $\sigma_1\leq\sigma_2\leq T_n$ (Diese Aussage muss ebenfalls gezeigt werden).}

Sei zunächst $S$ ein lokales Martingal, $H$ ein einfacher Prozess, sodass $S^{\tau_n}$ beschränkt ist.
Dann ist für die lokalisierende Folge $(T_k)_{k\in \mathbb{N}}$ von $S$ der Prozess $S^{T_k}$ ein Martingal.
Sei $k\in\mathbb{N}$ so, dass $T_k\geq\tau_n$, dann sind auch die $S^{\tau_i}$ Martingale.
Nach Definition 2 von Blatt 3 gilt $(H\cdot S)_t=\sum_{i=1}^nh_i(S_t^{\tau_i}-S_t^{\tau_{i-1}})$ mit $h_i\in L^\infty(\mathscr{F}_{\tau_{i-1}})$.
Somit gilt für $t=\infty$, dass $(H\cdot S)_\infty=\sum_{i=1}^nh_i(S_{\tau_i}-S_{\tau_{i-1}})$.
Entsprechend dem Beweis von Theorem 216 im Skript zur Vorlesung Wahr\-schein\-lich\-keits\-the\-o\-rie gilt wegen der Definition des stochastischen Integrals für einfache Prozesse
\begin{align*}
  E[(H\cdot S)_\infty]
  &=\sum_{i=1}^nE_Q[h_i(S_{\tau_i}-S_{\tau_{i-1}})]\,.
    \intertext{Mit der Turmeigenschaft der bedingten Erwartung erhalten wir}
  &=\sum_{i=1}^nE\bigl[E[h_i(S_{\tau_i}-S_{\tau_{i-1}})\mid\mathscr{F}_{\tau_{i-1}}]\bigr]\,.
    \intertext{Da die $h_i$ jeweils $\mathscr{F}_{\tau_{i-1}}$-messbar sind, folgt}
  &=\sum_{i=1}^nE\bigl[h_iE[S_{\tau_i}-S_{\tau_{i-1}}\mid\mathscr{F}_{\tau_{i-1}}]\bigr]\,.
    \intertext{Da $S^{T_k}$ ein Martingal ist, erhalten wir schließlich}
  &=\sum_{i=1}^nE\bigl[h_i(S_{\tau_{i-1}}-S_{\tau_{i-1}})\bigr]=0\,.
\end{align*}
% Dann ist $(H\cdot S)$ auch ein lokales Martingal, denn $S^{\tau_n}$ ist beschränkt.
%\emph{Das sollte noch genauer ausgeführt werden}.
%Somit ist $E_Q[(H\cdot S)_\infty]=E_Q[(H\cdot S)_\infty\mid \mathscr{F}_0]=(H\cdot X)_0=0$.

Es gelte andererseits $E[(H\cdot S)_\infty]=0$ für alle einfachen Prozesse $H$, sodass $S^{\tau_n}$ beschränkt ist.
Wir wollen zeigen, dass $S$ ein lokales Martingal ist, also eine Folge von Stoppzeiten $(T_n)_{n\in\mathbb{N}}$ finden, sodass $S^{T_n}$ für jedes $n\in\mathbb{N}$ ein Martingal ist.
Da $S$ lokal beschränkt ist gibt es eine Folge $(T_n)_{n\in\mathbb{N}}$, sodass $S^{T_n}$ beschränkt ist.
Gilt nun $E[S_{\sigma_2}^{T_n}\mid \mathscr{F}_{\sigma_1}]=S^{T_n}_{\sigma_1}$ für alle $\sigma_1\leq\sigma_2\leq T_n$, so gilt auch $E[S^{T_n}_t\mid\mathscr{F}_{\sigma_1}]=S^{T_n}_{\sigma_1}$ für alle $t>T_n$, denn hier gilt $S^{T_n}_t=S_{T_n}$.
Somit ist $S^{T_n}$ dann ein Martingal.
\emph{Es sollte noch gezeigt werden, dass $S^{T_n}$ Prozess gleichgradig integrierbar ist.}
\emph{Außerdem sollte noch gezeigt werden, dass wenn für alle einfachen Prozesse $H$ sodass $S^{\tau_n}$ beschränkt ist gilt $E[(H\cdot S)_\infty]=0$, wir $E[S^{T_n}_{\sigma_2}\mid\mathscr{F}_{\sigma_1}]=S^{T_n}_{\sigma_1}$ erhalten.}
\end{document}

%%% Local Variables:
%%% mode: latex
%%% ispell-local-dictionary: "german"
%%% TeX-master: t
%%% End:
